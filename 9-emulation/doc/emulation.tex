%-------------------------
%makefile
%(c) H.Buchmann FHNW 2014
%export TEXINPUTS=${HOME}/fhnw/edu/:${HOME}/fhnw/edu/tinL/config/latex:${HOME}/fhnw/edu/config//:
%-------------------------
\documentclass{beamer}
\usepackage{latex/beamer}
%---------------------
%local defines
%(c) H.Buchmann FHNW 2009
%$Id$
%---------------------
\newcommand{\target} {\beaglebone\xspace}
\newcommand{\targetS}{{\bf BBG}\xspace}
\newcommand{\host}   {{\em Host}\xspace}
\newcommand{\targetroot} {{\bf target-root}\xspace}
\newcommand{\kernel} {{\bf kernel}\xspace}
\renewcommand{\c}{{\bf C}\xspace}
\newcommand{\cpp}{{\bf C++}\xspace}
\newcommand{\posix}{{\bf POSIX}\xspace}

\input{/home/buchmann/latex/dirtree/dirtree.tex}

\usepackage[absolute]{textpos}
\setlength{\TPHorizModule}{1mm}
\setlength{\TPVertModule}{1mm}

\begin{document}

\newcommand{\qemu}{{\em qemu}\xspace}

\title[Emulation]{Emulation\\mit \qemu}

\frame{\titlepage}

\begin{frame}{Um was geht es ?}{VM: Virtuelle Maschine}
\begin{itemize}
 \item VM: Virtuelle Maschine 
 \item \url{qemu.org} unsere Maschine
\end{itemize}
\end{frame}

\begin{frame}{Terminologie}
 \begin{description}
  \item[Host] die Workstation
  \begin{itemize}
   \item mit \linux
  \end{itemize}
  \item[VM] virtuelle Maschine:
   \begin{itemize}
    \item Programm auf dem {\em Host}
   \end{itemize}
  \item[Gast] Code der auf der {\em VM} l�uft
 \end{description}
\end{frame}

\begin{frame}[fragile]{Ein atypisches Beispiel}{mit \qemu}
 \begin{description}
  \item[Host] \unix
  \item[VM] \qemu
  \item[Gast] \cod{hello-world} 
 \end{description} 
 \begin{block}{Aufruf}
  {\footnotesize
  \begin{verbatim}
qemu-system-arm -M realview-eb -kernel hello-world -serial stdio
  \end{verbatim}
  }
  \end{block}
\end{frame}

\begin{frame}{Aufgaben}
 \begin{description}
  \item[QEMU] Aus den {\em sourcen} 
  \begin{itemize}
   \item als �bung
  \end{itemize}
  \item[Kernel] f�r den Gast
  \begin{itemize}
   \item {\tiny\url{http://xecdesign.com/downloads/linux-qemu/kernel-qemu}}
   \item Eigener kernel
  \end{itemize}
  \item[RootFS] als (virtuelle) Partition 
  \remark{notes.txt}
  \item[Net] Verbindung
 \end{description}
 \remark{\cod{src/qemu.sh}}
\end{frame}
\end{document}
