%-------------------------
%minimal-unix
%(c) H.Buchmann FHNW 2014
%export TEXINPUTS=${HOME}/fhnw/edu/:${HOME}/fhnw/edu/tinL/config/latex:${HOME}/fhnw/edu/config//:
%-------------------------
\documentclass{beamer}
\usepackage{latex/beamer}
%---------------------
%local defines
%(c) H.Buchmann FHNW 2009
%$Id$
%---------------------
\newcommand{\target} {\beaglebone\xspace}
\newcommand{\targetS}{{\bf BBG}\xspace}
\newcommand{\host}   {{\em Host}\xspace}
\newcommand{\targetroot} {{\bf target-root}\xspace}
\newcommand{\kernel} {{\bf kernel}\xspace}
\renewcommand{\c}{{\bf C}\xspace}
\newcommand{\cpp}{{\bf C++}\xspace}
\newcommand{\posix}{{\bf POSIX}\xspace}

\input{/home/buchmann/latex/dirtree/dirtree.tex}

\usepackage[absolute]{textpos}
\setlength{\TPHorizModule}{1mm}
\setlength{\TPVertModule}{1mm}

\begin{document}

\newcommand{\qemu}{{\em qemu}\xspace}
\newcommand{\busybox}{{\em busybox}\xspace}
\newcommand{\yocto}{{\em yocto}\xspace}
\title{Cross Toolchain}

\frame{\titlepage}
\begin{frame}{Um was geht es ?}{Eigene neueste Toolchain f�r \raspberry}
 \begin{itemize}
  \item Eigene Toolchain aus den Sourcen
  \item Vorteil
  \begin{itemize}
   \item Toolchain auf den neuesten Stand
  \end{itemize}
  \item Nachteil
  \begin{itemize}
   \item  Muss selber konfiguriert werden
  \end{itemize}
 \end{itemize}
\end{frame}

\begin{frame}{Toolchain}
 \begin{itemize}
  \item die grossen zwei
  \begin{itemize}
  \item Compiler
  \item Linker
  \end{itemize}
  \item kleinere Programme
  \begin{itemize}
   \item Assembler
   \item ...
  \end{itemize}
 \end{itemize}
\end{frame}

\begin{frame}{Toolchain}{Beispiel}
 \begin{itemize}
  \item Sourcefile \cod{source.c}
  \item Compilat/object File \cod{source.o}
  \item Executable/Image \cod{source}
 \end{itemize}
\end{frame}

\begin{frame}{Cross toolchain}{2 Verschiedene Rechner}
\begin{description}[Cross\{Programm\}]
 \item[Host] Workstation leistungsf�higer Rechner
 \item[Target] Eingebettetes System (\raspberry)
 \item[Cross\{Programm\}] Programm (Compiler etc.) das
 \begin{itemize}
  \item l�uft auf dem {\em Host} und erzeugt Files f�r das {\em Target}
 \end{itemize}
\end{description}
\end{frame}

\begin{frame}{Cross toolchain}
 \begin{itemize}
  \item erzeugt auf dem {\em Host} Programme f�r das {\em Target}
 \end{itemize}
\end{frame}

\section{Build: GNU/Toolchain}
\begin{frame}{GNU/Toolchain}{Zwei Komponenten}
\begin{description}[binutils]
 \item[binutils] Linker, assembler, ...
 \item[gcc] Compiler
\end{description}
\end{frame}

\begin{frame}{Build}{die drei Schritte}
 \begin{itemize}
  \item configure
  \item make
  \item make install
 \end{itemize}
 \remark{auf dem {\Huge Host}}
\end{frame}

\begin{frame}{Build}{der Kontext}
 \begin{description}
  \item[prefix] wo die Toolchain auf dem {\em Host} installiert wird
  \begin{itemize}
   \item option \cod{--prefix={\em path-to-toolchain-install}} 
  \end{itemize}
  \item[sysroot] wo ist das {\em Target} root system (auf dem {\em Host})
  \begin{itemize}
   \item option \cod{--with-sysroot={\em path-to-target-sysroot}}
  \end{itemize}
  \item[target] was f�r eine {\em Target} System
  \begin{itemize}
   \item option \cod{--target=armv6l-unknown-linux-gnueabihf}
   \remark{Warum ???}
  \end{itemize}
 \end{description}
\end{frame}

\begin{frame}{Verzeichnisstruktur: Host}{Build}
 \dirtree{%
  .1 12-toolchain.
  .2 {..674c09cf-ac29-40e4-a768-5608a462a3e1} \DTcomment {tag}.
  .2 scripts.
  .3 common.sh \DTcomment{used for binutils \& gcc}.
  .3 binutils.sh.
  .3 gcc.sh.
  .2 build \DTcomment{for building}.
  .3 binutils.
  .3 gcc.
  .2 tc \DTcomment{the products}.
 }
\end{frame}

\begin{frame}{Verzeichnisstruktur: Host}{Cross-development}
 \dirtree{%
  .1 12-toolchain.
  .2 {..674c09cf-ac29-40e4-a768-5608a462a3e1} \DTcomment {tag}.
  .2 src.
  .2 tc \DTcomment{the products}.
  .2 config.
  .3 Makefile.
  .2 work \DTcomment{here we are/connected with \target}.
  .3 {$\to$ ../config/Makefile} \DTcomment{a link}.
 }
\end{frame}

\begin{frame}{Test}
 \begin{description}[\cpp]
  \item[\c] \cod{c-source.c}
  \item[\cpp] \cod{cc-source.cc}
 \end{description}
\end{frame}

\end{document}
