%-------------------------
%exercises.tex
%(c) H.Buchmann FHNW 2019
%export TEXINPUTS=${HOME}/fhnw/edu/tinL/config/latex:${HOME}/fhnw/edu/config//:
%-------------------------
\documentclass{beamer}
\usepackage{latex/beamer}
%---------------------
%local defines
%(c) H.Buchmann FHNW 2009
%$Id$
%---------------------
\newcommand{\target} {\beaglebone\xspace}
\newcommand{\targetS}{{\bf BBG}\xspace}
\newcommand{\host}   {{\em Host}\xspace}
\newcommand{\targetroot} {{\bf target-root}\xspace}
\newcommand{\kernel} {{\bf kernel}\xspace}
\renewcommand{\c}{{\bf C}\xspace}
\newcommand{\cpp}{{\bf C++}\xspace}
\newcommand{\posix}{{\bf POSIX}\xspace}

\begin{document}

\begin{frame}{20.Feb.2020}
\begin{itemize}
 \item Umgebung aufsetzen
 \begin{itemize}
  \item Linux
  \item git repo
 \end{itemize}
\item \target verbinden mit \host
 \begin{itemize}
  \item per USB
  \begin{itemize}
   \item serielle Schnittstelle 
   \item Internet
  \end{itemize}
  \item per WLAN
 \end{itemize}
\item sich auf dem \target zurechtfinden
\end{itemize}
\end{frame}

\begin{frame}{27.Feb.2020}
 \begin{itemize}
  \item Alles ist ein File
  \begin{itemize}
   \item \cod{sshfs} mount
   \item \cod{mount} SD-Karte
  \end{itemize}
  \item Crossdevelopment
  \begin{itemize}
   \item auf dem \targetS
  \end{itemize}
 \end{itemize}
\end{frame}

\begin{frame}{5.März 2020}
 \begin{itemize}
  \item \targetS kleines Image auf SD-Karte
  \item Zugriff via serielle Schnittstelle
  \item via USB am lokalen Netz
  \begin{itemize}
    \item ssh \& sshfs
  \end{itemize}
  \item via Wi-Fi am Internet
 \end{itemize}
\end{frame}
%
\begin{frame}{12.März 2020}
 \begin{itemize}
  \item 5-kernel
  \begin{itemize}
   \item basic config
   \item USB Gadget
   \item WiFi \& firmware
  \end{itemize}
 \end{itemize}
\end{frame}
%
%\begin{frame}{16.Okt. 2019}
% \begin{itemize}
%  \item 6-crossdevelopment
%  \begin{itemize}
%   \item Programme in \cod{src} auf \host \& \targetS
%   \item Vergleich Zeit von \cod{primes} auf \host \& \targetS
%  \end{itemize}
% \end{itemize}
%\end{frame}
%
%\begin{frame}{23.Okt. 2019}
% \begin{itemize}
%  \item 6-crossdevelopment Zugriff auf die Hardware
%  \begin{itemize}
%   \item via Skript \cod{/sys/class/gpio}
%   \begin{itemize}
%    \item \cod{led-enable.sh}, \cod{led-blink.sh}
%   \end{itemize}
%   \item via C++ \cod{/sys/class/gpio}
%   \begin{itemize}
%    \item \cod{led-enable.cc}, \cod{led-blink.cc}
%   \end{itemize}
%   \item direkt \cod{src/mem.h$|$cc}
%   \begin{itemize}
%    \item \cod{led-direct-0.cc}, \cod{led-direct-1.cc}
%   \end{itemize}
%  \end{itemize}
% \end{itemize}
%\end{frame}
%
%\begin{frame}{30.Okt.2019}
% \begin{itemize}
%  \item 6-crossdevelopment Zugriff auf die andere Hardware
%  \begin{itemize}
%   \item SWITCH $\to$ LED
%  \end{itemize}
%  \item 4-u-boot
%  \begin{itemize}
%   \item Herstellung
%   \item Installation
%  \end{itemize}
% \end{itemize}
%\end{frame}
%
%\begin{frame}{6.Nov.2019}{Zugriff auf die Hardware via Internet}
% \begin{itemize}
%  \item HTTP Server \cod{lighttpd}
%  \item CGI Common Gateway Interface
% \end{itemize}
%\end{frame}
%
%
%\begin{frame}{13.Nov 2019}{Build von Grund auf}
% \begin{itemize}
%  \item bestehendes {\em rootfs} nicht verlieren
%  \item einfache toolchain *)
%  \item kernel             *)
%  \item test mit \cod{src/s-bare-init.s}
%  \item glibc {\em POSIX}
%  \item volle toolchain    *)
%  \item test mit \cod{src/cpp-hello-world.cc}
%  \item busybox
%  \item ssh
%  \item wpa
% \end{itemize}
% *) fakultativ
%\end{frame}
%
%\begin{frame}{20.Nov.2019}{Build,rootfs flavours}
% \begin{itemize}
%  \item build
%  \begin{itemize}
%   \item ssh
%   \item wifi
%  \end{itemize}
%  \item roofs flavours
%  \begin{itemize}
%   \item nano
%   \item mini
%   \item full
%  \end{itemize}
% \end{itemize}
%\end{frame}
%
%\begin{frame}{4.Dez.2019}{Auteilung \cod{target-root}}
% \begin{itemize}
% \item einen Teile auf dem \host bzw. \target 
% \begin{itemize}
%  \item einfaches C-Programm
%  \item einfaches C++-Programm
%  \item komplexes C++-Programm
% \end{itemize}
% \end{itemize}
%\end{frame}
%
%\begin{frame}{11.Dez.2019}{Konsolidierung/Kernel-Userspace}
% \begin{itemize}
%  \item Konsolidierung
%  \begin{itemize}
%   \item aktuelle Toolchain
%   \item aktueller Kern
%   \begin{itemize}
%    \item mit WiFi
%   \end{itemize}
%   \item aktueller rootfs
%   \begin{itemize}
%    \item ssh
%    \item WiFi
%   \end{itemize} 
%  \end{itemize}
%  \item  Kernel-Userspace
%  \begin{itemize}
%   \item hotplug
%   \item socket NETLINK KOBJECT UEVENT
%  \end{itemize}
%  
% \end{itemize}
% 
%\end{frame}
%
%\begin{frame}{18.Dez.2019}{Kernel-Userspace}
% \begin{itemize}
%  \item hotplug
%  \item socket NETLINK KOBJECT UEVENT
% \end{itemize}
%\end{frame}
%
%\begin{frame}{8.Jan.2020}{Resume}
% \begin{itemize}
%  \item zusätzliche {\em packages} \cod{iw}, \cod{nano}
%  \item Separation
%  \begin{itemize}
%   \item Host {\bf user}
%   \item Target {\bf root}
%  \end{itemize}
%  \item Kernel Konfiguration
%  \begin{itemize}
%   \item \cod{omap\_i2c 4819c000.i2c: timeout waiting for bus ready}
%  \end{itemize}
% \end{itemize}
%\end{frame}
\end{document}
