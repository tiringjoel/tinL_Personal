%-------------------------
%exercises.tex
%(c) H.Buchmann FHNW 2019
%export TEXINPUTS=${HOME}/fhnw/edu/tinL/config/latex:${HOME}/fhnw/edu/config//:
%-------------------------
\documentclass{beamer}
\usepackage{latex/beamer}
%---------------------
%local defines
%(c) H.Buchmann FHNW 2009
%$Id$
%---------------------
\def \target {\raspberry\xspace}
\def \host {{\em Host \xspace}}

\begin{document}

\begin{frame}{18.Sept.2019}
\begin{itemize}
 \item Umgebung aufsetzen
 \begin{itemize}
  \item Linux
  \item git repo
 \end{itemize}
\item \target verbinden mit \host
 \begin{itemize}
  \item per USB
  \begin{itemize}
   \item serielle Schnittstelle 
   \item Internet
  \end{itemize}
  \item per WLAN
 \end{itemize}
\item sich auf dem \target zurechtfinden
\end{itemize}
\end{frame}

\begin{frame}{25.Sept.2019}
 \begin{itemize}
  \item Alles ist ein File
  \begin{itemize}
   \item \cod{sshfs} mount
   \item \cod{mount} SD-Karte
  \end{itemize}
  \item Netzwerk
  \begin{itemize}
   \item Host als Proxy
   \item Host als Gateway/Router
   \item \targetS via WiFi
  \end{itemize}
 \end{itemize}
\end{frame}

\begin{frame}{2.Okt. 2019}
 \begin{itemize}
  \item \targetS kleines Image auf SD-Karte
  \item Zugriff via serielle Schnittstelle
  \item via USB am lokalen Netz
  \begin{itemize}
    \item ssh \& sshfs
  \end{itemize}
  \item via Wi-Fi am Internet
 \end{itemize}
\end{frame}

\begin{frame}{9.Okt. 2019}
 \begin{itemize}
  \item init script für \targetS
  \item 5-kernel
  \begin{itemize}
   \item basic config
   \item USB Gadget
   \item WiFi \& firmware
  \end{itemize}
 \end{itemize}
\end{frame}

\begin{frame}{16.Okt. 2019}
 \begin{itemize}
  \item 6-crossdevelopment
  \begin{itemize}
   \item Programme in \cod{src} auf \host \& \targetS
   \item Vergleich Zeit von \cod{primes} auf \host \& \targetS
  \end{itemize}
 \end{itemize}
\end{frame}
\end{document}
