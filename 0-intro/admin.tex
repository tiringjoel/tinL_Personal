%-------------------------
%admin
%(c) H.Buchmann FHNW 2008
%$Id$
%-------------------------
\title[Admin]{Admin}
\frame{\titlepage}

\begin{frame}{Admin}
\begin{description}
 \item[Folien/Code] \url{websvn.fhnw.ch/trac/edu/browser/linux-lab}
 \item[Pr�fung] m�ndliche MSP
\end{description}
\end{frame}

\begin{frame}{Copyright}
 \begin{itemize}
  \item Alles �ffentlich zug�ngliche Material zu dieser Vorlesung unterliegt
  der {\em GNU GENERAL PUBLIC LICENSE}, auch wenn das in den einzelnen
  Dokumenten nicht explizit angegeben ist. 
  \item
  \url{http://www.gnu.org/copyleft/gpl.html}
 \end{itemize}
\end{frame}

\begin{frame}{Wichtig}
 \begin{itemize}
  \item \linux from Scratch
  \item Einblick in die Mechanismen
  \item Umgang mit verschiedenen Tools
  \item Weniger programmieren, mehr konfigurieren
  \item Schrittweises Vorgehen: (fast) immer lauff�higes System
 \end{itemize}
\end{frame}

\begin{frame}{Laborbuch}
 \begin{itemize}
  \item F�hren Sie ein Laborbuch bzw. Laborfile
 \end{itemize}
\end{frame}

\begin{frame}{Das Board SAM-9260}{Beachten Sie:}
 \begin{itemize}
  \item Ausleihe
  \item Spannung 4.5-6 Volt: {\Huge nicht mehr}
 \end{itemize}
\end{frame}
