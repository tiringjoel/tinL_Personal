%-------------------------
%development-environment
%(c) H.Buchmann FHNW 2008
%$Id$
%-------------------------

\title[Entwicklungsumgebung]{Entwicklungsumgebung}
\frame{\titlepage}

\section{Aufbau der Vorlesung/Labor}
\begin{frame}{Aufbau}{\url{websvn.fhnw.ch/trac/edu/browser/linux-lab/1-current}}
 \begin{description}[2-unix-use]
  \item[0-intro] Diese Folien
  \item[1-setup] \raspberry in Betrieb
  \item[2-unix-use] \unix aus Benutzersicht: Host und Target
  \item ...
%  \item[3-uboot] Wie startet ein Rechner
%  \item[4-kernel] Das \linux: Konfiguration/Herstellung
%  \item[5-libc] Verbindung \linux \unix
%  \item[6-unix] \unix Konfiguration/Herstellung
%  \item[7-build] ein {\em build}  System
 \end{description}
\end{frame}

\begin{frame}{Verzeichnisstruktur}
 \dirtree{%
  .1 tinL .
  .2 {\bf i}-topic \DTcomment{git versioniert}.
  .3 doc. 
  .3 src.
  .2 devel \DTcomment{development: git versioniert}.
  .2 resources \DTcomment{binaries {\bf nicht versioniert}}.
  .2 ziele.txt \DTcomment{pro Woche}.
  .2 tools.txt \DTcomment{wichtige \unix Befehle}.
  .2 {...}.
 }
\end{frame}

\section{Begriffe}
\begin{frame}
 \frametitle{Begriffe}
 \begin{description}[Target]
  \item[Host] Entwicklungsrechner, \linux Betriebssystem
  \item[Target] \raspberry
 \end{description}
\end{frame}

\begin{frame}
 \frametitle{Verbindungen:Host $\leftrightarrow$ Target}
 \begin{description}[MemoryCard]
  \item[RS232] u-boot {\em shell},\linux console
  \item[Ethernet] IP,TFTP etc. IP Stack
  \item[MemoryCard] u-boot,Kernel,\unix
 \end{description}
\end{frame}

\section{Tools}
\begin{frame}{Tools}
 \begin{itemize}
  \item \unix Befehle
  \item (Unvollst�ndige) Liste im File \cod{tools.txt}
 \end{itemize}
\end{frame}

%\begin{frame}
% \frametitle{Host:Tools}
% \begin{description}[Codeverwaltung]
%  \item[Editor] ASCII Editor
%  \item[Compiler] \cod{gcc}
%  \item[mkimage] Erzeugt u-boot Image
%  \item[Terminal] \cod{minicom} ev. USB $\leftrightarrow$ RS232 Kabel
%  \item[\unix] Basibefehle \cod{ls},\cod{cp}, etc.
%  \item[Codeverwaltung] \cod{svn} \cod{cvs} \cod{git} etc.
%  \item[Eclipse] kommt sp�ter
% \end{description}
%\end{frame}
%
%\section{Verzeichnisstruktur}
%\begin{frame}
% \frametitle{Verzeichnisstruktur}
% Umgebungsvariable \cod{BUILD\_HOME}
% \\
%{\small 
% \dirtree{%
%  .1 \envref{BUILD\_HOME} .
%  .2 u-boot \DTcomment{scripts and environment}.
%  .2 kernel \DTcomment{see kernel}. 
%  .2 libc   \DTcomment{for building}.
%  .2 unix   \DTcomment{the target \unix}.
%  .2 tc     \DTcomment{host toolchain}.
%  .2 devel  \DTcomment{for unix}.
%  .2 script \DTcomment{host scripts}.
%%  .2 3rdparty \DTcomment{other host tools}.
% }
%}
%\end{frame}
%
%\section{Vorbereitung}
%\begin{frame}
% \frametitle{Umgebungsvariablen}
% \begin{itemize}
%  \item \link{script/set-environment.sh}
%  \item \cod{source script/set-environment.sh}
% \end{itemize}
%\end{frame}
%
