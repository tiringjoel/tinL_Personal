%-------------------------
%big-picture
%(c) H.Buchmann FHNW 2008
%$Id$
%-------------------------
\title[The Big Picture]{The Big Picture}
\frame{\titlepage}

\begin{frame}{The Big Picture}
 \begin{itemize}
  \item \linux ist:
  \begin{itemize}
   \item Software mit klassischen Methoden hergestellt
   \item gross
   \item komplex {\tiny nicht kompliziert}
  \end{itemize}
  \item Darum:
  \begin{itemize}
   \item \alert{Die grundlegenden Mechanismen beachten}
   \item \alert{�bersicht bewahren}
   \item \alert{Verzeichnisstrukturen}: wo ist was.
   \end{itemize}
  \end{itemize}
\end{frame}

\begin{frame}{Ein paar Daten: zum \linux (Kernel)}
 \begin{itemize}
  \item $\approx 17M$ SLOC (Source Lines of Code)
  \item $\approx 4.3K$ Verzeichnisse
  \item $\approx 62K$ Files davon
  \begin{itemize}
   \item $\approx 46 K$ \cod{\{c$|$h\}}-Files
   \item $\approx 6 K$ Assembler Files
   \item $\approx 2 K$ Makefiles
   \item Rest: \cod{Makefile}, Scripts etc.
  \end{itemize}
 \end{itemize}
 \begin{remarks}
  \item $M=10^6$ $K=10^3$
  \item Gemacht mit \cod{sloccount} 
  \begin{itemize}
   \item Siehe \cod{tools/sloc-create.sh} und \cod{tools/sloc-analyze.sh}
  \end{itemize}
 \end{remarks}
\end{frame}

\section{ProgrammierSprachen}
\begin{frame}{Die ProgrammierSprachen}
 \begin{description}[Assembler]
  \item[C] 
     Unabh�ngig von Rechnerarchitektur,
     Hauptsprache f�r {\em Bootloader},{\em Kernel},{\em libc}
  \item[Assembler] F�r kleine Anpassungen
  \item[Skript] F�r Routineaufgaben
  \item[Makefile] F�r den Zusammenbau
 \end{description}
\end{frame}

\section{Tools}
\begin{frame}{Die wichtigsten Werkzeuge}
 \begin{description}
  \item[Compiler]  \cod{gcc} \url{gcc.gnu.org} 
  \item[binutils] Sammlung von Programmen\footnote{Liste nicht vollst�ndig} 
     (\url{www.gnu.org/software/binutils})
 \begin{description}
  \item[Assembler] \cod{as} 
  \item[Linker]    \cod{ld}
  \end{description}
  \item[Maker] \cod{make} \url{www.gnu.org/software/make}
 \end{description}
\end{frame}

\section{Komponenten}
\begin{frame}{Die Komponenten}
 \begin{description}[BootLoader]
  \item[BootLoader] {\em reset} Handler, SingleUser
  \item[Kernel] Prozessverwaltung,Treibersammlung
  \item[libc] Normierte (POSIX) Schnittstelle,Kernel-\unix 
  \item[\unix] Filesystem, Sammlung von Programmen und Daten
 \end{description}
\end{frame}

\begin{frame}{Die Komponenten:Eigenschaften}
 Komponenten lassen sich:
 \begin{itemize}
  \item einzeln hergestellen
  \item kombinieren 
  \item austauschen
 \end{itemize}
\end{frame}

\begin{frame}{Die Komponenten:Wo sind sie ?}
 \begin{description}[BootLoader]
  \item[BootLoader] nicht fl�chtiger Speicher: z.B. Flash
  \item[Kernel] RAM
  \item[libc] RAM
  \item[\unix] RAM,Harddisk,MemoryCard,Netz 
 \end{description}
\end{frame}

%\section{Zusammenbau}
%\subsection{Bestehende Systeme}
%\begin{frame}{Bestehende Systeme}
% \begin{description}[openEmbedded]
%  \item[openEmbedded] \url{www.openembedded.org}
%  
%  (\'a la Gentoo) Skriptsammlung/Repository, f�r viele
%  Rechnerarchitekturen
%
%  \item[buildroot] \url{buildroot.uclibc.org}
%  
%      Makefiles/patches f�r viele Rechnerarchitekturen
%      
%  \item[CLFS] \url{cross-lfs.org/view/clfs-embedded/arm} 
%  
%   {\bf C}ross-Compiled {\bf L}inux {\bf F}rom {\bf S}cratch 
%
%   Anleitung 
% \end{description}
%\end{frame}
%
%\begin{frame}{Bestehende Systeme:Kritik}
% \begin{description}
%  \item[openEmbedded/buildroot] 
%  \begin{itemize}
%   \item Grosse Systeme
%   \item Braucht zus�tzliche {\em tools}
%   \item F�hren zus�tzliche Komplikationen ein 
%  \end{itemize}      
%  \item[CLFS] Sehr rezeptartiger Aufbau     
% \end{description}
%\end{frame}
%
%\subsection{Unser System}
%\begin{frame}{Unser System: Modifiziertes CLFS}
% Warum ?
% \begin{itemize}
%  \item Basiert auf der originalen Software
%  \item Die klassischen Tools (\cod{Makefile}) sind schon sehr gut
%  ausgebaut.
%  \item Brauchen tieferen Einblick in das ganze System
%  \item Nur ein bis zwei Rechnerarchitekturen 
% \end{itemize} 
%\end{frame}

