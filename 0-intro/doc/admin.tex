%-------------------------
%admin
%(c) H.Buchmann FHNW 2008
%$Id$
%-------------------------
\title[Admin]{Admin}
\frame{\titlepage}

\begin{frame}{Admin}
\begin{description}[Folien/Code]
 \item[git]{\tiny\url{gitlab.fhnw.ch/edu/tinL.git}}
 \begin{itemize}
  \item \cod{git clone https://gitlab.fhnw.ch/edu/tinL.git}
 \end{itemize}
 \item[Pr�fung] m�ndliche MSP
\end{description}
\end{frame}

\begin{frame}{Copyright}
 \begin{itemize}
  \item Alles �ffentlich zug�ngliche Material zu dieser Vorlesung unterliegt
  der {\em GNU GENERAL PUBLIC LICENSE}, auch wenn das in den einzelnen
  Dokumenten nicht explizit angegeben ist. 
  \item
  \url{www.gnu.org/copyleft/gpl.html}
 \end{itemize}
\end{frame}

\begin{frame}{Wichtig}
 \begin{itemize}
  \item \linux auf dem \beaglebone
  \item Einblick in die Mechanismen
  \item Umgang mit verschiedenen Tools
  \item Weniger programmieren, mehr konfigurieren
  \item Schrittweises Vorgehen: (fast) immer lauff�higes System
 \end{itemize}
\end{frame}

\begin{frame}{Laborbuch}
 \begin{itemize}
  \item F�hren Sie ein Laborbuch bzw. Laborfile
 \end{itemize}
\end{frame}

