%-------------------------
%development-environment
%(c) H.Buchmann FHNW 2008
%$Id$
%-------------------------

\title[Entwicklungsumgebung]{Entwicklungsumgebung\\{\tiny \svnId}}
\frame{\titlepage}

\section{Begriffe}
\begin{frame}
 \frametitle{Begriffe}
 \begin{description}[Target]
  \item[Host] Entwicklungsrechner, \linux Betriebssystem
  \item[Target] \target
 \end{description}
\end{frame}

\begin{frame}
 \frametitle{Verbindungen:Host $\leftrightarrow$ Target}
 \begin{description}[MemoryCard]
  \item[RS232] u-boot {\em shell},\linux console
  \item[Ethernet] IP,TFTP etc. IP Stack
  \item[MemoryCard] u-boot,Kernel,\unix
 \end{description}
\end{frame}

\section{Tools}

\begin{frame}
 \frametitle{Host:Tools}
 \begin{description}[Codeverwaltung]
  \item[Editor] ASCII Editor,\cod{joe},\cod{nedit} etc. etc.
  \item[Compiler] \cod{gcc}
  \item[mkimage] Erzeugt u-boot Image
  \item[Terminal] \cod{minicom} ev. USB $\leftrightarrow$ RS232 Kabel
  \item[\unix] Basibefehle \cod{ls},\cod{cp}, etc.
  \item[Codeverwaltung] \cod{svn} \cod{cvs} \cod{git} etc.
  \item[Eclipse] kommt sp�ter
 \end{description}
\end{frame}

%\section{Verzeichnisstruktur}
%\begin{frame}
% \frametitle{Verzeichnisstruktur}
% Umgebungsvariable \cod{BUILD\_HOME}
% \\
%{\small 
% \dirtree{%
%  .1 \envref{BUILD\_HOME} .
%  .2 u-boot \DTcomment{scripts and environment}.
%  .2 kernel \DTcomment{see kernel}. 
%  .2 libc   \DTcomment{for building}.
%  .2 unix   \DTcomment{the target \unix}.
%  .2 tc     \DTcomment{host toolchain}.
%  .2 devel  \DTcomment{for unix}.
%  .2 script \DTcomment{host scripts}.
%%  .2 3rdparty \DTcomment{other host tools}.
% }
%}
%\end{frame}
%
%\section{Vorbereitung}
%\begin{frame}
% \frametitle{Umgebungsvariablen}
% \begin{itemize}
%  \item \link{script/set-environment.sh}
%  \item \cod{source script/set-environment.sh}
% \end{itemize}
%\end{frame}
%
