\section{Aufgaben}
\begin{frame}{Toolchain}
 Die {\em toolchain} kommt �berall vor und wird per {\em link} in die einzelnen Verzeichnisse
 importiert \cod{ln -s}.
 \begin{itemize}
  \item Machen Sie einen eignen Abschnitt {\em toolchain-bare}. Benutzen Sie dazu
  \begin{itemize}
   \item \cod{17-build/tools/binutils.sh}
   \item \cod{17-build/tools/gcc-bare.sh}
  \end{itemize}
 \end{itemize}
\end{frame}

\begin{frame}{Kernel: in Partition 2}
Die beiden Files: {\em KernelImage} {\em DeviceTree} k�nnen auch auf Partition 2
liegen.
 \begin{itemize}
  \item Legen Sie das {\em KernelImage}/{\em DeviceTree} auf Partition 2
  \item Passen Sie \cod{u-boot} an
  \begin{itemize}
   \item Umgebungsvariable \cod{kernel}
  \end{itemize}
 \end{itemize}
\end{frame}

\begin{frame}{LibC}
Neben der \cod{glibc} gibt es noch weitere \cod{libc}'s. 
\url{www.musl-libc.org/} ist eine kleine \cod{libc}.
\begin{itemize}
 \item erzeugen Sie an Stelle von \cod{glibc} \cod{musl}
 
 (Siehe \cod{17-build/tools/musl.sh})
\end{itemize} 

\end{frame}

\begin{frame}{Toybox}
 Neben \cod{busybox} gibt es noch ein kleineres \unix: 
  \url[http]{www.landley.net/toybox}.
 \begin{itemize}
  \item erzeugen Sie an Stelle von \cod{busybox} \cod{toybox}
  
  (Siehe \cod{17-build/tools/toybox.sh})
 \end{itemize}   
\end{frame}

\begin{frame}{SD Karte}
 Das Target RootFS auf der SD-Karte kann nur als \cod{root} modifiziert werden
 \begin{itemize}
  \item passen Sie Ihrem \host so an, dass Sie als (gew�hlicher) {\em user} das RootFS
  modifizieren k�nnen
 \end{itemize}
\end{frame}
