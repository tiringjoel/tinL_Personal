%-------------------------
%makefile
%(c) H.Buchmann FHNW 2014
%export TEXINPUTS=${HOME}/fhnw/edu/:${HOME}/fhnw/edu/tinL/config/latex:${HOME}/fhnw/edu/config//:
%-------------------------
\documentclass{beamer}
\usepackage{latex/beamer}
%---------------------
%local defines
%(c) H.Buchmann FHNW 2009
%$Id$
%---------------------
\def \target {\raspberry\xspace}
\def \host {{\em Host \xspace}}

\input{/home/buchmann/latex/dirtree/dirtree.tex}

\usepackage[absolute]{textpos}
\setlength{\TPHorizModule}{1mm}
\setlength{\TPVertModule}{1mm}

\begin{document}

\newcommand{\uboot}{{U-Boot \xspace}}

\title{Startup}

\frame{\titlepage}

\begin{frame}{Um was geht es ?}{wie startet ein Rechner}
 \begin{itemize}
  \item am Beispiel \target
  \item mit \uboot 
  \begin{itemize}
   \item dazwischen
  \end{itemize}
 \end{itemize}
\end{frame}

\begin{frame}{Informationen}
 \begin{itemize}
  \item \url{http://www.denx.de/wiki/U-Boot}
 \end{itemize}
\end{frame}

\section{Startup}
\begin{frame}{Reset}{der Big-Bang}
 \begin{enumerate}
  \item {\Large Reset} Signal
  \item Programmcounter \cod{pc} bekommt einen Wert:
   \begin{itemize}
    \item z.B. $pc\leftarrow 0$
   \end{itemize}
  \item der Code bei $pc$ wird ausgef�hrt
 \end{enumerate}
\end{frame}

\begin{frame}{Reset}{beim \target}
 \begin{enumerate}
  \item  {\Large Reset} Signal
  \item {\em first stage bootloader} 
  \begin{itemize}
   \item nicht zug�nglich
  \end{itemize}
  \item {\em second stage bootloader} \cod{bootcode.bin}
  \begin{itemize}
   \item schwierig zug�nglich
  \end{itemize}
  \item {\em GPU firmware} \cod{start.elf}
  \begin{itemize}
   \item \target GPU Code
   \item Konfiguration:
   \begin{itemize}
    \item \cod{config.txt}
    \item \cod{cmdline.txt} f�r \linux
   \end{itemize}
   \item ziemlich schwierig zug�nglich
  \end{itemize}
  
  \item {\em User Code}
  \begin{itemize}
   \item normalerweise \linux Kernel
  \end{itemize}
 \end{enumerate}
\end{frame}

\begin{frame}{User Code}{\linux Startup}
 \begin{enumerate}
  \setcounter{enumi}{5}
  \item {\em kernel} \cod{kernel.img}
  \item {\unix} \cod{init} Prozess
 \end{enumerate}
\end{frame}

\section{Aufgaben}
\subsection{Init}
\begin{frame}{Aufgabe}{eigener \unix \cod{init} Prozess}
 \begin{itemize}
  \item \cod{init} ein normales \unix Programm
  \item command line: \cod{init={\em myProcess}}
 \end{itemize}
\end{frame}

\subsection{u-boot}
\begin{frame}{Aufgabe}{\uboot als zus�tzlicher Zwischenschritt}
 \begin{enumerate}
  \setcounter{enumi}{5}
  \item \uboot nun \kernel 
  \item[] startet
  \item \linux \kernel
  \item \unix
 \end{enumerate}
\end{frame}

\begin{frame}{Herstellung}
\begin{itemize}
 \item Code \cod{git://github.com/swarren/u-boot.git}
 \item fast gleich wie der \linux \kernel
 \item im File \cod{config.txt} Eintrag:
 \begin{itemize}
  \item \cod{kernel={\em name-of-uboot-image}}
 \end{itemize}
\end{itemize}
\end{frame}

\begin{frame}{\linux}{mit \uboot starten}
 \begin{itemize}
  \item von der SD-Karte
  \item vom Netz
  \item ein U-Boot Image
 \end{itemize}
\end{frame}

\section{Probleme}
\begin{frame}[fragile]{Etwas Terminologie}
 \begin{description}[direct-boot]
  \item[direct-boot] {\bf ohne} \uboot
  \item[u-boot] {\bf mit} \uboot
  \item[kernel.img] das orignale {\em image}
  \item[Image] der selber gemachte
 \end{description}
 \remark{Test mit \cod{uname -a}}
\end{frame}

\begin{frame}{Die Probleme}
\begin{tabular}{l|l|l}
		& kernel.img 	& Image\\
\hline
direct-boot	& ok	     	& ok\\
u-boot	   	& (ok) ohne rootfs	& error\\
\end{tabular}
\begin{block}{Aufgabe}
\begin{itemize}
 \item nachpr�fen
\end{itemize}
\end{block}
\end{frame}

\subsection{Was kann man machen ?}
\begin{frame}{Memory}{\uboot Sicht}
\begin{center}
\includegraphics{memory.pdf}
\end{center}
\begin{itemize}
 \item {\color{red} reserviert}
 \item \uboot \cod{fatload mmc 0:1 addr file}
\end{itemize}
\end{frame}

\begin{frame}{Verschiedene Image Formate}
 \begin{itemize}
  \item Image
  \item zImage
  \item uImage
 \end{itemize}
\end{frame}

\end{document}
