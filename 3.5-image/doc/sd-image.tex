\section{SD Karte}
\begin{frame}{Initiale SD-Karte}{Herstellung}
 \begin{itemize}
  \item das Image:
  \begin{itemize}
   \item \href{https://drive.switch.ch/index.php/s/A6H382zEGDrgfAL}{sd-2020-03-04.img.xz}
  \end{itemize}
  \item auf SD-Karte
  \begin{itemize}
   \item \cod{xz -d -c sd-2020-03-04.img.xz | sudo dd of=/dev/sdX}
   \begin{remarks}
   \item {\Huge Vorsicht} bei \cod{/dev/sdX}
   \item File \cod{sd-2020-03-04.img.xz} im \href{https://drive.switch.ch/index.php/s/A6H382zEGDrgfAL}
                      {\color{red}$\to$Verzeichnis}
   \end{remarks}
  \end{itemize}
 \end{itemize}
\end{frame}

\begin{frame}[fragile]{2 Partitionen}{gemacht mit \cod{fdisk -l /dev/sdX}}
{
\footnotesize
\begin{verbatim}
Disk /dev/sda: 14.5 GiB, 15552479232 bytes, 30375936 sectors
Disk model: STORAGE DEVICE  
Units: sectors of 1 * 512 = 512 bytes
Sector size (logical/physical): 512 bytes / 512 bytes
I/O size (minimum/optimal): 512 bytes / 512 bytes
Disklabel type: dos
Disk identifier: 0x00000000

Device     Boot Start    End Sectors  Size Id Type
/dev/sda1  *     2048  34815   32768   16M  b W95 FAT32
/dev/sda2       34816 559103  524288  256M 83 Linux
\end{verbatim}
}
\begin{description}
 \item[p1] Boot Partition
 \item[p2] Root Filesystem das Linux
\end{description}
\end{frame}

%\subsection{Distribution}
%\begin{frame}[fragile]{Alles ist ein File}{und umgekehrt}
%\begin{itemize}
% \item Der File \cod{\distro}
% ist das Bild einer ganzen SD-Karte
% \item Mache SD-Karte
% \begin{itemize}
%  \item Kopiere {\tiny\url{sourceforge.net/projects/fhnw-tinl/files/{\distro}/download}} 
%  \item entzippe
%  \item kopiere
%  \begin{lstlisting}
%dd if=distro.img of=/dev/sd-card
%  \end{lstlisting}
%  \cod{sd-card} typ \cod{mmcblk\em{N}} $N=0|1 ..$
% \end{itemize}
% \remark{Alles mit {\em pipes}}
% 
%\end{itemize}
%
%\end{frame}
