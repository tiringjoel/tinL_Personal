\part{Installation TFT Display}
%\section{Installation TFT Display}
\frame{\partpage}
\begin{frame}{Um was geht es ?}{Adafruit TFT Touchscreen}
 \begin{itemize}
  \item Installation auf einem minimalen System
  \item \cod{git} update
  \item Konfiguration
  \item Kernelcompilation
  \item Driver
  \item Module
  \item {\em framebuffer}
 \end{itemize}
\end{frame}

\begin{frame}{Hardware}
 \begin{itemize}
  \item {\tiny\url{https://learn.adafruit.com/adafruit-2-8-pitft-capacitive-touch/downloads}}
  \item Zwei Chips
  \begin{itemize}
   \item \cod{STMPE610} für den {\em touchscreen}
   \item \cod{ILI9341} für die Graphik
  \end{itemize}
  \item Verbunden per \cod{SPI}
  \end{itemize}
\end{frame}

\begin{frame}{Funktionierendes System}
\begin{itemize}
 \item {\tiny \url{http://adafruit-download.s3.amazonaws.com/2015-02-16-raspbian-pitft28r_150312.zip}}
 \item Verbindung per \cod{ssh}
 \begin{remarks}
  \item user \cod{pi}
  \item pw: \cod{raspberry}
 \end{remarks}
 \item Host \cod{dhcp}
 \item Host \cod{nmap}
 \item Framebuffer \cod{/dev/fb*}
 \begin{itemize}
  \item alles ist ein File
 \end{itemize}
 \item Backlight Control
\end{itemize}
\end{frame}

\begin{frame}{Kernel}{Neue Version}
\begin{itemize}
 \item \cod{git pull}
 \item \cod{drivers/staging/fbtft/fb\_ili9341.c}
\end{itemize}
\end{frame}

\section{Module}

\begin{frame}{Module herstellen}
 \begin{itemize}
  \item als Modul herstellen: {\em menuconfig}
  \begin{itemize}
   \item \cod{FB\_TFT\_FBTFT\_DEVICE}
   \begin{itemize}\item\cod{drivers/staging/fbtft/fbtft\_device.c}\end{itemize}
   \item \cod{FB\_TFT\_ILI9341}
   \begin{itemize}\item\cod{drivers/staging/fbtft/fb\_ili9341.c}\end{itemize}
  \end{itemize}
  \item nur einzelne Module herstellen
  \begin{itemize}
   \item {\em dir/file.ko}
   \begin{itemize}
    \item {\em dir} relativ zu kernel source
   \end{itemize}
  \end{itemize}
 \end{itemize}
\end{frame}

\begin{frame}{Module ausführen}
 \begin{itemize}
  \item zum {\em target} kopieren
  \item \cod{insmod {\em name parameters}}
  \item \cod{mkdev -s} erzeugt \cod{/dev/fb1}
  \item siehe auch
   \begin{itemize}\item\cod{/sys/class/graphics/}\end{itemize}
 \end{itemize}
\end{frame}

\begin{frame}{Zusätzliche Infos}{von einem funktionierenden System}
 \begin{itemize}
  \item \cod{dmesg-wheezy-raspbian.log}
  \ite \cod{config-wheezy-raspbian.gz}
 \end{itemize}
\end{frame}
