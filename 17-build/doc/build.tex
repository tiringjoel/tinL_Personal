%-------------------------
%clang
%(c) H.Buchmann FHNW 2008
%export TEXINPUTS=.:${HOME}/fhnw/edu/:${HOME}/fhnw/edu/tinL/config/latex:${HOME}/fhnw/edu/config//:
%-------------------------
\documentclass{beamer}
\usepackage{beamer}
%---------------------
%local defines
%(c) H.Buchmann FHNW 2009
%$Id$
%---------------------
\def \target {\raspberry\xspace}
\def \host {{\em Host \xspace}}

\input{/home/buchmann/latex/dirtree/dirtree.tex}

\usepackage[absolute]{textpos}
\setlength{\TPHorizModule}{1mm}
\setlength{\TPVertModule}{1mm}

\begin{document}

\title[Build]{Ein ganzes \linux}

\frame{\titlepage}

\begin{frame}{Um was geht es ?}
 \begin{itemize}
  \item ein \linux von Grund auf bauen
  \begin{itemize}
   \item nicht mehr so schwer wie auch schon
  \end{itemize}
  \item nicht v�llig automatisiert
  \item Alternative zu {\bf yocto} \& Co.
 \end{itemize}
\end{frame}

\begin{frame}{Ziel}{\linux auf dem \target}
\begin{itemize}
 \item command based
 \item Ethernet
 \item \cod{ssh} 
 \item \cod{sshfs}
 \item moderne Toolchain inkl. {\em c++14} \cpp
 \remark{parallel zu \linux bauen wir die Toolchain}
\end{itemize}
\end{frame}

\section{Die Komponenten}
\begin{frame}{Komponenten}{\target und \host}
 \begin{block}{\target}
  \begin{description}
   \item[Kernel] wenige Files (zwei)
   \item[root] ein Filesystem viele Files
  \end{description}
 \end{block}
 \begin{block}{\host}
  \begin{description}
   \item[Toolchain] binutils, gcc, Bibliotheken für den Compiler
  \end{description}
 \end{block}
\end{frame}
\subsection{Target \target}

\begin{frame}{Übersicht}
 \begin{center}
  \includegraphics[width=8cm]{layers.pdf}
 \end{center}
\end{frame}

\begin{frame}{Die Komponenten}{für \target}
 \begin{description}
  \item[Hardware] \target
  \item[Kernel] zugeschnitten auf \target
  \begin{itemize}
    \item {\tiny \url{github.com/beagleboard/linux}}
  \end{itemize}	
  \item[root] das Filesystem
  \begin{description}
   \item[LibC] glibc 
   \begin{itemize}
    \item {\tiny \url{www.gnu.org/software/libc/index.html}}
   \end{itemize}	
   \item[\unix] busybox
   \begin{itemize}
    \item {\tiny \url{www.busybox.net/}}
   \end{itemize}
   \item[...] Weitere \unix basierte Komponenten
   \begin{itemize}
    \item das \cod{configure}, \cod{make}, \cod{make install} Triple 
   \end{itemize}
   \end{description}
 \end{description}
\end{frame}

\subsection{\host}
\begin{frame}{Toolchain}
 \begin{description}
  \item[binutils] linker \& Co.
  \item[gcc] compiler
  \begin{itemize}
   \item \cod{libgcc} die Bibliothek für den Compiler
  \end{itemize}
 \end{description}
 
 \begin{remarks}
 \item die Toolchain muss zweimal gebaut werden
 \begin{itemize}
  \item für den \kernel und \cod{libc} 
  \item für \unix/\posix
 \end{itemize}
 \item das target
 \begin{itemize} 
  \item \cod{cpu-vendor-os}
  
 \end{itemize}
 \end{remarks}
\end{frame}

\begin{frame}{Die Verzeichnisstruktur}
\dirtree{%
 .1 {\em somewhere\_on\_the\_host}.
 .2 tools.
 .3 config.sh \DTcomment{used in (all) scripts}.
 .3 {\em component}.sh \DTcomment{how to build}.
 .2 build \DTcomment{home of the build files}.
 .3 {\em component} \DTcomment{directory}.
 .2 target-root \DTcomment{top of targer root}.
 .2 tc \DTcomment{the new toolchain}.
 .2 config \DTcomment{of some components}.
 .2 mount \DTcomment{for mounting the \targetS (sshfs)}.
}
\end{frame}


\input{buildit.tex}

\section{Configure}
\begin{frame}{{\em configure}-{\em make}-{\em make install}}{Installation neuer Komponenten}
 \begin{itemize}
  \item aus den Quellen
  \item immer etwa gleich
  \begin{itemize}
   \item download
   \item \cod{configure {\em options}}
   \item \cod{make}
   \item \cod{make install}
  \end{itemize}
  \item Unterschiede in den Details
 \end{itemize}
\end{frame}

%\begin{frame}{\url{rsync.samba.org}}{als Beispiel}
% \begin{itemize}
%  \item auf dem \host
%  \item auf dem \targetS
% \end{itemize}
%\end{frame}
%
%
%\subsection{Auf dem \host}
%
%\begin{frame}{Verzeichnisstruktur}
% \begin{itemize}
%  \item Source:\cod{rsync-3.1.2}
%  \item Build:   für die (vielen) Zwischenfiles
%  \item Install: prefix
%  \item rsync.sh: das Skript
% \end{itemize}
%\end{frame}
%
%\begin{frame}{Skript: \cod{rsync.sh}}{schrittweise für den \host}
% \begin{itemize}
%  \item \cod{configure --help}
%  \item \cod{configure --prefix}
%  \begin{itemize}
%   \item prefix: wohin kommt das Resultat
%   \item Files in \cod{rsync-build}
%  \end{itemize}
%  \item \cod{make}
%  \begin{itemize}
%   \item Files in \cod{rsync-build}
%  \end{itemize}
%  \item \cod{make install}
%  \begin{itemize}
%   \item Files in \cod{prefix}
%  \end{itemize}
% \end{itemize}
%\end{frame}
%
%\begin{frame}{Aufgabe Skript: \cod{rsync.sh}}{für \targetS}
% \begin{itemize}
%  \item Crosscompile \cod{--sysroot}
%  \item \cod{--prefix}
%  \item \cod{DESTDIR}
% \end{itemize}
% \remark{tools/rsync.sh}
%\end{frame}

%\section{Modules}
%\begin{frame}{Modules}
%\begin{itemize}
% \item siehe \cod{6-modules}
%\end{itemize}
%\end{frame}
%
%\begin{frame}{Modules}{Verzeichnisstruktur}
%\begin{block}{Host}
%\dirtree{%
% .1 BUILD\_HOME.
% .2 modules.
% .3 Makefile \DTcomment{for own modules}.
% .3 *.{c/h} \DTcomment{the own sources}.
% .2 scripts.
% .3 module.sh.
%}
%\end{block}
%\begin{block}{\target}
%\dirtree{%
%.1 /.
%.2 work.
%}
%\end{block}
%\end{frame}
%\begin{frame}{Herstellung}{Workflow}
% \begin{block}{Host}
% \begin{itemize}
%  \item \cod{sh scripts/module.sh {\em module}}
%  \item \cod{scp}, \cod{sftp}, \cod{ftp}
% \end{itemize}
% \end{block}
% \begin{block}{\target}
%  \begin{itemize}
%   \item \cod{insmod}
%   \item \cod{rmmod}
%  \end{itemize}
% \end{block}
%\end{frame}

%\part{Installation TFT Display}
%\section{Installation TFT Display}
\frame{\partpage}
\begin{frame}{Um was geht es ?}{Adafruit TFT Touchscreen}
 \begin{itemize}
  \item Installation auf einem minimalen System
  \item \cod{git} update
  \item Konfiguration
  \item Kernelcompilation
  \item Driver
  \item Module
  \item {\em framebuffer}
 \end{itemize}
\end{frame}

\begin{frame}{Hardware}
 \begin{itemize}
  \item {\tiny\url{https://learn.adafruit.com/adafruit-2-8-pitft-capacitive-touch/downloads}}
  \item Zwei Chips
  \begin{itemize}
   \item \cod{STMPE610} für den {\em touchscreen}
   \item \cod{ILI9341} für die Graphik
  \end{itemize}
  \item Verbunden per \cod{SPI}
  \end{itemize}
\end{frame}

\begin{frame}{Funktionierendes System}
\begin{itemize}
 \item {\tiny \url{http://adafruit-download.s3.amazonaws.com/2015-02-16-raspbian-pitft28r_150312.zip}}
 \item Verbindung per \cod{ssh}
 \begin{remarks}
  \item user \cod{pi}
  \item pw: \cod{raspberry}
 \end{remarks}
 \item Host \cod{dhcp}
 \item Host \cod{nmap}
 \item Framebuffer \cod{/dev/fb*}
 \begin{itemize}
  \item alles ist ein File
 \end{itemize}
 \item Backlight Control
\end{itemize}
\end{frame}

\begin{frame}{Kernel}{Neue Version}
\begin{itemize}
 \item \cod{git pull}
 \item \cod{drivers/staging/fbtft/fb\_ili9341.c}
\end{itemize}
\end{frame}

\section{Module}

\begin{frame}{Module herstellen}
 \begin{itemize}
  \item als Modul herstellen: {\em menuconfig}
  \begin{itemize}
   \item \cod{FB\_TFT\_FBTFT\_DEVICE}
   \begin{itemize}\item\cod{drivers/staging/fbtft/fbtft\_device.c}\end{itemize}
   \item \cod{FB\_TFT\_ILI9341}
   \begin{itemize}\item\cod{drivers/staging/fbtft/fb\_ili9341.c}\end{itemize}
  \end{itemize}
  \item nur einzelne Module herstellen
  \begin{itemize}
   \item {\em dir/file.ko}
   \begin{itemize}
    \item {\em dir} relativ zu kernel source
   \end{itemize}
  \end{itemize}
 \end{itemize}
\end{frame}

\begin{frame}{Module ausführen}
 \begin{itemize}
  \item zum {\em target} kopieren
  \item \cod{insmod {\em name parameters}}
  \item \cod{mkdev -s} erzeugt \cod{/dev/fb1}
  \item siehe auch
   \begin{itemize}\item\cod{/sys/class/graphics/}\end{itemize}
 \end{itemize}
\end{frame}

\begin{frame}{Zusätzliche Infos}{von einem funktionierenden System}
 \begin{itemize}
  \item \cod{dmesg-wheezy-raspbian.log}
  \ite \cod{config-wheezy-raspbian.gz}
 \end{itemize}
\end{frame}

\end{document}
