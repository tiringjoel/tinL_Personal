%-------------------------
%minimal-unix
%(c) H.Buchmann FHNW 2014
%export TEXINPUTS=${HOME}/fhnw/edu/:${HOME}/fhnw/edu/tinL/config/latex:${HOME}/fhnw/edu/config//:
%-------------------------
\setlength{\TPHorizModule}{1mm}
\setlength{\TPVertModule}{1mm}

\newcommand{\qemu}{{\em qemu}\xspace}
\newcommand{\busybox}{{\em busybox}\xspace}
\newcommand{\yocto}{{\em yocto}\xspace}

\section{Toolchain}
\begin{frame}{Toolchain}{tc}
 \begin{itemize}
  \item die grossen zwei:
  \begin{itemize}
  \item Compiler
  \item Linker
  \end{itemize}
  \item kleinere Programme:
  \begin{itemize}
   \item Assembler
   \item ...
  \end{itemize}
 \end{itemize}
\end{frame}

\begin{frame}{Toolchain}{Beispiel}
 \begin{itemize}
  \item Sourcefile \cod{\{c|cc\}-source.\{c|cc\}}
  \item Compilat/object File \cod{\{c|cc\}-source.o}
  \item Executable/Image \cod{\{c|cc\}-source}
 \end{itemize}
\end{frame}

\subsection{Cross toolchain}
\begin{frame}{Cross toolchain}{2 Verschiedene Rechner}
\begin{description}[Cross\{Programm\}]
 \item[Host] Workstation leistungsf�higer Rechner
 \item[Target] Eingebettetes System (\beaglebone)
 \item[Cross\{Programm\}] Programm (Compiler etc.) das
 \begin{itemize}
  \item l�uft auf dem {\em Host} und erzeugt Files f�r das {\em Target}
 \end{itemize}
\end{description}
\end{frame}

\begin{frame}{Cross toolchain}
 \begin{itemize}
  \item erzeugt auf dem {\em Host} Programme f�r das {\em Target}
 \end{itemize}
\end{frame}

\subsection{GNU/Toolchain}

\begin{frame}{GNU/Toolchain}{Zwei Komponenten}
\begin{description}[binutils]
 \item[binutils] Linker, assembler, ...
 \item[gcc] Compiler
\end{description}
\end{frame}

\subsection{Das klassische Build}
\begin{frame}{Build}{die drei Schritte}
 \begin{itemize}
  \item configure
  \item make
  \item make install
 \end{itemize}
 \remark{auf dem {\Huge Host}}
\end{frame}

\begin{frame}{Build}{der Kontext}
 \begin{description}
  \item[prefix] wo die Toolchain auf dem {\em Host} installiert wird
  \begin{itemize}
   \item option \cod{--prefix={\em path-to-toolchain-install}} 
  \end{itemize}
  \item[sysroot] wo ist das {\em Target} root system (auf dem {\em Host})
  \begin{itemize}
   \item option \cod{--with-sysroot={\em path-to-target-sysroot}}
  \end{itemize}
  \item[target] was f�r eine {\em Target} System
  \begin{itemize}
   \item option \cod{--target=armv6l-unknown-linux-gnueabihf}
   \remark{Warum ???}
  \end{itemize}
 \end{description}
\end{frame}

