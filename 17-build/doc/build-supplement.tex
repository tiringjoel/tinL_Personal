%-------------------------
%build-supplement.tex
%(c) H.Buchmann FHNW 2018
%export TEXINPUTS=.:${HOME}/fhnw/edu/:${HOME}/fhnw/edu/tinL/config/latex:${HOME}/fhnw/edu/config//:
%-------------------------
\documentclass{beamer}
\usepackage{beamer}
%---------------------
%local defines
%(c) H.Buchmann FHNW 2009
%$Id$
%---------------------
\newcommand{\target} {\beaglebone\xspace}
\newcommand{\targetS}{{\bf BBG}\xspace}
\newcommand{\host}   {{\em Host}\xspace}
\newcommand{\targetroot} {{\bf target-root}\xspace}
\newcommand{\kernel} {{\bf kernel}\xspace}
\renewcommand{\c}{{\bf C}\xspace}
\newcommand{\cpp}{{\bf C++}\xspace}
\newcommand{\posix}{{\bf POSIX}\xspace}


\usepackage[absolute]{textpos}
\setlength{\TPHorizModule}{1mm}
\setlength{\TPVertModule}{1mm}

\begin{document}

\title[Build Again]{Ein ganzes \linux}

\begin{frame}{Komponente}{ein Softwarepaket}
 \begin{itemize}
  \item Sammlung von Source Files
  \item Konfiguration für verschiedene Architekturen
  \item kann heruntergalden werden
  \item kann auch selber geschrieben werden
  \item können von anderen Komponenten abhängen
 \end{itemize}
\end{frame}

\begin{frame}{Komponente}{Beispiele}
 \begin{itemize}
  \item linux, kernel
  \item gcc, Compiler 
 \end{itemize}
\end{frame}

\begin{frame}{Komponente}
 \begin{itemize}
  \item hat einen Namen
  \item existiert als einzelner Archivfile 
  \begin{itemize}
   \item *.tar.gz, *.tar.xy 
  \end{itemize}
  \item existiert als Repository
 \end{itemize}
\end{frame}

\begin{frame}{Unser einfaches Buildsystem}
 \begin{itemize}
  \item halbautomatisch
  \item Skripts
 \end{itemize}
\end{frame}

\begin{frame}{Prinzipien}
 \begin{itemize}
  \item zu jeder Komponente gehört:
  \begin{itemize}
   \item ein Sourceverzeichnis
   \item ein Skript
   \item ein Buildverzeichnis
  \end{itemize}
 \end{itemize}
\end{frame}

{\newcommand{\comp}{{\em comp}}
\begin{frame}{Verzeichnisstruktur}{am Beispiel der Komponente \comp}
 \dirtree{%
 .1 17-build \DTcomment{the home}.
 .2 tools    \DTcomment{the home scripts}.
 .3 {..}.
 .3 \comp.
 .3 {..}.
 .2 {..}.
 .2 \comp \DTcomment{das Buildverzeichnis}.
 .3 {..} \DTcomment{erzeugt beim build}.
 .2 {..}.
 }
\end{frame}
}

\begin{frame}{Ein typisches Beispiel}{\cod{binutils}}
 \begin{itemize}
  \item Source \cod{repositories/binutils-2.31}
  \item Skript \cod{tools/binutils.sh} mit \cod{tools/common.sh}
  \item Buildverzeichnis: \cod{binutils.sh}
 \end{itemize}
\end{frame}

\section{Build}

\begin{frame}{Toolchain 1}
\begin{tabular}{l|l|l}
 Comp & args & Verzeichnis\\
 \hline\hline
 binutils & & \cod{tc}\\
 \hline
 gmp      & & \cod{tc}\\
 \hline
 mpfr     & & \cod{tc}\\ 
 \hline
 mpc      & & \cod{tc}\\ 
 \hline 
 gcc-bare & & \cod{tc}\\
\end{tabular}
\end{frame}

\begin{frame}{Kernel/glibc}
\begin{tabular}{l|l|l} 
 Comp & args \\
 \hline\hline
 kernel	  & defconfig &\\
          & menuconfig &\\
 	  & zImage  & \cod{target-root/boot}\\
	  & dtbs    & \cod{target-root/boot}\\
	  & headers\_install & \cod{target-root/usr/include}\\
 \hline
 glibc.sh & & \cod{usr/lib}, \cod{target-root/usr/include}
\end{tabular}
\end{frame}

\begin{frame}{Toolchain 2}
\begin{tabular}{l|l|l}
Comp & args \\
 \hline\hline
gcc & &  \cod{tc}
\end{tabular}
\begin{block}{Download}
\begin{itemize}
 \item \href{https://drive.switch.ch/index.php/s/PrfwR3SUYQkc29B}
            {tc-tinl-gcc-8.2.0-2018.12.11.tar.gz}
            
\end{itemize}
\end{block}
\end{frame}

\begin{frame}{Rootfilesystem}{minimal}
\begin{tabular}{l|l|l}
Comp & args \\
\hline\hline
busybox & menuconfig &\\
	& busybox    &\\
	& install    &\cod{target-root}\\
\end{tabular}
\begin{block}{Download}
\begin{itemize}
 \item \href{https://drive.switch.ch/index.php/s/9kENTq9RacaCzZs}
            {target-root-2018.12.11.tar.gz}
\end{itemize}
\end{block}
\end{frame}

\begin{frame}{Rootfilesystem}{ssh}
\begin{tabular}{l|l|l}
Comp & args \\
\hline\hline
zlib    &  &\\
\hline
openssl	& &\\
\hline
openssh
\end{tabular}
\remark{Jetzt funktioniert \cod{ssh}/\cod{sshfs}}
\end{frame}

\begin{frame}{Rootfilesystem}{wi-fi}
\begin{tabular}{l|l|l}
Comp & args \\
\hline\hline
libnl    &  &\url{github.com/thom311/libnl/releases/}\\
wpa\_supplicant & &
\end{tabular}
\begin{remarks}
\item Kernel anpassen
\item \href{https://drive.switch.ch/index.php/s/ysiZvdI9chgWcqN}{ti-connectivity}
\item Setup wlan0 Siehe \cod{setup-wlan0.txt}
\end{remarks}
\end{frame}

\end{document}
