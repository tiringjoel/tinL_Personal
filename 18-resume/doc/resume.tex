%-------------------------
%resume
%(c) H.Buchmann FHNW 2018
%export TEXINPUTS=.:${HOME}/fhnw/edu/:${HOME}/fhnw/edu/tinL/config/latex:${HOME}/fhnw/edu/config//:
%-------------------------
\documentclass{beamer}
\usepackage{beamer}
%---------------------
%local defines
%(c) H.Buchmann FHNW 2009
%$Id$
%---------------------
\def \target {\raspberry\xspace}
\def \host {{\em Host \xspace}}


\usepackage[absolute]{textpos}
\setlength{\TPHorizModule}{1mm}
\setlength{\TPVertModule}{1mm}

\begin{document}

\title[Resume]{Resume}

\frame{\titlepage}

\begin{frame}{Um was geht es ?}{\linux Teil 1}
 \begin{itemize}
  \item Toolchain bare 
  \item \cod{kernel}
  \item \cod{libc}
  \item Toolchain full
  \begin{itemize}
   \item inkl. \cpp
  \end{itemize}
  \item \cod{busybox}
  \item \cod{ssh}
  \begin{itemize}
   \item Client/Server
  \end{itemize}
 \end{itemize}
 
 \remark{basierend auf 17-build}
\end{frame}

\begin{frame}{Um was geht es ?}{\linux Teil 2}
 \begin{itemize}
  \item Network
  \begin{itemize}
   \item per USB
   \item per WiFi
  \end{itemize}
  \item CrossDevelopment
  \begin{itemize}
   \item Entwicklung auf dem \host
   \item Executables ouf dem \targetS
  \end{itemize}
  \item Weitere {\em packages}
  \begin{itemize}
   \item lighttpd
  \end{itemize}
 \end{itemize}
\end{frame}

\begin{frame}{Host}{Verzeichnisstruktur}
 \dirtree{%
  .1 17-build.
  .2 SIGNATURE \DTcomment{invisible uniqe number}.
  .2 config.
  .2 tools \DTcomment{scripts}.
  .3 {\em component.}sh. 
  .2 build \DTcomment{data during build}.
  .3 {\em component}.
  .2 tc \DTcomment{toolchain}.
  .2 target-root \DTcomment{on the \host}.
  .2 mount \DTcomment{root on \targetS per sshfs}.
 }
\end{frame}

\begin{frame}{Build}
 \begin{tabular}{rl|l} 
 		 &	& args\\
 \hline\hline
  toolchain bare & \cod{binutils.sh} \\
 		 &\cod{gcc-bare.sh}\\ 
 \hline
   kernel 	 & \cod{kernel.sh} & [\cod{bb.org\_defconfig}]\\
 		 & &\cod{zImage}\\
                 & &\cod{dtbs}\\
                 & &\cod{headers\_install}\\
 \hline
 libc 		 & \cod{glibc.sh}\\
 \end{tabular}
\end{frame}

\begin{frame}{Build}{Fortsetzung}
\begin{tabular}{rl|l}
 		 &	& args\\
 \hline \hline
 toolchain full  & \cod{gcc-target.sh}\\
 \hline
 \unix & \cod{busybox.sh} & [\cod{menuconfig}] \\
       &		  & \cod{busybox}\\
       &	          & \cod{install}\\
 & \cod{zlib.sh}\\
 & \cod{openssl.sh}\\
 & \cod{openssh.sh}\\
 \hline
 toolchain-host & \cod{gcc-host.sh}\\
 \hline
 Distro & \cod{target-root.sh}\\
 	& \cod{tc.sh}
 \end{tabular}
\end{frame}

\begin{frame}{\target}{Verzeichnisstruktur}
\dirtree{%
.1 / \DTcomment{target-root on \host}.
.2 etc.
.3 ssh* \DTcomment{private key server}.
.3 init.d.
.4 rcS \DTcomment{init script}.
.2 boot \DTcomment{new}.
.3 zImage \DTcomment{kernel}.
.3 m335x-boneblack-wireless.dtb \DTcomment{device tree}.
.2 {...} \DTcomment{normal \unix}.
}
\end{frame}

\begin{frame}{Development}
 \dirtree{%
 .1 17-build.
 .2 config.
 .3 Makefile.
 .2 src \DTcomment{source files}.
 .2 work.
 .3 $\to$ Makefile \DTcomment{link}.
 .2 mount \DTcomment{mounted e.g. via sshfs}.
 }
\end{frame}

\begin{frame}{Was ist noch zu tun ?}
 \begin{itemize}
  \item Wo befinden sich die {\em Source Files} 
  \begin{itemize}
   \item nicht immer suchen
  \end{itemize}
  \item user-root Separation
  \begin{itemize}
   \item \host\ {\bf user} (fast immer)
   \item \targetS\ {\bf root}: für den Moment mindestens
  \end{itemize}
  \item Workflow für schrittweise Entwicklung
  \begin{itemize}
   \item Saubere Anwendung von \cod{rsync}
   \item Ev. anderes Tool: (wie steht es mit \cod{git}) ?
  \end{itemize}
  \item Kernel Konfiguration
  \begin{itemize}
   \item \cod{omap\_i2c 4819c000.i2c: timeout waiting for bus ready}  
  \end{itemize}
  \item Device Tree
 \end{itemize}
\end{frame}


\end{document}
