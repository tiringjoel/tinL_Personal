%-------------------------
%resume
%(c) H.Buchmann FHNW 2018
%export TEXINPUTS=.:${HOME}/fhnw/edu/:${HOME}/fhnw/edu/tinL/config/latex:${HOME}/fhnw/edu/config//:
%-------------------------
\documentclass{beamer}
\usepackage{beamer}
%---------------------
%local defines
%(c) H.Buchmann FHNW 2009
%$Id$
%---------------------
\newcommand{\target} {\beaglebone\xspace}
\newcommand{\targetS}{{\bf BBG}\xspace}
\newcommand{\host}   {{\em Host}\xspace}
\newcommand{\targetroot} {{\bf target-root}\xspace}
\newcommand{\kernel} {{\bf kernel}\xspace}
\renewcommand{\c}{{\bf C}\xspace}
\newcommand{\cpp}{{\bf C++}\xspace}
\newcommand{\posix}{{\bf POSIX}\xspace}

\input{/home/buchmann/latex/dirtree/dirtree.tex}

\usepackage[absolute]{textpos}
\setlength{\TPHorizModule}{1mm}
\setlength{\TPVertModule}{1mm}

\begin{document}

\title[Resume]{Resume}

\frame{\titlepage}

\begin{frame}{Um was geht es ?}{\linux}
 \begin{itemize}
  \item Toolchain bare 
  \item \cod{kernel}
  \item \cod{libc}
  \item Toolchain full
  \begin{itemize}
   \item inkl. \cpp
  \end{itemize}
  \item \cod{busybox}
  \item \cod{ssh}
  \begin{itemize}
   \item Client/Server
  \end{itemize}
 \end{itemize}
 \remark{basierend auf 17-build}
\end{frame}

\begin{frame}{Host}{Verzeichnisstruktur}
 \dirtree{%
  .1 18-resume.
  .2 SIGNATURE \DTcomment{invisible uniqe number}.
  .2 config.
  .2 tools \DTcomment{scripts}.
  .3 {\em component.}sh. 
  .2 build \DTcomment{data during build}.
  .3 {\em component}.
  .2 tc \DTcomment{toolchain}.
  .2 target-root.
 }
\end{frame}

\begin{frame}{Build}
 \begin{tabular}{rl|l}
 		 &	& args\\
 \hline\hline
  toolchain bare & \cod{binutils.sh} \\
 		 &\cod{gcc-bare.sh}\\ 
 \hline
   kernel 	 & \cod{kernel.sh} & [\cod{bb.org\_defconfig}]\\
 		 & &\cod{zImage}\\
                 & &\cod{dtbs}\\
                 & &\cod{headers\_install}\\
 \hline
 libc 		 & \cod{glibc.sh}\\
 \hline
 toolchain full  & \cod{gcc.sh}\\
 \hline
 \unix & \cod{busybox.sh} & [\cod{menuconfig}] \\
       &		  & \cod{busybox}\\
       &	          & \cod{install}\\
 & \cod{zlib.sh}\\
 & \cod{openssl.sh}\\
 & \cod{openssh.sh}\\
 \hline
 Distro & \cod{target-root.sh}\\
 	& \cod{tc.sh}
 \end{tabular}
\end{frame}

\begin{frame}{\target}{Verzeichnisstruktur}
\dirtree{%
.1 / \DTcomment{target-root on \host}.
.2 etc.
.3 ssh* \DTcomment{private key server}.
.3 init.d.
.4 rcS \DTcomment{init script}.
.2 boot \DTcomment{new}.
.3 zImage \DTcomment{kernel}.
.3 m335x-boneblack-wireless.dtb \DTcomment{device tree}.
.2 {...} \DTcomment{normal \unix}.
}
\end{frame}

\begin{frame}{Development}
 \dirtree{%
 .1 18-resume.
 .2 config.
 .3 Makefile.
 .2 src \DTcomment{source files}.
 .2 work.
 .3 $\to$ Makefile \DTcomment{link}.
 .2 target-devel \DTcomment{mounted e.g. via sshfs}.
 }
\end{frame}

\end{document}
