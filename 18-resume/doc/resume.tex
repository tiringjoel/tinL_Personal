%-------------------------
%resume
%(c) H.Buchmann FHNW 2018
%export TEXINPUTS=.:${HOME}/fhnw/edu/:${HOME}/fhnw/edu/tinL/config/latex:${HOME}/fhnw/edu/config//:
%-------------------------
\documentclass{beamer}
\usepackage{beamer}
%---------------------
%local defines
%(c) H.Buchmann FHNW 2009
%$Id$
%---------------------
\newcommand{\target} {\beaglebone\xspace}
\newcommand{\targetS}{{\bf BBG}\xspace}
\newcommand{\host}   {{\em Host}\xspace}
\newcommand{\targetroot} {{\bf target-root}\xspace}
\newcommand{\kernel} {{\bf kernel}\xspace}
\renewcommand{\c}{{\bf C}\xspace}
\newcommand{\cpp}{{\bf C++}\xspace}
\newcommand{\posix}{{\bf POSIX}\xspace}

\input{/home/buchmann/latex/dirtree/dirtree.tex}

\usepackage[absolute]{textpos}
\setlength{\TPHorizModule}{1mm}
\setlength{\TPVertModule}{1mm}

\begin{document}

\title[Resume]{Resume}

\frame{\titlepage}

\begin{frame}{Um was geht es ?}{\linux}
 \begin{itemize}
  \item Toolchain 1 \cod{tc1}
  \item \cod{kernel}
  \item \cod{libc}
  \item Toolchain 2 \cod{tc2}
  \begin{itemize}
   \item inkl. \cpp
  \end{itemize}
  \item \cod{busybox}
  \item \cod{ssh}
  \begin{itemize}
   \item Client/Server
  \end{itemize}
 \end{itemize}
 \remark{basierend auf 17-build}
\end{frame}

\begin{frame}{Host}{Verzeichnisstruktur}
 \dirtree{%
  .1 SIGNATURE \DTcomment{invisible uniqe number}.
  .1 18-resume.
  .2 tools \DTcomment{scripts}.
  .3 config.sh.
  .3 kernel.sh.
  .3 glibc.sh.
  .3 busybox.sh.
  .3 {...}sh. 
  .2 build \DTcomment{data during build}.
  .3 kernel.
  .3 glibc.
  .3 busybox.
  .3 {...}.
  .2 tc \DTcomment{toolchain}.
  .2 target-root.
 }
\end{frame}

\begin{frame}{Build}
 \begin{tabular}{lll}
 
 binutils& \cod{binutils.sh} \\
         & \cod{gcc-bare.sh}\\ 
 kernel  & see \cod{kernel.sh} ohne Parameter\\ 
 libc    & \cod{glibc.sh}\\
 tc2     & \cod{gcc.sh}\\
 busybox & \cod{busybox.sh}\\
 zlib	 & \cod{zlib.sh}\\
 openssl & \cod{openssl.sh}\\
 openssh & \cod{openssh.sh}
 \end{tabular}
\end{frame}

\end{document}
