\begin{frame}{Crossdevelopment}
 \begin{itemize}
  \item zwei Rechner
  \begin{description}
   \item[Host] der Entwicklungsrechner
   \item[Target] \targetS der Zielrechner
  \end{description}
  \item Development
  \begin{itemize}
   \item Wo sind die Files
  \end{itemize}
  \item CrossDevelopment
  \begin{itemize}
   \item Wo sind die Files
  \end{itemize}
 \end{itemize}
\end{frame}

\begin{frame}{Outline}
 \begin{itemize}
  \item Development
  \begin{itemize}
   \item Programme auf dem \host für den \host
  \end{itemize}
  \item CrossDevelopment
  \begin{itemize}
   \item Programme auf dem \host für für den \target
  \end{itemize}
 \end{itemize}
\end{frame}

\begin{frame}{Ziel}
 \begin{itemize}
  \item CrossDevelopment (fast) gleich wie Development
  \item wegen {\bf POSIX}
 \end{itemize}
\end{frame}

\begin{frame}{Verzeichnisstruktur}
 \dirtree{%
 .1 18-crossdevelopment.
 .2 src \DTcomment{the source files}.
 .2 config.
 .3 Makefile.
 .2 tc  \DTcomment{target toolchain normally link}.
 .2 work.
 .3 Makefile $\to$ {../config/Makefile}.
 .2 mount \DTcomment{per sshfs on \targetS}.
 }
\end{frame}

\begin{frame}{Die Komponenten}
 \begin{itemize}
  \item der \cod{Makefile}
  \begin{itemize}
   \item steuert die Herstellung
   \item das Programm \cod{make}
  \end{itemize}
  
  \item Die Sourcefiles
  \begin{itemize}
   \item fast alle {\bf POSIX}
   \item der Compiler/Assembler/Linker
  \end{itemize}
 \end{itemize}
\end{frame}

\section{Aufgaben}

\begin{frame}{POSIX Programme}{Siehe \cod{Makefile}}
\begin{itemize}
 \item \cod{hello-world.c} \cod{cpp-hello-world.cc} 
 \begin{itemize}
  \item {\em dynamische} vs. {\em statische} Libraries
  \item Grösse der {\em executables}
 \end{itemize}
 \item \cod{thread-demo-1.cc}
 \begin{itemize}
  \item viel \cpp
 \end{itemize}
 \item \cod{primes.cc}
 \begin{itemize}
  \item auf dem \host und auf dem \targetS
  \item Ausführungszeiten
 \end{itemize}
\end{itemize}
\end{frame}

\begin{frame}{Elementare Programme}{Siehe \cod{Makefile}}
 \begin{itemize}
  \item \cod{s-without-posix.S}
  \begin{itemize}
   \item ein sehr einfaches \linux Assembler-Programm im {\em userspace}
  \end{itemize}
  \item \cod{c-without-posix.c}
  \begin{itemize}
   \item ein sehr einfaches \linux C-Programm im {\em userspace}
  \end{itemize}
 \end{itemize}
\end{frame}

%\subsection{Development}

\begin{frame}{Development (noch nicht Cross):Verzeichnis: \cod{host-work}}{die einzelnen Schritte}
 \begin{itemize}
  \item Source file \cod{src/hello-world.cc}
  \begin{itemize}
   \item C++/POSIX
  \end{itemize}
  unabhängig von Platform
  \item Object file (Maschinencode) \cod{hello-world.o}
  \begin{itemize}
   \item erzeugt mit: \cod{g++ -c ../src/hello-world.cc -o hello-world.o}
   \item Maschinencode: 
   \begin{itemize}
    \item \cod{file hello-world.o} 
    \item \cod{objdump -d hello-world.o}
   \end{itemize}
  \end{itemize}
  \item Executable file \cod{hello-world}
  \begin{itemize}
   \item \cod{g++  hello-world.o -o hello-world}
   \item Maschinencode:
   \begin{itemize}
    \item \cod{file hello-world-c} 
    \item \cod{objdump -d hello-world-c}
   \end{itemize}
  \end{itemize}
 \end{itemize}
\end{frame}

\begin{frame}{In einem Schritt}{für kleine Projekte}
 \begin{itemize}
  \item \cod{g++ ../src/hello-world.cc -o hello-world}
 \end{itemize}
\end{frame}

\begin{frame}{Was es braucht ?}{Files}
 \begin{itemize}
  \item Source file
  \begin{itemize}
   \item wo ist der {\em include file} \cod{iostream}
  \end{itemize}
  \item Object File
  \begin{itemize}
   \item \cod{nm hello-world.o}
   \item wo ist z.B. \cod{\_ZSt4cout}
  \end{itemize}
  \item Executable
  \begin{itemize}
  \item \cod{nm hello-world}
  \item \cod{ldd hello-world}
  \item wo sind die Bibliotheken
  \end{itemize}
 \end{itemize}
\end{frame}

\begin{frame}{Wo sind die Files ?}{irgendwo in einem Unterverzeichnis von {\Huge\tt /}}
 \begin{itemize}
  \item Include Files \cod{g++  -v ../src/hello-world.cc -o hello-world}
  \begin{itemize}
   \item \cod{iostream} ?
  \end{itemize}
  \item Bibliotheken
  \begin{itemize}
   \item z.B. \cod{libc.so}
  \end{itemize}
 \end{itemize}
\end{frame}

\begin{frame}{Development}
 \begin{itemize}
  \item Host==Target
  \item root Host==root Target
 \end{itemize}
\end{frame}

%\subsection{CrossDevelopment}

\begin{frame}{CrossDevelopment}
 \begin{itemize}
  \item Host!=Target
  \item root Host != root Target
 \end{itemize}
\end{frame}



\begin{frame}{CrossDevelopment}{Target \target}
 \begin{itemize}
  \item toolchain
  \begin{itemize}
   \item \cod{tc/bin/arm-linux-gnueabihf-*}
   \item \cod{*}: \cod{g++},\cod{nm},\cod{objdump} $\dots$
  \end{itemize}
  \item \cod{target-root} Mehrere Möglichkeiten:
  \begin{itemize}
   \item Kopie von SD-Karte 
   \item \cod{sshfs debian@192.168.7.2:/ target-root}
  \end{itemize}
 \end{itemize}
\end{frame}

\begin{frame}{CrossDevelopment: im Verzeichnis \cod{target-work}}{die einzelnen Schritte}
 \begin{itemize}
  \item Source file \cod{src/hello-world.cc}
  \begin{itemize}
   \item C++/POSIX
  \end{itemize}
  unabhängig von Platform
  \item Object file (Maschinencode) \cod{hello-world.o}
  \begin{itemize}
   \item erzeugt mit: \cod{../tc/bin/arm-linux-gnueabihf-g++ --sysroot=../target-root -c ../src/hello-world.cc -o hello-world.o}
   \item Maschinencode: 
   \begin{itemize}
    \item \cod{file hello-world.o} 
    \item \cod{../tc/bin/arm-linux-gnueabihf-objdump -d hello-world.o}
   \end{itemize}
  \end{itemize}
  \item Executable file \cod{hello-world}
  \begin{itemize}
   \item \cod{../tc/bin/arm-linux-gnueabihf-g++ --sysroot=../target-root hello-world.o -o hello-world}
   \item Maschinencode:
   \begin{itemize}
    \item \cod{file hello-world-c} 
    \item \cod{../tc/bin/arm-linux-gnueabihf-objdump -d hello-world-c}
   \end{itemize}
  \end{itemize}
 \end{itemize}
\end{frame}

\begin{frame}{In einem Schritt}{für kleine Projekte}
 \begin{itemize}
  \item \cod{../tc/bin/arm-linux-gnueabihf-g++ --sysroot=../target-root ../src/hello-world.cc -o hello-world}
 \end{itemize}
\end{frame}


