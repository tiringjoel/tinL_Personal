\section{CrossDevelopment}
\begin{frame}{Crossdevelopment}
 \begin{itemize}
  \item zwei Rechner
  \begin{description}
   \item[Host] der Entwicklungsrechner
   \item[Target] \targetS der Zielrechner
  \end{description}
  \item Development
  \begin{itemize}
   \item Wo sind die Files
  \end{itemize}
  \item CrossDevelopment
  \begin{itemize}
   \item Wo sind die Files
  \end{itemize}
 \end{itemize}
\end{frame}

\subsection{Outline}
\begin{frame}{Outline}
 \begin{itemize}
  \item Development
  \begin{itemize}
   \item Programme auf dem \host für den \host
  \end{itemize}
  \item CrossDevelopment
  \begin{itemize}
   \item Programme auf dem \host für für den \target
  \end{itemize}
 \end{itemize}
\end{frame}

\begin{frame}{Verzeichnisstruktur}
 \dirtree{%
 .1 6-crossdevelopment.
 .2 src \DTcomment{the source files}.
 .2 tc \DTcomment{target toolchain normally link}.
 .2 target-work \DTcomment{files for \targetS}.
 .2 host-work \DTcomment{files for \host}.
 .2 target-root \DTcomment{copy from SD-card $|$ link $|$ mounted}.
 }
\end{frame}



\section{Aufgaben}

\begin{frame}{Ziel}
 \begin{itemize}
  \item \cod{hello-world} auf dem \host und auf dem \targetS
  \item \cod{primes} auf dem \host und auf dem \targetS
 \end{itemize}
\end{frame}


\begin{frame}{The big Picture}
 \begin{itemize}
  \item Source File: \cod{hello-world.cc}
  \item falls es nicht klapt ?
  \begin{itemize}
   \item wo ist der File ?
  \end{itemize}
 \end{itemize}
\end{frame}


%\subsection{Die Programme}
%\begin{frame}{Development}{\cod{hello-world-c.c}}
%\hspace*{-8mm}
%{
%\begin{tabular}{llllll}
% Host & Target & OS & Toolchain & Verbindung & Bemerkungen\\
% \hline
% \targetS & \targetS & Debian & mitgeliefert&&\\
% \host   & \targetS & Debian & \cod{\tiny tc-tinl-gcc-8.1.0-2018.05.21.tar.gz} & sshfs\\
% \host   & \targetS & minimal & \cod{\tiny tc-tinl-gcc-8.1.0-2018.05.21.tar.gz} & SD-Card  &später\\
% \host   & \targetS & minimal & \cod{\tiny tc-tinl-gcc-8.1.0-2018.05.21.tar.gz} & curlftpfs&später\\
%\end{tabular}
%}
%\remark{Toolchain auf der Cloud: \href{https://drive.switch.ch/index.php/s/A6H382zEGDrgfAL}
%       {\Huge tinL}}
%\end{frame}


%\begin{frame}{Crossdevelopment}{\cod{hello-world-c.c}}
%\begin{itemize}
% \item Target \target Debian
% \item Host \host:
% \begin{itemize}
%  \item \cod{target-root}
%  \item \cod{sshfs}
% \end{itemize} 
% \end{itemize} 
%\end{frame}
%
%\begin{frame}{Crossdevelopment}{\cod{hello-world-c.c}}
% \begin{itemize}
% \item Target \target minimal \cod{mmc0} 
% \item Host \host\ \cod{target-root}:
% \begin{itemize}
%  \item toolchain:
%  \begin{itemize}
%   \item \cod{gcc-6.2.0-arm-64bit.tar.gz} in 
%   \url{sourceforge.net/projects/fhnw-tinl/files}
%  \end{itemize}
%  \item SD-Karte ein-\& ausstecken
%  \item Kopie 
%  \item \cod{curlftpfs}
%  \begin{itemize}
%   \item basiert auf \cod{ftp} 
%  \end{itemize}
% \end{itemize} 
% \end{itemize} 
%\end{frame}
%
%\begin{frame}{Vorbereitung}
% \begin{itemize}
%  \item Verbindung mit \target via \cod{ssh}
%%  \item upgrade \cod{pacman -Suy}
%%  \item {\em user} auf \target \cod{useradd}
%  \item Installation {\em toolchain} auf \host
%  
%  {\tiny\url{sourceforge.net/projects/fhnw-tinl/files/}}
% \end{itemize}
%\end{frame}
%
%\begin{frame}{Verbindung}{mit \target Debian}
% \begin{itemize}
%  \item \cod{ssh} für die Ausführung der Programme
%  \item {\em mount} Varianten
%  \begin{itemize}
%   \item \host auf \target 
%  \end{itemize}
%  \item Kopiere  {\em executable} auf \target 
%   \begin{itemize}
%    \item \cod{scp} secure copy
%    \item \cod{scp {\em executable} user@target:}
%   \end{itemize}
% \end{itemize}
%\end{frame}
%
%\begin{frame}{Programme}{auf \target}
% \begin{itemize}
%  \item \c Programme:, \cod{hello-world-c.c} 
%  \item \cpp Programme \cod{hello-world-cpp.cc}, \cod{primes.cc} 
%  \remark{Toolchain funktioniert für \cpp noch nicht}
% \end{itemize}
%\end{frame}

%\begin{frame}{Verzeichnisstruktur}{\host$\leftrightarrow$ \target}
% \begin{block}{Host} 
% \dirtree{%
%  .1 2-unix-use \DTcomment{somewhere on the host}.
%  .2 config.
%  .3 Makefile \DTcomment{for making \target executables}.
%  .2 src  \DTcomment{c,c++}.
%  .2 tc   \DTcomment{normally toolchain}.
%  .2 {\bf work} \DTcomment{connected with \target {\em current dir}}.
% }
% \end{block}
% \begin{block}{\target} 
% \dirtree{%
%  .1 user \DTcomment{somewhere on the \target}.
%  .2 {\bf work} \DTcomment{connected with \host {\em current dir}}.
% }
% \end{block}
%\end{frame}
