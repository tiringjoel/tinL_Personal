\subsection{CrossDevelopment}

\begin{frame}{CrossDevelopment}
 \begin{itemize}
  \item Host!=Target
  \item root Host != root Target
 \end{itemize}
\end{frame}



\begin{frame}{CrossDevelopment}{Target \target}
 \begin{itemize}
  \item toolchain
  \begin{itemize}
   \item \cod{tc/bin/arm-linux-gnueabihf-*}
   \item \cod{*}: \cod{g++},\cod{nm},\cod{objdump} $\dots$
  \end{itemize}
  \item \cod{target-root} Mehrere Möglichkeiten:
  \begin{itemize}
   \item Kopie von SD-Karte 
   \item \cod{sshfs debian@192.168.7.2:/ target-root}
  \end{itemize}
 \end{itemize}
\end{frame}

\begin{frame}{CrossDevelopment: im Verzeichnis \cod{target-work}}{die einzelnen Schritte}
 \begin{itemize}
  \item Source file \cod{src/hello-world.cc}
  \begin{itemize}
   \item C++/POSIX
  \end{itemize}
  unabhängig von Platform
  \item Object file (Maschinencode) \cod{hello-world.o}
  \begin{itemize}
   \item erzeugt mit: \cod{../tc/bin/arm-linux-gnueabihf-g++ --sysroot=../target-root -c ../src/hello-world.cc -o hello-world.o}
   \item Maschinencode: 
   \begin{itemize}
    \item \cod{file hello-world.o} 
    \item \cod{../tc/bin/arm-linux-gnueabihf-objdump -d hello-world.o}
   \end{itemize}
  \end{itemize}
  \item Executable file \cod{hello-world}
  \begin{itemize}
   \item \cod{../tc/bin/arm-linux-gnueabihf-g++ --sysroot=../target-root hello-world.o -o hello-world}
   \item Maschinencode:
   \begin{itemize}
    \item \cod{file hello-world-c} 
    \item \cod{../tc/bin/arm-linux-gnueabihf-objdump -d hello-world-c}
   \end{itemize}
  \end{itemize}
 \end{itemize}
\end{frame}

\begin{frame}{In einem Schritt}{für kleine Projekte}
 \begin{itemize}
  \item \cod{../tc/bin/arm-linux-gnueabihf-g++ --sysroot=../target-root ../src/hello-world.cc -o hello-world}
 \end{itemize}
\end{frame}
