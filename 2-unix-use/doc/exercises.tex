\section{Aufgaben}
\begin{frame}{Ziel}
 \begin{itemize}
  \item \cod{hello-world-c}/\cod{hello-world-cpp} auf dem \host und auf dem \targetS
  \item \cod{primes} auf dem \host und auf dem \targetS
 \end{itemize}
\end{frame}


\begin{frame}{The big Picture}
 \begin{itemize}
  \item Source File: \cod{hello-world-c.c}
  \item falls es nicht klapt ?
  \begin{itemize}
   \item wo ist der File ?
  \end{itemize}
 \end{itemize}
\end{frame}

\subsection{Einrichten}
\begin{frame}{Einrichten:Host}{Verzeichnisstruktur}
\dirtree{%
 .1 2-unix-use.
 .2 config \DTcomment{git/Makefile}.
 .2 src \DTcomment{git/die source files}.
 .2 tc \DTcomment{selber machen/toolchain link}.
 .2 work \DTcomment{selber machen}.
}
\end{frame}

\begin{frame}{Einrichten:\targetS}{Verzeichnisstruktur}
\dirtree{%
 .1 / \DTcomment{root}.
 .2 home \DTcomment{root}.
 .3 debian \DTcomment{das home}.
 .4 work \DTcomment{verbunden mit \host via sshfs}.
}
\end{frame}

\subsection{Verbinden}

\begin{frame}[fragile]{Verbinden}
 \begin{description}
  \item[mount] \host $\to$ \targetS
  \begin{lstlisting}
   sshfs debian@192.168.7.2:work work
  \end{lstlisting}
 \item[shell]\host $\to$ \targetS
  \begin{lstlisting}
   ssh debian@192.168.7.2
  \end{lstlisting}
 \end{description}
\end{frame}

\subsection{Die Programme}
\begin{frame}{Development}{\cod{hello-world-c.c}}
\hspace*{-8mm}
{
\begin{tabular}{llllll}
 Host & Target & OS & Toolchain & Verbindung & Bemerkungen\\
 \hline
 \targetS & \targetS & Debian & mitgeliefert&&\\
 \host   & \targetS & Debian & \cod{\tiny tc-tinl-gcc-9.2.0-2019.10.09.tar.gz} & sshfs\\
 \host   & \targetS & minimal & \cod{\tiny tc-tinl-gcc-9.2.0-2019.10.09.tar.gz} & SD-Card  &später\\
 \host   & \targetS & minimal & \cod{\tiny tc-tinl-gcc-9.2.0-2019.10.09.tar.gz} & curlftpfs&später\\
\end{tabular}
}
\remark{Toolchain auf der Cloud: \href{https://drive.switch.ch/index.php/s/A6H382zEGDrgfAL}
       {\Huge tinL}}
\end{frame}
