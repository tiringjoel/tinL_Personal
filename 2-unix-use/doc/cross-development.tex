\begin{frame}{Crossdevelopment}
 \begin{itemize}
  \item zwei Rechner
  \begin{description}
   \item[Host] der Entwicklungsrechner
   \item[Target] \targetS der Zielrechner
  \end{description}
  \item Development
  \begin{itemize}
   \item Wo sind die Files
  \end{itemize}
  \item CrossDevelopment
  \begin{itemize}
   \item Wo sind die Files
  \end{itemize}
 \end{itemize}
\end{frame}

\subsection{Development}

\begin{frame}{Development}{noch nicht Cross}
 \begin{itemize}
  \item Source file \cod{src/hello-world-c.c}
  \begin{itemize}
   \item C
   \item POSIX
  \end{itemize}
  unabhängig von Platform
  \item Object file (Maschinencode) \cod{hello-world-c.o}
  \begin{itemize}
   \item erzeugt mit: \cod{gcc -c ../src/hello-world-c.c -o hello-world-c.o}
   \item Maschinencode: 
   \begin{itemize}
    \item \cod{file hello-world-c.o} 
    \item \cod{objdump -d hello-world-c.o}
   \end{itemize}
  \end{itemize}
  \item Executable file \cod{hello-world-c}
  \begin{itemize}
   \item \cod{gcc  hello-world-c.o -o hello-world-c}
   \item Maschinencode:
   \begin{itemize}
    \item \cod{file hello-world-c} 
    \item \cod{objdump -d hello-world-c}
   \end{itemize}
  \end{itemize}
 \end{itemize}
\end{frame}

\begin{frame}{Was es braucht ?}{Files}
 \begin{itemize}
  \item Source file
  \begin{itemize}
   \item wo ist der {\em include file} \cod{stdio.h}
  \end{itemize}
  \item Object File
  \begin{itemize}
   \item \cod{nm hello-world-c.o}
   \item wo ist \cod{puts}
  \end{itemize}
  \item Executable
  \begin{itemize}
  \item \cod{nm hello-world-c}
  \item \cod{ldd hello-world-c}
  \item wo sind die Bibliotheken
  \end{itemize}
 \end{itemize}
\end{frame}

\begin{frame}{Wo sind die Files ?}{irgendwo in einem Unterverzeichnis von {\Huge\tt /}}
 \begin{itemize}
  \item Include Files \cod{gcc -c -v ../src/hello-world-c.c -o hello-world-c.o}
  \begin{itemize}
   \item \cod{stdio.h} ?
   \item \cod{stddef.h} ? 
  \end{itemize}
  \item Bibliotheken
  \begin{itemize}
   \item z.B. \cod{libc.so}
  \end{itemize}
 \end{itemize}
\end{frame}

\begin{frame}{Development}{der Normalfall}
 \begin{itemize}
  \item Host==Target
  \item root Host==root Target
 \end{itemize}
\end{frame}

\subsection{CrossDevelopment}

\begin{frame}{CrossDevelopment}
 \begin{itemize}
  \item immer noch POSIX
  \item Host!=Target
 \end{itemize}
\end{frame}

\begin{frame}{Verzeichnisstruktur}{\host}
 \dirtree{%
  .1 2-unix-use \DTcomment{somewhere on the host}.
  .2 target-root \DTcomment{different possibilities}.
  .2 config.
  .3 Makefile \DTcomment{for making \target executables}.
  .2 src  \DTcomment{c,c++}.
  .2 tc   \DTcomment{toolchain}.
  .2 {\bf work} \DTcomment{connected with \target {\em current dir}}.
 }
\end{frame}

%\begin{frame}{CrossDevelopment}{Target \host:}
% \begin{itemize}
%  \item Makefile
%  \begin{itemize}
%   \item \cod{PREFFIX=}
%  \end{itemize}
%  \item \cod{target-root}
%  \begin{itemize}
%   \item \cod{ln -s / target-root}
%  \end{itemize}
% \end{itemize}
% \remark{unüblich !}
%\end{frame}

\begin{frame}{CrossDevelopment}{Target \target}
 \begin{itemize}
  \item Makefile
  \begin{itemize}
   \item \cod{PREFFIX=../tc/bin/arm-linux-gnueabihf-}
  \end{itemize}
  \item \cod{target-root}
  \begin{itemize}
   \item \cod{sshfs debian@192.168.7.2:/ target-root}
  \end{itemize}
 \end{itemize}
\end{frame}

\begin{frame}{Verzeichnisstruktur}{\target}
 \dirtree{%
  .1 /.
  .2 home.
  .3 debian \DTcomment{user}.
  .4 {\bf work}.
 }
\end{frame}

\section{Aufgaben}

\begin{frame}{Ziel}
 \begin{itemize}
  \item \cod{hello-world} auf dem \host und auf dem \targetS
  \item \cod{primes} auf dem \host und auf dem \targetS
 \end{itemize}
\end{frame}


\begin{frame}{The big Picture}
 \begin{itemize}
  \item Source File: \cod{hello-world.cc}
  \item falls es nicht klapt ?
  \begin{itemize}
   \item wo ist der File ?
  \end{itemize}
 \end{itemize}
\end{frame}


%\subsection{Die Programme}
%\begin{frame}{Development}{\cod{hello-world-c.c}}
%\hspace*{-8mm}
%{
%\begin{tabular}{llllll}
% Host & Target & OS & Toolchain & Verbindung & Bemerkungen\\
% \hline
% \targetS & \targetS & Debian & mitgeliefert&&\\
% \host   & \targetS & Debian & \cod{\tiny tc-tinl-gcc-8.1.0-2018.05.21.tar.gz} & sshfs\\
% \host   & \targetS & minimal & \cod{\tiny tc-tinl-gcc-8.1.0-2018.05.21.tar.gz} & SD-Card  &später\\
% \host   & \targetS & minimal & \cod{\tiny tc-tinl-gcc-8.1.0-2018.05.21.tar.gz} & curlftpfs&später\\
%\end{tabular}
%}
%\remark{Toolchain auf der Cloud: \href{https://drive.switch.ch/index.php/s/A6H382zEGDrgfAL}
%       {\Huge tinL}}
%\end{frame}


%\begin{frame}{Crossdevelopment}{\cod{hello-world-c.c}}
%\begin{itemize}
% \item Target \target Debian
% \item Host \host:
% \begin{itemize}
%  \item \cod{target-root}
%  \item \cod{sshfs}
% \end{itemize} 
% \end{itemize} 
%\end{frame}
%
%\begin{frame}{Crossdevelopment}{\cod{hello-world-c.c}}
% \begin{itemize}
% \item Target \target minimal \cod{mmc0} 
% \item Host \host\ \cod{target-root}:
% \begin{itemize}
%  \item toolchain:
%  \begin{itemize}
%   \item \cod{gcc-6.2.0-arm-64bit.tar.gz} in 
%   \url{sourceforge.net/projects/fhnw-tinl/files}
%  \end{itemize}
%  \item SD-Karte ein-\& ausstecken
%  \item Kopie 
%  \item \cod{curlftpfs}
%  \begin{itemize}
%   \item basiert auf \cod{ftp} 
%  \end{itemize}
% \end{itemize} 
% \end{itemize} 
%\end{frame}
%
%\begin{frame}{Vorbereitung}
% \begin{itemize}
%  \item Verbindung mit \target via \cod{ssh}
%%  \item upgrade \cod{pacman -Suy}
%%  \item {\em user} auf \target \cod{useradd}
%  \item Installation {\em toolchain} auf \host
%  
%  {\tiny\url{sourceforge.net/projects/fhnw-tinl/files/}}
% \end{itemize}
%\end{frame}
%
%\begin{frame}{Verbindung}{mit \target Debian}
% \begin{itemize}
%  \item \cod{ssh} für die Ausführung der Programme
%  \item {\em mount} Varianten
%  \begin{itemize}
%   \item \host auf \target 
%  \end{itemize}
%  \item Kopiere  {\em executable} auf \target 
%   \begin{itemize}
%    \item \cod{scp} secure copy
%    \item \cod{scp {\em executable} user@target:}
%   \end{itemize}
% \end{itemize}
%\end{frame}
%
%\begin{frame}{Programme}{auf \target}
% \begin{itemize}
%  \item \c Programme:, \cod{hello-world-c.c} 
%  \item \cpp Programme \cod{hello-world-cpp.cc}, \cod{primes.cc} 
%  \remark{Toolchain funktioniert für \cpp noch nicht}
% \end{itemize}
%\end{frame}

%\begin{frame}{Verzeichnisstruktur}{\host$\leftrightarrow$ \target}
% \begin{block}{Host} 
% \dirtree{%
%  .1 2-unix-use \DTcomment{somewhere on the host}.
%  .2 config.
%  .3 Makefile \DTcomment{for making \target executables}.
%  .2 src  \DTcomment{c,c++}.
%  .2 tc   \DTcomment{normally toolchain}.
%  .2 {\bf work} \DTcomment{connected with \target {\em current dir}}.
% }
% \end{block}
% \begin{block}{\target} 
% \dirtree{%
%  .1 user \DTcomment{somewhere on the \target}.
%  .2 {\bf work} \DTcomment{connected with \host {\em current dir}}.
% }
% \end{block}
%\end{frame}
