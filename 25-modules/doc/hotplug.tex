\subsection{Hotplug}
\begin{frame}{Hotplug}{/proc/sys/kernel/hotplug}
 \begin{itemize}
  \item Kernel: \ksp entdeckt ein neues Ger�t
  \item Informiert (notify) User \usp: neues Ger�t 
  \begin{itemize}
   \item call-back: \cod{/proc/sys/kernel/hotplug}
   \begin{itemize}
    \item enth�lt \usp executable
   \end{itemize}
  \end{itemize}
 \end{itemize}
 \remark{Kernel muss f�r {\em hotplug} konfiguriert sein}
\end{frame}

{
\begin{frame}[fragile]{Beispiel}{ein Skript im \usp}
\begin{lstlisting}[language=bash]
#!/bin/ash 
#---------------------
#my-hotplug.sh
#(c) H.Buchmann FHNW 2018
#---------------------
echo "{----- my-hotplug ---- ${@}" >> /home/root/my-hotplug.log
env >> /home/root/my-hotplug.log
echo "----- my-hotplug ---- }" >> /home/root/my-hotplug.log
\end{lstlisting}
\begin{itemize}
 \item \cod{chmod a+x my-hotplug.sh} :executable
 \item \cod{echo /home/root/my-hotplug.sh > /proc/sys/kernel/hotplug}
\end{itemize}
\end{frame}
}

\begin{frame}{Test}
 \begin{itemize}
  \item shell1: 
  \begin{itemize}
   \item \cod{tail -f //home/root/my-hotplug.log}
  \end{itemize}
  \item shell2:
   \begin{itemize}
    \item \cod{insmod simple-device-5.ko}
    \item \cod{rmmod simple-device-5.ko}
   \end{itemize}
 \end{itemize}
\end{frame}
