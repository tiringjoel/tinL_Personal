\subsection{simple-device-2.c}
\begin{frame}{\cod{simple-device-2.c}}{empty read/write}
 \begin{itemize}
  \item {\em kernel-space}
  \begin{itemize}
  \item \cod{file\_operations}: read/write
  \begin{itemize}
   \item mit \cod{printk} call-back anzeigen
  \end{itemize}
  \item verschiedene \cod{return} Werte für read/write
  \end{itemize}
  \item {\em user-space}
  \begin{itemize}
  \item \cod{mknod device c {\em Major} 0} 
  \item read: \cod{cat device}
  \item write wenig Bytes \cod{echo abcd > device} 
  \item viel Bytes \cod{cat file > device}
  \end{itemize}
 \end{itemize}
\end{frame}

\begin{frame}[fragile]{Code}
 \begin{lstlisting}
 printk("simple_read len=%d= *ofs= %lld buffer*=0x%p\n",
                         len,      *ofs,          buffer);
 \end{lstlisting}
\remark{wie \cod{printf} im \usp}
\begin{lstlisting}
static struct file_operations fops =  /* the call backs */
{
 read :simple_read, /* register call-backs */
 write:simple_write,
};
\end{lstlisting}
\end{frame}
