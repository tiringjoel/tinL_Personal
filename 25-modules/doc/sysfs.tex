%-------------------------
%device-revisited.tex
%(c) H.Buchmann FHNW 2018
%export TEXINPUTS=${HOME}/fhnw/edu/:${HOME}/fhnw/edu/tinL/config/latex:${HOME}/fhnw/edu/config//:
%-------------------------
\documentclass{beamer}
\usepackage{latex/beamer}
%---------------------
%local defines
%(c) H.Buchmann FHNW 2009
%$Id$
%---------------------
\def \target {\raspberry\xspace}
\def \host {{\em Host \xspace}}

\input{/home/buchmann/latex/dirtree/dirtree.tex}
\usepackage{svg}
\usepackage[absolute]{textpos}
\setlength{\TPHorizModule}{1mm}
\setlength{\TPVertModule}{1mm}

\begin{document}


\newcommand{\ksp}{{\em kernel-space}\xspace}
\newcommand{\usp}{{\em user-space}\xspace}

\title[sysfs]{Das Filesystem \\\cod{sysfs}}

\frame{\titlepage}

\begin{frame}{Um was geht es ?}{\cod{/sys}: die Verbindung \usp $\leftrightarrow$ \ksp}
\dirtree{%
.1 / \DTcomment{root}.
.2 sys.
.3 {...} \DTcomment{ziemlich tiefe Verzeichnisstruktur}.
}
 \begin{itemize}
  \item \ksp
  \begin{itemize}
    \item erzeut Verzeichnisse und Files
    \item notifiziert: call-back
  \end{itemize} 
  \item \usp
  \begin{itemize}
   \item reagiert 
   \item liest/schreibt auf Files
  \end{itemize} 
 \end{itemize} 
 \begin{block}{Info}
 \begin{itemize}
  \item \cod{Documentation/filesystems/sysfs.txt}
  \item \url{lxr.free-electrons.com}
 \end{itemize}
 \end{block}
\end{frame}


\begin{frame}{Beispiele}
 \begin{itemize}
  \item \cod{/sys/class/hwmon}
  \item \cod{/sys/class/leds}
  \item \cod{/sys/class/backlight}
  \item \cod{/sys/class/rtc}
 \end{itemize}
 \remark{Vorsicht}
\end{frame}

\begin{frame}{Unsere Module}
 \begin{itemize}
  \item \cod{sysfs-0.c} erzeugt Directory in \cod{/sys}
  \item \cod{sysfs-1.c} erzeugt File
  \item \cod{sysfs-2.c} read/write
  \begin{itemize}
   \item call-backs
  \end{itemize}
 \end{itemize}
\end{frame}

\begin{frame}{\cod{sysfs-0.c}}{erzeugt Directory in \cod{/sys}}
 \begin{itemize}
  \item \href{https://elixir.bootlin.com/linux/latest/source/include/linux/kobject.h\#L98}
             {\cod{kobject\_create\_and\_add}}
  \item \href{https://elixir.bootlin.com/linux/latest/source/include/linux/kobject.h\#L114}
             {\cod{kobject\_put}}
 \end{itemize}
\end{frame}

\begin{frame}[fragile]{\cod{sysfs-1.c}}{erzeugt File}
\begin{itemize}
 \item \href{https://elixir.bootlin.com/linux/latest/source/include/linux/sysfs.h\#L496}
            {\cod{sysfs\_create\_file}}
 \item \href{https://elixir.bootlin.com/linux/latest/source/include/linux/sysfs.h\#L30}
            {\cod{attribute}}
 \end{itemize}
 \begin{block}{Initialisierung von \cod{structs}}
\begin{lstlisting}
static struct attribute attr=
{
 name: ...., /* some reasonable value */
 mode: ....  /* some reasonable value */
};
\end{lstlisting}
 \end{block}
 \begin{block}{Read/Write}
  \begin{itemize}
   \item \cod{cat cat /sys/my-kobj/file}
   \item \cod{echo abc >  /sys/my-kobj/file}
  \end{itemize}
 \end{block}
\end{frame}

\begin{frame}[fragile]{\cod{sysfs-2.c}}{read/write}
\begin{itemize}
 \item \href{https://elixir.bootlin.com/linux/latest/source/include/linux/kobject.h\#L142}
            {kobj\_attribute}
 \item call-back           
\end{itemize}
\begin{lstlisting}
struct kobj_attribute {
       struct attribute attr;
       ssize_t (*show)(struct kobject *kobj, 
                       struct kobj_attribute *attr,
		       char *buf);
       ssize_t (*store)(struct kobject *kobj, 
                        struct kobj_attribute *attr,
			const char *buf, size_t count);
};
\end{lstlisting} 

\end{frame}

\end{document}
