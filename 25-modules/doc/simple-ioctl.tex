\begin{frame}{ioctl}{Die Kontrolle von Files/devices}
 \begin{itemize}
  \item alles ist ein File
  \begin{itemize}
   \item besser: ein {\em stream of bits}
  \end{itemize}
  \item aber:
  \begin{itemize}
   \item der {\em stream of bits} muss kontrolliert/konfiguriert werden
  \end{itemize}
 \end{itemize}
\end{frame}

\begin{frame}{Beispiel:Serielle Schnittstelle}{Baudrate}
 \begin{itemize}
  \item \cod{ioctl-c-example.c} \c Programm
  \item \cod{ioctl-cpp-example.cc} \cpp Programm
 \end{itemize}
\end{frame}

\begin{frame}{Typischer Ablauf}
 \begin{enumerate}
  \item öffne File: {\em stream of bits} \cod{fid} 
  \begin{itemize}
   \item \cod{fid=open(name,mode)}
  \end{itemize}
  \item konfiguriere {\em stream of bits}
  \begin{itemize}
   \item \cod{ioctl(fid,{\em what},{\em parameter})}
  \end{itemize}
  \item lese/schreibe
  \begin{itemize}
   \item \cod{read(fid,data,length)}/\cod{write(fid,data,length)}
  \end{itemize}
  \item schliesse:
  \begin{itemize}
   \item \cod{close(fid)}
  \end{itemize}
 \end{enumerate}
\end{frame}

\begin{frame}{Ziel}{\src{simple-ioctl.c|h} und \src{simple-ioctl-userspace.c}}
 \begin{itemize}
  \item Einstellungen
  \item \cod{ioctl(fileId,cmd,data)}
 \end{itemize}
\end{frame}

\begin{frame}{ioctl - control device}{userspace}
\begin{itemize}
 \item \cod{man 2 ioctl}
\begin{quote}
{\footnotesize
The  {\bf ioctl()} function manipulates the underlying device parameters of special
files.  In particular, many operating characteristics of character special files
(e.g., terminals) may be controlled with {\bf ioctl()} requests.  The argument d
must be an open file descriptor
}
\end{quote}
 \item \cod{int ioctl(int d, unsigned long request, ...);}
\end{itemize}
\end{frame}

\begin{frame}[fragile]{Im userspace}{\src{simple-ioctl-userspace.c}}
\begin{itemize}
 \item die {\em requests} \src{simple-ioctl.h}
 \begin{lstlisting}
#define SIMPLE_IOCTL_WRITE _IOW(0x23,5,int)  
#define SIMPLE_IOCTL_READ  _IOR(0x23,5,int) 
 \end{lstlisting}
 \item \src{simple-ioctl-userspace.c}
 \begin{itemize}
  \item read
  \vspace{-2mm}
  \begin{lstlisting}
int val=1;
int res=ioctl(id,SIMPLE_IOCTL_READ,&val);
  \end{lstlisting}
  \item write
  \vspace{-2mm}
  \begin{lstlisting}
int res=ioctl(id,SIMPLE_IOCTL_WRITE,0x1234);
  \end{lstlisting}
 \end{itemize}
\end{itemize}
\end{frame}

\begin{frame}{Test}{siehe \src{simple-device.c} Test}
 \begin{itemize}
  \item \cod{insmod}
  \item \cod{mknod}
  \item \cod{./simple-ioctl-userspace}
 \end{itemize}
\end{frame}
