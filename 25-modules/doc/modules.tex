%-------------------------
%makefile
%(c) H.Buchmann FHNW 2008
%export TEXINPUTS=${HOME}/fhnw/edu/:${HOME}/fhnw/edu/tinL/config/latex:${HOME}/fhnw/edu/config//:
%-------------------------
\documentclass{beamer}
\usepackage{latex/beamer}
%---------------------
%local defines
%(c) H.Buchmann FHNW 2009
%$Id$
%---------------------
\newcommand{\target} {\beaglebone\xspace}
\newcommand{\targetS}{{\bf BBG}\xspace}
\newcommand{\host}   {{\em Host}\xspace}
\newcommand{\targetroot} {{\bf target-root}\xspace}
\newcommand{\kernel} {{\bf kernel}\xspace}
\renewcommand{\c}{{\bf C}\xspace}
\newcommand{\cpp}{{\bf C++}\xspace}
\newcommand{\posix}{{\bf POSIX}\xspace}

\input{/home/buchmann/latex/dirtree/dirtree.tex}

\usepackage[absolute]{textpos}
\setlength{\TPHorizModule}{1mm}
\setlength{\TPVertModule}{1mm}

\begin{document}


\title[Kernel Module]{Kernel Module}

\frame{\titlepage}

\begin{frame}{Um was geht es ?}
 \begin{itemize}
  \item Code f�r dem {\em kernel}: Drivers
  \item Den {\em kernel} nicht immer neu kompilieren 
  \item Module laden/l�schen
 \end{itemize}
\end{frame}

\begin{frame}{Informationen}
 \begin{itemize}
  \item \url{tldp.org/LDP/lkmpg/2.6/html/}
  \item \url{Documentation/kbuild/modules.txt}
  \item \url{www.kernel.org/doc/}
  \item \url{lxr.free-electrons.com}
 \end{itemize}
\end{frame}

\section{Beispiele}
\subsection{simple-device}
\begin{frame}{simple-module}{Modul f�r \host}
 \begin{description}[Makefile]
  \item[Code]     \cod{scr/simple-module.c}
  \item[Makefile] \cod{src/Makefile} von linux Makefile aufgerufen
  \item[Script]   \cod{config/make-it.sh} f�r einfachen Aufruf
  \item[Test]
      \begin{itemize}
        \item \cod{tail -f /var/log/kern.log} f�r die {\em kernel messages}
	\item \cod{dmesg -w}
        \item \cod{sudo insmod simple-module.ko} wir sind in \cod{src} 
	\item \cod{lsmod | grep simple} ist installiert
	\item \cod{sudo rmmod simple-module} deinstalliert
	\item Der File \url{proc/modules}
      \end{itemize}
 \end{description}
\end{frame}

\subsection{simple-device}

\begin{frame}{simple-device}
 \begin{description}
  \item[Module] \cod{src/simple-device.c}, \cod{src/Makefile}
  \item[Test] 
     \begin{itemize}
       \item \cod{insmod simple-device.ko} Major Number
       \item \cod{mknod -ma=rw simple c Major 1} Device File Zugriff f�r alle
       \item \cod{cat simple} read 
       \item \cod{cat aFile > simple} write
     \end{itemize} 
 \end{description}
 \begin{todos}
  \item Automatische Erzeugung von \cod{simple}
  \item die \cod{file\_operations}: \cod{open} und \cod{close}
 \end{todos}
\end{frame}

\begin{frame}{\cod{simple-device.c}}{userspace $\leftrightarrow$ kernelspace}
 \begin{description}[kernelspace $\to$ userspace]
  \item[Module] brauche  \cod{print\_hex\_dump} eine praktische Funktion
  \item[userspace $\to$ kernelspace] \cod{simple\_write}
  \item[kernelspace $\to$ userspace] \cod{simple\_read}
  \item[Test] \cod{simple-device} im userspace
 \end{description}
 \todo{Erzeuge Absturz}
\end{frame}

\subsection{simple-ioctl}
\begin{frame}{\cod{simple-ioctl.c}}
\end{frame}


%\begin{frame}{TODO}
% \begin{itemize}
%  \item call-back demo
%  \item userspace - kernelspace
%  \item Files
%  \item open,close,read, write
%  \item ioctl
%  \item roadmap
%  
% \end{itemize}
%\end{frame}
%
\begin{frame}{Aufgaben}
 \begin{itemize}
  \item \url{scr/simple-module.c} f�r \target
  \begin{itemize}
   \item Machen Sie eine 'ewige Schlaufe'
  \end{itemize}
  \item \url{read-device.c} \url{scr/simple-device.c} f�r \target
  \item \url{call-ioctl.c} \url{scr/simple-ioctl.c} f�r \target

%  \item Das {\em userspace }Programm \url{9-accessing-hw/hw-access.c}
%  als Modul mit \cod{ioctl} 
 \end{itemize}
\end{frame}


%\section{Interrupt}
%\begin{frame}{Die Hardware ruft die Software}{call back}
% \begin{itemize}
%  \item \cod{9-accessing-hw/button.c} userland 
%  \item \cod{10-module/button.c} als {\em device}
%  \begin{itemize}
%   \item \cod{ioremap}
%  \end{itemize}
% \end{itemize}
%\end{frame}
%
%\begin{frame}{\cod{button.c}}{Wichtige Punkte}
% \begin{itemize}
%  \item \cod{typedef PIO}
%  \item \cod{static PIO* \_\_iomem pioC}
%  \item \cod{ioremap(BASE,SIZE)}
% \end{itemize}
%\end{frame}
%
%\begin{frame}{static PIO* \_\_iomem pioC}
%  \fig{pio.pdf}{1}{0}
%\end{frame}
%
%\begin{frame}{\cod{ioremap(BASE,SIZE)}}{Memory Managament Unit}
% \begin{columns}
%  \begin{column}{6cm}
%   \fig{memory-map.pdf}{0.625}{0}
%  \end{column}
%  \begin{column}{6cm}
%   \begin{description}
%    \item[p-mem] Physikalisches Memory
%    \item[v-mem] Virtuelles Memory
%    \item[BASE] Page aligned
%    \item[offset] innerhalb Page
%    \item[size] Page Size \cod{4kiB = 0x1000}
%   \end{description}
%  \end{column}
% \end{columns}
%\end{frame}
%
%\begin{frame}{Aufgaben}
% \begin{itemize}
%  \item NFS Network File System einrichten
%  \item Das Modul \cod{button.c}
%  \item Die Probleme mit dem Interrupt
%  \item Brauche \url{lxr.free-electrons.com/}
%  \item Codedemo mit modifiziertem Code \cod{kernel/irq/manage.c}
%  \item \cod{cat /proc/interrupts}
% \end{itemize}
%\end{frame}
\end{document}
