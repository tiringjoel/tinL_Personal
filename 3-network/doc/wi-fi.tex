\begin{frame}{Outline}{Wi-Fi von Hand}
 \begin{itemize}
  \item normalerweise automatische Konfiguration
  \item einmal von Hand
  \item wo sind die Passw�rter
  \item zwei Sachen 
  \begin{itemize}
   \item das Wireless Netz
   \item das Internet
  \end{itemize}
 \end{itemize}
\end{frame}

\subsection{Das Wireless Netz}
\begin{frame}{Das Wireless Netz}
\begin{block}{Information}
 \begin{itemize}
  \item \cod{iw dev}
  \item \cod{ip link show} 
 \end{itemize}
\end{block}
\begin{block}{Start}
 \begin{itemize}
  \item \cod{ip link set wlan0 up}
 \end{itemize}
\end{block}
\begin{block}{Examine}
 \begin{itemize}
  \item \cod{ip link show wlan0}
  \item \cod{iw wlan0 scan} bzw. \cod{iw wlan0 scan | grep SSID}
 \end{itemize}
\end{block}
\end{frame}


\begin{frame}[fragile]{Das Wireless Netz}{Connect}
\begin{block}{fhnw-public}
\begin{itemize}
 \item \cod{iwconfig wlan0 essid fhnw-public}
\end{itemize}
\end{block}
\begin{block}{eduroam}
 \begin{itemize}
  \item \cod{wpa\_supplicant  -D wext -i wlan0 -c {\em config-file}}
  \item Erzeuge \cod{{\em config-file}} auf dem \host:
  \begin{lstlisting}[language=bash]
./tools/wpa_eduroam.sh
  \end{lstlisting}
  \end{itemize}
\end{block}
\end{frame}

\subsection{Internet}
\begin{frame}{Internet}
\begin{block}{Connect}
 \begin{itemize}
  \item \cod{dhclient wlan0}
 \end{itemize}
\end{block}
\begin{block}{Test}
 \begin{itemize}
  \item \cod{ping ...}
  \item \cod{curl ...}
 \end{itemize}
\end{block}
\end{frame}

