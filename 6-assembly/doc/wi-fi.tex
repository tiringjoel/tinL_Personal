\subsection{wi-fi}

\begin{frame}{Kernel}{Konfiguration}
{\footnotesize
 \dirtree{%
 .1 Kernel Configuration.
 .2 Networking support.
 .3 Wireless.
 .4 <*> cfg80211 - wireless configuration API.
 .4 <*> Generic IEEE 802.11 Networking Stack (mac80211).
 .2 Device Drivers.
 .3 Network device support.
 .4 Wireless LAN.
 .5 <*> TI wl18xx support.
 .5 <*> TI wlcore SPI support.
 .5 <*> TI wlcore SDIO support.
 }
}
\vspace{-5mm}
\begin{block}{Resultate}
 \begin{itemize}
  \item \cod{dmesg | grep wl}
 \end{itemize}
\end{block}
\end{frame}

\begin{frame}{Firmware}{Weitere Konfiguration}
 {\footnotesize
 \dirtree{%
 .1 Kernel Configuration.
 .2 Device Drivers.
 .3 Generic Driver Options.
 .4 [*] Include in-kernel firmware blobs in kernel binary.
 }
}
\vspace{-5mm}
\begin{block}{Resultate}
 \begin{itemize}
  \item \cod{dmesg | grep wl}
  \item \cod{ip link set wlan0 up}
  \item \cod{iw wlan0 scan}
  \item Weitere Konfiguration
 \end{itemize}
\end{block}
\end{frame}

\begin{frame}[fragile]{wpa}
 \begin{itemize}
  \item \cod{wpa\_passphrase}
  \item eduroam Versuch
{\footnotesize
\begin{verbatim}
network={
  ssid="eduroam"
  key_mgmt=WPA-EAP
  identity="hans.buchmann@fhnw.ch"
  domain_suffix_match="welcome.fhnw.ch"
  phase2="auth=MSCHAPV2"
  password="----"
}
\end{verbatim}
}
 \end{itemize}
\end{frame}
