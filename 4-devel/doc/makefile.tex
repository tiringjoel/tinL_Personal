%-------------------------
%makefile
%(c) H.Buchmann FHNW 2015
%export TEXINPUTS=.:${HOME}/fhnw/edu/:${HOME}/fhnw/edu/tinL/config/latex:${HOME}/fhnw/edu/config//:
%-------------------------
\title{Makefile}

\frame{\titlepage}

\section{�berblick}
\begin{frame}{Programmentwicklung}{von der {\em Source} zum {\em Image}}
 \begin{description}
  \item[Gegeben:] SourceFiles: viele Files
  \item[Gesucht:] ImageFile: ein File
 \end{description}
\end{frame}


\begin{frame}{Programmentwicklung}{Files sind die Grundelemente}
 \begin{itemize}
   \item Klassische Programmentwicklung
   \item Verschiedene Arten von \alert{Files}
   \item Programme/Tools erzeugen die Files
   \item Die Files h�ngen voneinander ab
   \item F�r etwas komplexere Projekte gibt es viele Files $\approx 100$
 \end{itemize}
\end{frame}

\begin{frame}{Ein grosses Projekt}{\linux}
 \begin{itemize}
  \item Anzahl Files 
  \begin{itemize}
    \item \cod{tools/count-files.sh}
  \end{itemize}
  \item SLOC: Source Line Of Code
  \begin{itemize}
   \item \cod{tools/sloc-count.sh}
   \item Analyse mit z.B. {\em excel}
  \end{itemize}
 \end{itemize}
\end{frame}

\begin{frame}{Der File {\tt Makefile}}{das Programm {\tt make}}
 \begin{description}[Makefile]
  \item[Makefile] Muss selber geschrieben werden:\\
   Beschreibt, wie Files gemacht werden.
   \remark{Es gibt Programme z.B. \cod{automake} die erzeugen \cod{Makefile}'s}
  \item[make] Programm: \\
  interpretiert den \cod{Makefile} 
 \end{description}
 
\end{frame}

\begin{frame}{Dokumentation}
 \url{http://www.gnu.org/software/make/manual/make.html}
\end{frame}

\begin{frame}{make: Aufruf}
 \begin{code}
  make name
 \end{code}
 \begin{description}
  \item[make] Das Programm
  \item[name] Name des Files, der hergestellt werden soll 
       \footnote{Allgemeiner: \cod{name} ist der Name einer Regel}
  \item[Makefile] muss nicht angegeben werden. \cod{make} sucht den
  File mit dem Namen \cod{Makefile} im {\em current directory}
  \begin{itemize}
   \item Alternative:
   
   \cod{make -f {\em path-to-makefile} name}
  \end{itemize}
 \end{description}
\end{frame}

\section{Aufbau}
\begin{frame}{Makefile: Struktur}
 \begin{description}[Variablen]
  \item[Variablen] Siehe \codelink{config/Makefile}
  \item[Rules]  der wichtige Teil
  \remark{Eine Regel beschreibt Abh�ngigkeiten}
 \end{description}
\end{frame}


\subsection{Regeln}
\begin{frame}
 \frametitle{Makefile: {\em rule} Regel}
 \begin{code}
  target:  $\to$ file1 file2 file3 ..\\
\ \ $\to$ tool      
 \end{code}
 \remark{'$\to$' steht f�r das unsichtbare {\em Tabulator} Zeichen}
 \begin{description}
  \item[target] File, der hergestellt wird
  \item[file1,file2..] {\em prerequisites} Files, die es braucht um das
  \cod{target} herzustellen
  \item[tool] Programm, das aus den {\em prerequisites} das \cod{target}
      herstellt. 
      \begin{itemize}
       \item Muss normalerweise nicht angegeben werden. \cod{make}
      kann aus den {\em Fileextensions} das \cod{tool} bestimmen.
      \end{itemize}
 \end{description}
\end{frame}

\subsection{Unser Beispiel}

\begin{frame}{Ziel}
\begin{block}{Programme}
 \begin{enumerate}
  \item lauff�hig auf \host
  \item lauff�hig auf \target
 \end{enumerate}
 dank {\bf POSIX}
\end{block}

\end{frame}

\begin{frame}{Verzeichnisstruktur}{auf dem \host}
 \dirtree{%
  .1 somewhere \DTcomment{on your \host}.
  .2 src \DTcomment{home of source files}.
  .2 config \DTcomment{home of configuration files}.
  .3 Makefile.
  .2 work \DTcomment{workspace, connected with \targetS}.
  .3 Makefile $\to$ ../config/Makefile \DTcomment{link}.
 }
 \remark{Wie immer!}
\end{frame}

\begin{frame}[fragile]{Abh�ngigkeiten}{9 Files}
\vspace{-2cm}
\fig{rules.pdf}{0.5}{0}
\vspace{-1cm}
\listing{rules}
\end{frame}

\begin{frame}{Die Operationen}
 \begin{tabular}{lll}
 target & prerequisites & action\\
 \hline
 \cod{udp-server-demo}:	&\cod{udp-server-demo.o udp.o} & link\\
 \cod{udp-client-demo}:	&\cod{udp-client-demo.o udp.o} & link \\
 \cod{udp-server-demo.o}:	&\cod{udp-server-demo.c udp.h} & compile \\
 \cod{udp-client-demo.o}:	&\cod{udp-client-demo.c udp.h} & compile\\
 \cod{udp.o}:			&\cod{udp.c udp.h} & compile
 \end{tabular}
\end{frame}

\begin{frame}{Die Include Files}
 Die {\em include files}:
 \begin{itemize}
  \item m�ssen im \cod{Makefile} angegeben werden
  \item werden erst im vom Pr�prozessor includiert
 \end{itemize} 
\end{frame}

\begin{frame}{Die Include Files}{Andere Sichtweise}
 \fig{rules-1.pdf}{0.5}{0}
\end{frame}

\section{Aufgaben}

\begin{frame}{Aufgabe}
 \begin{itemize}
  \item Anpassung an \target
  \item F�r \target {\em und} \host
  \begin{itemize}
   \item ist POSIX
  \end{itemize}
  \item Nutzen Sie die {\em tools}
  \begin{itemize}
   \item \cod{netcat}
   \item \cod{wireshark}
  \end{itemize}
 \end{itemize}
\end{frame}
