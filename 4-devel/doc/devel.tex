%-------------------------
%big-picture
%(c) H.Buchmann FHNW 2008
%$Id$
%export TEXINPUTS=${HOME}/fhnw/edu/:${HOME}/fhnw/edu/tinL/config/latex:${HOME}/fhnw/edu/config//:
%-------------------------
\documentclass{beamer}
\usepackage{latex/beamer}
%---------------------
%local defines
%(c) H.Buchmann FHNW 2009
%$Id$
%---------------------
\newcommand{\target} {\beaglebone\xspace}
\newcommand{\targetS}{{\bf BBG}\xspace}
\newcommand{\host}   {{\em Host}\xspace}
\newcommand{\targetroot} {{\bf target-root}\xspace}
\newcommand{\kernel} {{\bf kernel}\xspace}
\renewcommand{\c}{{\bf C}\xspace}
\newcommand{\cpp}{{\bf C++}\xspace}
\newcommand{\posix}{{\bf POSIX}\xspace}

\input{/home/buchmann/latex/dirtree/dirtree.tex}

\title{Programmentwicklung}
\begin{document}

\frame{\titlepage}

\begin{frame}{Ziele}{Programmierung auf dem \target}
 \begin{itemize}
  \item (fast) wie auf dem \host
  \item {\em Toolchain} auf dem \host
  \item nur Programme ({\em runtime}) auf dem \target
  \item Entwicklung f�r 
  \begin{description}[C/C++]
   \item[Java]  Java SE Runtime Environment \target
   \item[C/C++] Posix {\em runtime} 
  \end{description}
  vom \target aus gesehen
 \end{itemize}
\end{frame}

\begin{frame}{Outline}{f�r Java und C/C++}
 \begin{block}{\host}
  \dirtree{%
   .1 somewhere\_on\_your\_host.
   .2 config.
   .2 src \DTcomment{the own source files}.
   .2 work \DTcomment{seen by \target}.
   .2 target-root \DTcomment{for the toolchain}.
   .2 tc \DTcomment{toolchain}.
  }
 \end{block}
 \begin{block}{\target}
  \dirtree{%
   .1 somewhere\_on\_your\_\target.
   .2 work \DTcomment{mounted on \host per \cod{sshfs}}.
  }
 \end{block}
\end{frame}

\section{Java}
\begin{frame}{Entwicklung}
 \begin{block}{\host}
  \begin{itemize}
   \item Toolchain sollte schon vorhanden sein
   \item Beispiel \cod{java/HelloWorld.java}
   \item Herstellung \cod{java -d. {\em sourceFile}}
  \end{itemize}
 \end{block}
 \begin{block}{\target}
  \begin{itemize}
   \item Runtime \cod{jre7-openjdk-headless}
  \end{itemize}
 \end{block}
 \remark{Suche kleine Runtime}
\end{frame}

\section{C/C++}
\begin{frame}{Entwicklung}
 \begin{block}{\host}
  \begin{itemize}
   \item Toolchain \url{http://sourceforge.net/projects/fhnw-tinl/files/}
   \item Beispiele: \cod{src/*}
   \item Herstellung: \cod{make -f ../config/Makefile {\em the-app}}
  \end{itemize}
 \end{block}
 \begin{block}{\target}
  \begin{itemize}
   \item Runtime \linux POSIX
  \end{itemize}
 \end{block}
\end{frame}
\end{document}

