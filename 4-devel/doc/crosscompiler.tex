\renewcommand{\src}[1]{\cod{{\bf #1}.src}}
\newcommand{\exe}[2]{\cod{{\bf #1}(#2)}}
\newcommand{\compiler}[3]
% #1 file
% #2 exec
% #3 target
{
 \src{#1}\to\boxed{\exe{compiler}{#2}}\to\exe{#1}{#3}
}
\begin{frame}{Crosscompiler}{Notationen}
\begin{description}[Targetrechner]
 \item[Hostrechner] $H$
 \item[Targetrechner] $T$ 
  \begin{itemize}
   \item Beispiel \target
  \end{itemize}
 \item[Sourcefile] \src{file}
  \begin{itemize}
   \item Beispiel \cod{hello-world.c}
  \end{itemize}
 \item[Executable] \exe{file}{M} ausf�hrbar auf dem Rechner $M$, \\$M=H|T$
  \begin{itemize}
   \item Beispiel \cod{hello-world(T)} \\f�r \target
  \end{itemize}
\end{description}
\end{frame}

\begin{frame}{Compiler}{Definition}
\[
\compiler{file}{M}{N}
\]
\begin{description}[$\exe{compiler}{M}$]
 \item[$\src{file}$] der Source File
 \item[$\exe{compiler}{M}$] der Compiler ein {\em executable} f�r den Rechner $M$
 \item[$\exe{file}{N}$] das {\em executable} f�r den Rechner $N$
\end{description}
\end{frame}

\begin{frame}[fragile]{Beispiel}{Programm auf dem \host}
 \[
  \compiler{{hello\_world}}{H}{H}
 \]
 \begin{lstlisting}{language=bash}
  gcc -O2  -std=c99 \
  ../src/hello-world-c.c \
  -o hello-world-c
 \end{lstlisting}
\end{frame}

\begin{frame}[fragile]{Beispiel}{Crosscompilation}
 \[
  \compiler{{hello\_world}}{H}{T}
 \]
 \begin{lstlisting}{language=bash}
 ../tc/bin/arm-linux-gnueabihf-gcc -O2 \
  --sysroot=../target-root -std=c99\
  ../src/hello-world-c.c \
  -o hello-world-c
 \end{lstlisting}
\end{frame}

\begin{frame}{Crosscompiler}{Herstellung}
\[
 \compiler{compiler}{H}{H}
\]
\end{frame}
