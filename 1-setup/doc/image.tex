\begin{frame}{Image}{F�r \target}

\begin{itemize}
 \item Bootcode
 \item Kernel
 \item \unix
\end{itemize}
\end{frame}


\begin{frame}{Die Files}
 \begin{description}
  \item[Original] 
    {\footnotesize
    \url{http://downloads.raspberrypi.org/arch_latest}
    }
  \item[Modifikation]
    {\footnotesize
    \url{http://sourceforge.net/projects/fhnw-tinl/files}
    }
 \end{description}
\end{frame}

\begin{frame}[fragile]{Inhalt}{Partitionen}
 \begin{itemize}
  \item \cod{fdisk -l {\em image}}
\end{itemize}
{\tiny
\begin{verbatim}
Disk ArchLinuxARM-2014.06-rpi.img: 1.8 GiB, 1960837120 bytes, 3829760 sectors
Units: sectors of 1 * 512 = 512 bytes
Sector size (logical/physical): 512 bytes / 512 bytes
I/O size (minimum/optimal): 512 bytes / 512 bytes
Disklabel type: dos
Disk identifier: 0x417ee54b

Device                        Boot  Start     End Sectors  Size Id Type
ArchLinuxARM-2014.06-rpi.img1        2048  186367  184320   90M  c W95 FAT32 (LB
ArchLinuxARM-2014.06-rpi.img2      186368 3667967 3481600  1.7G  5 Extended
ArchLinuxARM-2014.06-rpi.img5      188416 3667967 3479552  1.7G 83 Linux
\end{verbatim}
}
\end{frame}

\begin{frame}{Partitionen}{vom Original}
 \begin{description}
  \item[ArchLinuxARM-2014.06-rpi.img1] Boot Kernel
  \item[ArchLinuxARM-2014.06-rpi.img2] Extended
  \item[ArchLinuxARM-2014.06-rpi.img5] \unix 
 \end{description}
 \remark{Extended braucht es nicht}
\end{frame}

\begin{frame}{Transfer $\to$ \target}{Verschiedene Verfahren}
 \begin{description}[Zusammenbau]
  \item[Original] direkt auf die \cod{SD-Card}
  \begin{itemize}
   \item Einstellung von \cod{ssh} auf dem \target 
  \end{itemize}
  \item[sd-card.gz] direkt auf die \cod{SD-Card}
  \item[Zusammenbau] aus \cod{boot.tar.gz} und \cod{target-root.tar.gz}
 \begin{itemize}
  \item als File auf dem Host
  \item direkt auf der SD-Card
 \end{itemize}
 \end{description}
\end{frame}

\subsection{Original, sd-card.gz}
\begin{frame}{Original, \cod{sd-card.gz}}{Der Befehl \cod{dd}}
 \begin{itemize}
  \item \cod{dd if={\em image} of={\em device}}
  \begin{itemize}
   \item {\em image} von \url{raspberry.org} oder \url{sd-card.gz}
   \item typisches {\em device} \cod{/dev/mmcblk{\bf i}} $i=0,1 ..$
  \end{itemize}
 \end{itemize}
\end{frame}

\subsection{Zusammenbau}
\begin{frame}[fragile]{Zusammenbau Host}{alles ist ein File}
%{\cod{dd} \cod{fdisk}, \cod{losetup}, \cod{mount}, \cod{tar}}
\begin{enumerate}
 \item erzeuge 'leeren' File \cod{\em sd-card} 1GiB:
 \begin{itemize}
  \item {\footnotesize\cod{dd if=/dev/zero of={\em sd-scard} bs={\em bs} count={\em count}}}
  \begin{itemize}
   \item mit $bs\times count = 1GiB$  
  \end{itemize}
 \end{itemize}
 \item formatiere \cod{\em sd-card} wie wenn es eine 'richtige' SD-Card w�re:
 \begin{itemize}
  \item \cod{fdisk sd-card}
 \end{itemize}
{\tiny
\begin{verbatim} 
sd-card1          2048  133119  131072   64M  c W95 FAT32 (LBA)
sd-card2        133120 2097151 1964032  959M 83 Linux
\end{verbatim}
}
 \item fasse \cod{sd-card\{1\textbar 2\}} als {\em device} auf:
 \begin{itemize}
  \item \cod{losetup -o {\em offset} /dev/loop{\bf i} {\em sd-card}}
  \begin{itemize}
   \item mit $offset$ in Bytes
   \item $i=0,1$
  \end{itemize}
 \end{itemize}
 \item erzeuge {\em Filesysteme} auf \cod{/dev/loop{\bf i}} 
 \begin{itemize}
  \item \cod{mkfs.{\em fs} /dev/loop{\bf i}}
  \begin{itemize}
   \item mit {\em fs=\cod{vfat}} f�r \cod{sd-card1} und {\em fs=\cod{ext4}} f�r
        \cod{sd-card2} 
  \end{itemize}
 \end{itemize}
\end{enumerate}
\hint[15]{42,52}{Offset in {\em sector}}
\end{frame}

\begin{frame}{Zusammenbau Host}{Fortsetzung}
\begin{enumerate}
  \setcounter{enumi}{4}
 \item montiere {\em devices} \cod{/dev/loop{\bf i}} 
 \begin{itemize}
  \item \cod{mount /dev/loop{\bf i} {\em mount-point}}
 \end{itemize}
 \item kopiere Files 
 \begin{itemize}
  \item \cod{tar -xzf {\em part}.tar.gz {\em mount-point}}
  \begin{itemize}
   \item mit \cod{{\em part}=boot\textbar target-root}
  \end{itemize}
 \end{itemize}
 \item autr�umen
 \begin{itemize}
  \item \cod{umount /dev/loop{\bf i}}
  \item \cod{losetup -D}
 \end{itemize}
 \item kopieren auf SD-Card
 \begin{itemize}
  \item \cod{dd if=sd-card of=/dev/mmcblk{\bf i}}
 \end{itemize}
\end{enumerate}
\end{frame}

\begin{frame}{Zusammenbau SD-Card}
 \begin{enumerate}
  \item formatiere SD-Card 2 Partitionen:
  \begin{description}
   \item[Partition 1] Kleine $64MiB$ f�r Boot/Kernel
   \item[Partition 2] Grosse $>1GiB$ f�r \unix
  \end{description}
  \item erzeuge Filesystem 
  \begin{description}
   \item[Partition 1] vfat
   \item[Partition 2] ext4
  \end{description}
  \item Kopiere 
  \begin{description}
   \item[Partition 1] \cod{boot.tar.bz2}
   \item[Partition 2] \cod{target-root.tar.bz2}
  \end{description}
 \end{enumerate}
\end{frame}
