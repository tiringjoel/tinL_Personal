\section{\mdev}

\begin{frame}{Die Entwicklungsumgebung}

Die die {\color{green} Verzeichnisse} sind versionert

\dirtree{%
 .1 somewhere-on-your-\host.
 .2 {\color{green} src} \DTcomment{your source}.
 .2 {\color{green}config}.
 .3 Makefile \DTcomment{linked from work}.
 .2 tc \DTcomment{gcc toolchain}.
 .2 target-root \DTcomment{the devel part}.
 .2 {\color{green}tools} \DTcomment{for downloading}.
 .2 work \DTcomment{can be deleted without pain}.
 .3 Makefile $\to$ ../config/Makefile.
}
\end{frame}

\subsection{Das {\em devel} root Filesystem}

\begin{frame}{Das {\em devel} root Filesystem}
\label{devel-target-fs}
\begin{columns}
\begin{column}{0.375\textwidth}
{\scriptsize
\begin{verbatim}
/somewhere-on-your-workstation
|-- bin
|-- dev
|-- etc
|   |-- init.d
|-- lib
|-- libexec
|-- proc
|-- sbin
|-- sys
|-- usr
|   |-- bin
|   |-- include
|   |-- lib
|   |-- libexec
|   |-- sbin
`-- var
\end{verbatim}
}

\end{column}
\begin{column}{0.75\textwidth}
\begin{block}{Was braucht es f�r  {\em devel}}
\begin{itemize}
 \item include Files \cod{*.h}
 \item Statische Bibliotheken \cod{*.a} Files
 \item Dynamische Libraries \cod{*.so} Files
 \item ...
\end{itemize}
\end{block}
\begin{block}{Was braucht es f�r {\em devel} sicher {\bf nicht}}
 \begin{itemize}
  \item Executables \cod{bin/*},\cod{sbin/*}
  \item Initialisierungen \cod{/etc/init.d/*}
  \item ...
 \end{itemize}
\end{block}
\end{column}
\end{columns}
\hyperlink{runtime-target-fs}{Vergleiche mit {\em runtime} root Filesystem}
\end{frame}

\subsection{Herstellung}

\begin{frame}{Herstellung}
 \begin{block}{Gegeben}
  \vspace{-2mm}
  \begin{itemize}
   \item die Toolchain \cod{beaglebone-black-toolchain-64bit.tar.bz2}
    in \url{http://sourceforge.net/projects/fhnw-tinl/files}
   \item ein vollst�ndiges \cod{target-root} als \cod{tar.gz} File 
   \item ein paar Sourcefiles 
   \item ein Makefile
  \end{itemize}
 \end{block}
 \vspace{-2mm}
 \begin{block}{Gesucht}
  \vspace{-2mm}
  \begin{itemize}
   \item die Entwicklungsumgebung \mdev
  \end{itemize}
 \end{block}
 \vspace{-2mm}
  \begin{block}{Test/Installation}
  \vspace{-2mm}
  \begin{itemize}
   \item \cod{tar -xjf md-bbb-devel-version.tar.gz -Cpath-to-bbb-devel}
   \item Mache work 
   \item \cod{make} 
  \end{itemize}
 \end{block}
\end{frame}

