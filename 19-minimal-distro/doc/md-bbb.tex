\section{\md}
\begin{frame}[fragile]{Die Partionen}
\begin{verbatim}
NAME        FSTYPE  SIZE
mmcblk0            29.9G
|-mmcblk0p1 vfat     16M
`-mmcblk0p2 ext4    127M
\end{verbatim}
\begin{itemize}
 \item Ausdruck gemacht mit \\
 \cod{lsblk  -i -o'NAME,FSTYPE,SIZE'  /dev/mmcblk0}
\end{itemize}
\end{frame}

\subsection{Partition 1}

\begin{frame}[fragile]{Partition 1}{\cod{vfat}}
\label{partition1}
\begin{tabular}{rll}
   SIZE    \\
\hline 
    58K &  am335x-boneblack.dtb & {\em 5-kernel}\\
   7.9M & zImage				& {\em 5-kernel}\\
   102K & MLO					& n�chste Folie\\
   350K & u-boot.img			& \"{}\\
    788 & uEnv.txt				& \"{}\\
\end{tabular}
\end{frame}

\begin{frame}{u-boot}
 \begin{itemize}
  \item Source 
  \begin{itemize}
   \item \url{https://github.com/beagleboard/u-boot}
  \end{itemize}
  \item Image  
  \begin{itemize}
   \item \url{http://s3.armhf.com/dist/bone/bone-uboot.tar.xz}
   \item mit den Files:
   \begin{itemize}
    \item \cod{MLO} Siehe Abschnitt {\em 18-boot}
	\item \cod{u-boot.img}
	\item \cod{uEnv.txt}
   \end{itemize}	   
  \end{itemize}
 \end{itemize}
\end{frame}

\subsection{Partition 2}
\begin{frame}[fragile]{Partition 2}{Das {\em runtime} root Filesystem}
\label{runtime-target-fs}
\begin{columns}
\begin{column}{0.375\textwidth}
{\scriptsize
\begin{verbatim}
/somewhere-on-your-workstation
|-- bin
|-- dev
|-- etc
|   |-- init.d
|-- lib
|-- libexec
|-- proc
|-- sbin
|-- sys
|-- usr
|   |-- bin
|   |-- include
|   |-- lib
|   |-- libexec
|   |-- sbin
`-- var
\end{verbatim}
}

\end{column}
\begin{column}{0.75\textwidth}
\begin{block}{Was braucht es f�r die {\em runtime}}
\begin{itemize}
 \item Executables \cod{bin/*},\cod{sbin/*}
 \item Dynamische Libraries \cod{*.so} Files
 \item Initialisierungen \cod{/etc/init.d/*}
 \item ...
\end{itemize}
\end{block}
\begin{block}{Was braucht es f�r die {\em runtime} sicher {\bf nicht}}
 \begin{itemize}
  \item include Files \cod{*.h}
  \item Statische Bibliotheken \cod{*.a} Files
  \item ...
 \end{itemize}
\end{block}
\end{column}
\end{columns}
\hyperlink{devel-target-fs}{Vergleiche mit {\em devel} root Filesystem}
\end{frame}

\subsection{Herstellung}
\begin{frame}{Herstellung}
 \begin{block}{Gegeben}
  \begin{itemize}
   \item die Files \hyperlink{partition1}{f�r Partition 1}
   \item ein vollst�ndiges \cod{target-root} als \cod{tar.gz} File 
  \end{itemize}
 \end{block}
 \begin{block}{Gesucht}
  \begin{itemize}
   \item der Image File \md\cod{.gz}
  \end{itemize}
 \end{block}
 \begin{block}{Test/Installation}
  \begin{itemize}
   \item SD-Card: \cod{dd if=\md of=/dev/mmcblk{\em N}} $N=0,1 ..$
   \item SD-Card in den \targetS schieben reset start siehe \cod{u-boot.cmd}
  \end{itemize}
 \end{block}
\end{frame}


