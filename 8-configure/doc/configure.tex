%-------------------------
%makefile
%(c) H.Buchmann FHNW 2014
%export TEXINPUTS=${HOME}/fhnw/edu/:${HOME}/fhnw/edu/tinL/config/latex:${HOME}/fhnw/edu/config//:
%-------------------------
\documentclass{beamer}
\usepackage{latex/beamer}
%---------------------
%local defines
%(c) H.Buchmann FHNW 2009
%$Id$
%---------------------
\newcommand{\target} {\beaglebone\xspace}
\newcommand{\targetS}{{\bf BBG}\xspace}
\newcommand{\host}   {{\em Host}\xspace}
\newcommand{\targetroot} {{\bf target-root}\xspace}
\newcommand{\kernel} {{\bf kernel}\xspace}
\renewcommand{\c}{{\bf C}\xspace}
\newcommand{\cpp}{{\bf C++}\xspace}
\newcommand{\posix}{{\bf POSIX}\xspace}

\input{/home/buchmann/latex/dirtree/dirtree.tex}

\usepackage[absolute]{textpos}
\setlength{\TPHorizModule}{1mm}
\setlength{\TPVertModule}{1mm}

\begin{document}


\title{Configure}

\frame{\titlepage}

\begin{frame}{Um was geht es ?}{Herstellung von Software aus den Quellen}
 \begin{itemize}
  \item drei Schritte:
  \begin{description}
   \item[configure] f�r das \target
   \item[make] die {\em binaries} aus den {\em Quellen} im \host
   \item[install] auf dem \target
  \end{description}
  \item die Schwierigkeiten:
  \begin{itemize}
   \item \host {\&} \target sind verschieden
  \end{itemize}
 \end{itemize}
\end{frame}

\section{Die Quellen}
\begin{frame}{Die Quellen}{\C/{\tiny \CPP Code}}
 \begin{itemize}
  \item meistens als
  \begin{itemize}
   \item \C Code
   \item {\em tar.gz} File:
   \begin{itemize}
    \item \cod{{\bf name}-{\em version}.tar.gz}
   \end{itemize}
  \end{itemize}
  \item es gibt aber noch andere M�glichkeiten
 \end{itemize}
\end{frame}

\subsection{Beispiel: rsync-3.1.1}
\begin{frame}{Beispiel}{\cod{rsync-3.1.1}}
\begin{block}{Wichtige Files}
 \begin{itemize}
  \item \cod{README}
  \item \cod{INSTALL}
  \item \cod{*.c} die Sourcen
  \item f�r die Herstellung:
  \begin{itemize}
   \item Scripts
   \item Makefile(s)
  \end{itemize}  
  \item \cod{configure}: unser Thema
 \end{itemize}
\end{block}
\end{frame}

\begin{frame}{Die Sourcen}{Beispiel \cod{main.c}}
 \begin{block}{der Pr�prozessor \cod{cpp}}
  \begin{itemize}
   \item \cod{\#define}
   \item \cod{\#ifdef ...} \cod{\#endif}
   \item \cod{\#include}
  \end{itemize}
 \end{block}
 \begin{block}{l�sst Code zur Compilation}
 \begin{itemize}
  \item zu 
  \item[]oder 
  \item nicht zu
 \end{itemize}
 \end{block}
\end{frame}

\section{Die Aufgabe}
\begin{frame}{Von den Quellen zum Programm}
 \begin{block}{Gegeben}
  \begin{itemize}
   \item die Quellen
   \item \host
   \item {\em target} (\target)
  \end{itemize}
 \end{block}
 \vspace{-5mm}
 \begin{block}{Gesucht}
  \begin{itemize}
   \item das in den Quellen beschriebene Programm lauff�hig auf dem 
   {\em Target} (\target)
   \item gemacht auf dem \host
  \end{itemize}
 \end{block}
 \vspace{-2mm}
 \remark{Der einfachere Fall:
  \begin{itemize}
   \item \host = {\em Target}
  \end{itemize}
 } 
\end{frame}

\begin{frame}{Das Problem}{die Vielfalt}
 \begin{itemize}
  \item es gibt viele verschiedene \host's
  \item es gibt viele verschiedene {\em target}'s
  \begin{itemize}
   \item verschiedene Architekturen
   \begin{itemize}
    \item {\em ARM}
    \item {\em Intel}
   \end{itemize}
  \end{itemize}
 \end{itemize}
\end{frame}

\section{Das Skript {\tt configure}}

\begin{frame}{Das Skript {\tt configure}}
\begin{block}{erzeugt auf dem \host}
 \begin{itemize}
  \item einen \cod{Makefile}
  \item[] f�r die 
  \item {\em binaries} auf dem {\em Target}
 \end{itemize}
\end{block}
 \remark{Der einfachere Fall:
  \begin{itemize}
   \item \host = {\em Target}
  \end{itemize}
 } 
\end{frame}

\begin{frame}{\host - {\em Target}}{die Verzeichnisse}
 \begin{tabular}{p{5.5cm}p{5cm}}
  \host	& {\em Target}\\
  \hline
  \dirtree{%
   .1 somewhere.
   .2 tc \DTcomment{toolchain}.
   .2 sources.
   .2 build.
   .2 target-root \DTcomment{libraries include}.
   .2 {\em install}.
  }&
   \dirtree{%
    .1 / \DTcomment{root}.
    .2 {\em install} \DTcomment{somewhere}.
   }
 \end{tabular}
 \begin{block}{Verbindung}
  \begin{itemize}
   \item \host:\cod{install} -- {\em Target}:\cod{install}
  \end{itemize}
  per \cod{sshfs}, \cod{ftp}, manueller SD-Card transfer
 \end{block}
\end{frame}

\begin{frame}{Die option \cod{--prefix}}{von \cod{configure}}
 \begin{block}{-{}-prefix}
 \begin{itemize}
  \item gibt an wo die {\em binaries} installiert werden sollen
 \end{itemize}
 \end{block}
 \begin{block}{Ziel}
  \begin{itemize}
   \item {\Large ohne} \cod{root} Privilegien auf dem \host
  \end{itemize}
 \end{block}
\end{frame}

\begin{frame}{Eine m�gliche Verzeichnisstruktur}{auf dem Host}
 \dirtree{%
  .1 tinL.
  .2 resources \DTcomment{readonly}.
  .3 {\em sources}.
  .2 8-configure.
  .3 tc \DTcomment{toolchain}.
  .3 src \DTcomment{scripts}.
  .3 target-root \DTcomment{link}.
  .3 work  \DTcomment{where to install, connected with \target}.
  .3 build-{\em source} \DTcomment{where to build}.
 }
\end{frame}

\begin{frame}{Die Scripts}{ in \cod{src}}
 \begin{itemize}
  \item \cod{{\em sources}/configure} in 
   \begin{itemize}
    \item\cod{build-{\em source}}
    \end{itemize}
    aufrufen
  \item \cod{make}, \cod{make install} in 
  \begin{itemize}
   \item\cod{build-{\em source}}
  \end{itemize}
  aufrufen
  \item resultat in
  \begin{itemize}
   \item \cod{work}
  \end{itemize}
 \end{itemize}
\end{frame}

\section{Aufgaben}
\begin{frame}{Aufgaben}{auf dem \host}
\begin{itemize}
 \item Die {\em sourcen}:
 \begin{itemize}
  \item \url{http://rsync.samba.org}
  \item \url{http://www.lighttpd.net}
  \item \url{http://sox.sourceforge.net}
  \item \url{http://www.openssh.com}
  \remark{die richtigen {\em sourcen}}
 \end{itemize}
 \item auf dem \host 
  \begin{itemize}
   \item unter \cod{work}
  \end{itemize}
\end{itemize}
\begin{remarks}
 \item {\Large Ohne} \cod{root} Privilegien
 \item Testen speziell \cod{lighttpd} und \cod{sox}
\end{remarks}
\end{frame}

\begin{frame}{Aufgaben}{auf dem \target}
\begin{itemize}
 \item Die {\em sourcen}:
 \begin{itemize}
  \item \url{http://rsync.samba.org/}
  \item \url{http://www.lighttpd.net/}
  \item \url{http://sox.sourceforge.net}
  \item \url{http://www.openssh.com}
  \remark{die richtigen {\em sourcen}}
 \end{itemize}
 \item auf dem \target 
  \begin{itemize}
   \item unter \cod{work}
  \end{itemize}
\end{itemize}
\begin{remarks}
 \item {\Large Ohne} \cod{root} Privilegien
 \item Testen speziell \cod{lighttpd} und \cod{sox}
\end{remarks}
\end{frame}

\begin{frame}{Auf was ist zu achten}
 \begin{itemize}
  \item alle erzeugten Daten in \cod{build-{\em source}}
  \item Es kann sein, dass im Kernel Treiber fehlen
  \begin{itemize}
   \item Kernel neu anpassen
  \end{itemize}
 \end{itemize}
\end{frame}

\end{document}
