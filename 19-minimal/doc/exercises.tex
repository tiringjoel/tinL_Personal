\section{Aufgaben}

\begin{frame}{Verzeichnisstruktur}
\dirtree{%
.1 19-minimal.
.2 config .
.3 Makefile .
.2 doc \DTcomment{die Folien}.
.2 src \DTcomment{die source files}.
.2 tc \DTcomment{toolchain normalerweise link}.
.2 work \DTcomment{hier sind Sie}.
.3 Makefile $\to$ ../config/Makefile.
}
\end{frame}

\begin{frame}{Die Files}{\url{drive.switch.ch/index.php/s/SR9s26Wppx1Zvzq}}
 \begin{itemize}
  \item \cod{target-root-2017.12.06.tar.gz} minimales \cod{RootFS} 
  \item \cod{gcc-7.2.0-arm-64bit-2017-12-06.tar.gz} volle Toolchain
 \end{itemize}
\end{frame}

\begin{frame}{Aufgaben}
 \begin{itemize}
  \item Erzeugen Sie ein:
  \begin{itemize}
   \item {\bf s-nano}
   \item {\bf c-nano}
   \item {\bf mini-static}
   \item {\bf mini-dynamic}
  \end{itemize}
  auf dem \targetS\ \cod{RootFS}
  \item �ndern Sie den \cod{Makefile} so ab, dass er {\bf s-nano}, {\bf c-nano}, 
  {\bf mini-static} und {\bf mini-dynamic} auf den \host erzeugt
 \end{itemize}
\end{frame}

