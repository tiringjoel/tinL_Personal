%-------------------------
%makefile
%(c) H.Buchmann FHNW 2014
%export TEXINPUTS=${HOME}/fhnw/edu/:${HOME}/fhnw/edu/tinL/config/latex:${HOME}/fhnw/edu/config//:
%-------------------------
\documentclass{beamer}
\usepackage{latex/beamer}
%---------------------
%local defines
%(c) H.Buchmann FHNW 2009
%$Id$
%---------------------
\def \target {\raspberry\xspace}
\def \host {{\em Host \xspace}}

\input{/home/buchmann/latex/dirtree/dirtree.tex}

\usepackage[absolute]{textpos}
\setlength{\TPHorizModule}{1mm}
\setlength{\TPVertModule}{1mm}

\begin{document}

\newcommand{\uboot}{{U-Boot \xspace}}
\newcommand{\tar}{\cod{target-root-{\em VERSION}.tar.gz}}
\title[Assembly]{Zusammenbau\\Assembly}

\frame{\titlepage}

\begin{frame}{Um was geht es ?}{Ein erstes vollst�ndiges System}
 \begin{itemize}
  \item Bootloader \uboot
  \item \kernel
  \item \unix 
 \end{itemize}
\end{frame}

\begin{frame}{Die Schichten}
\begin{center}
 \includegraphics[width=0.75\textwidth]{../../5-kernel/doc/layers.pdf}
\end{center}
\end{frame}

\begin{frame}{Das Ziel}{f�r \targetS}
 \begin{block}{Nach dem Reset:}
 \begin{enumerate}
  \item \uboot startet \kernel
  \item \kernel startet \unix
  \item \unix 
   \begin{itemize}
    \item konfiguriert {\em ethernet �ber USB}
    \item startet \cod{ssh} Server 
    \item verbindet sich per Wi-Fi
   \end{itemize}
 \end{enumerate}
 \end{block}
\end{frame}

\begin{frame}{Was wir schon haben}
 \begin{description}[root Filesystem:]
  \item[Toolchain:] download
  \item[\uboot:] selber gemacht (siehe 4-uboot) 
  \item [\kernel:] selber gemacht
  \item[root Filesystem:] download
  \begin{itemize}
   \item libc/\unix
  \end{itemize} 
 \end{description}
\end{frame}

\begin{frame}{Die Partitionen und Filesysteme}
 \begin{description}
  \item[p1] bootfs:vfat $\approx 20MiB$
   \begin{itemize}
    \item \uboot
    \begin{itemize}
     \item \cod{MLO}  
     \item \cod{u-boot.img}
%     \item \cod{uEnv.txt} Konfiguration
    \end{itemize}
   \end{itemize}
  \item[p2] rootfs:ext4 $\approx 200MiB$
  \begin{itemize}
   \item \cod{etc/init.d/rcS} init-script
    \item \kernel
    \begin{itemize}
      \item \cod{/boot/zImage}
	  \item \cod{boot/am335x-boneblack-wireless.dtb}
    \end{itemize}
  \end{itemize}
 \end{description}
\end{frame}

\section{Erster Zusammenbau}
\section{U-Boot}
\begin{frame}{\url{www.denx.de/wiki/U-Boot/WebHome}}{ein typischer Bootloader f�r eingebettete Systeme}
 \begin{itemize}
  \item Kommandozeilen
  \item Verbindung zum \host via RS232/USB
  \begin{itemize}
   \item \host: \cod{minicom -D /dev/ttyUSB{\em N}}, $N=0,1..$
   \item 115200 Baud \cod{8N1}
   \item {\Large no} Handshaking
  \end{itemize}
  \item Kopiert Daten von
  \begin{itemize}
   \item SD-Karten
   \item Netz
  \end{itemize}
  in das Memory vom \targetS
 \end{itemize}
\end{frame}

\subsection{Bedienung}
\begin{frame}{Ein paar typische Befehle}
 \begin{itemize}
  \item \cod{help}
  \item \cod{printenv} Zeigt die Umgebung
  \item \cod{md addr} Memory display
  \item \cod{fatls mmc p}  vfat sd-card partition p
  \item \cod{fatload mmc p memAddr file}
  \item \cod{tftpboot [loadAddress] [[hostIPaddr:]bootfilename]} 
  \item \cod{bootz kernelAddr - fdt}
 \end{itemize}
\end{frame}


\begin{frame}{U-Boot Bedienung}{Siehe \cod{tools/u-boot-copy-paste.cmd}}
 \begin{itemize}
  \item copy paste
  \remark{kann sich �ndern}
 \end{itemize}
\end{frame}


\subsection{Herstellung}

\begin{frame}{Verzeichnisstruktur}
\dirtree{%
.1 4-uboot \DTcomment{u-boot section}.
.2 tc \DTcomment{link to toolchain}.
.2 tools \DTcomment{scripts}.
.2 config \DTcomment{configuration}.
.2 build \DTcomment{home of binaries}.
.3 MLO \DTcomment{second stage bootloader}.
.3 u-boot.img \DTcomment{the U-Boot}.
}
\end{frame}

\begin{frame}{Herstellung}
 \begin{itemize}
  \item Repository \url[http]{git.denx.de/u-boot.git}
  \item Script \cod{tools/u-boot.sh {\em what}}
  \begin{itemize}
   \item {\em what}:\cod{am335x\_boneblack\_defconfig}, \cod{all}, \cod{help}, \cod{distclean}
  \end{itemize}
 \end{itemize}
\end{frame}

\subsection{Installation}

\begin{frame}[fragile]{SD-Karte:Partionierung}
 \begin{itemize}
  \item Partitionen: \cod{fdisk}
{\footnotesize
\begin{verbatim}  
Device         Boot Start    End Sectors  Size Id Type
/dev/mmcblk0p1 *     2048  36863   34816   17M  b W95 FAT32
/dev/mmcblk0p2      36864 299007  262144  128M 83 Linux
\end{verbatim}
}
 \item Filesysteme
 \begin{itemize}
  \item Partition 1: vfat: \cod{mkfs.vfat} f�r U-Boot
  \item Partition 1: ext4: \cod{mkfs.ext4}
 \end{itemize}
 \end{itemize}
\end{frame}

\begin{frame}[fragile]{Partition 1}
\begin{block}{Die U-Boot Files}
{\footnotesize
\begin{verbatim} 
  76048 Oct 17 11:41 MLO
 376224 Oct 17 11:41 u-boot.img
\end{verbatim}
}
\end{block}
\begin{block}{F�r \linux}
{\footnotesize
\begin{verbatim}
9941512 Oct 17 17:52 zImage
  56658 Oct 17 12:21 am335x-boneblack-wireless.dtb
\end{verbatim}
}
\end{block}
\end{frame}



\subsection{Kernel}

\begin{frame}{Konfiguration}{USB-Gadget Support}
\begin{center}
\includegraphics[height=0.875\textheight]{usb-gadget-support.png}
\end{center}
\end{frame}


\subsection{RootFS}
\begin{frame}{Init Script}{\cod{target-root-{\em version}.tar.gz}}
 \begin{itemize}
  \item \cod{/etc/init.d/rcS} das {\em Init-Script}
  \item \cod{ifconfg} f�r Internet
  \item \cod{sshd} Server f�r Verbindung
 \end{itemize}
\end{frame}


\begin{frame}{Workflow}{Notationen}
 \begin{description}[\cod{target-root-{\em V}.tar.gz}]
  \item[\cod{\em sd-card}] die Partition vom rootfs auf der SD Karte
  \item[\cod{target-root-{\em V}.tar.gz}] das heruntergeladene rootfs
  \item[\cod{\em target-root}] das rootfs von \targetS auf dem \host
 \end{description}
\end{frame}

\begin{frame}{Workflow}{schrittweise Verbesserung}
 \begin{enumerate}
  \item Initialer Download \cod{target-root-{\em V}.tar.gz}
  \item \cod{target-root}
  \begin{itemize}
   \item \cod{tar -xf target-root-{\em V}.tar.gz -C {\em target-root}}
  \end{itemize} 
  \item Transfer auf \cod{\em sd-card}
  \begin{itemize}
   \item \cod{rsync -av {\em target-root}/ {\em sd-card}/}
   \item \cod{sync} 
  \end{itemize} 
  \item Test/Konfiguration auf dem \targetS
  \item Update auf dem \host
  \begin{itemize}
   \item \cod{rsync -av {\em sd-card}/ {\em target-root}/}
  \end{itemize} 
  \item $\to$ 4
 \end{enumerate}
\end{frame}

\begin{frame}{Die Files}
 \begin{block}{Partition 1: vfat}
  \begin{itemize}
   \item \cod{MLO}
   \item \cod{u-boot.img}
   \item \cod{zImage}
   \item \cod{am335x-boneblack-wireless.dtb}
  \end{itemize}
 \end{block}
% \vspace{-4mm}
 \begin{block}{Partition 1: ext4}
  \begin{itemize}
   \item rootfs auf dem \host
   \item \cod{rsync -av target-root/ {\em sd-card/}}
   \item \cod{sync}
  \end{itemize}
 \end{block}
 
\end{frame}




\section{Wi-Fi}
\begin{frame}{Ziele}{Wi-Fi}
 \begin{itemize}
  \item Konfiguration: \kernel
%  \item Neues root-fs: download
  \item Konfiguration: wi-fi Zugang
  \item schrittweises Vorgehen
 \end{itemize}
\end{frame}

\subsection{Kernel}

\begin{frame}{Konfiguration}{}
\begin{center}
\includegraphics[height=0.75\textheight]{wl18xx.png}
\end{center}
\vspace{-2mm}
\begin{itemize}
 \item Test: \cod{dmesg | grep wl}
\end{itemize}
\end{frame}

\begin{frame}{Abh�ngigkeiten}
\begin{center}
\includegraphics[height=0.75\textheight]{wl18xx-dependencies.png}
\end{center}
\end{frame}

\begin{frame}{Firmware}
\begin{center}
\includegraphics[height=0.75\textheight]{firmware.png}
\end{center}
\end{frame}

\begin{frame}{Test}{wlan0}
 \begin{itemize}
  \item \cod{dmesg | grep wl}
  \item \cod{ip link set wlan0 up}
  \item \cod{iw wlan0 scan}
 \end{itemize}
\end{frame}

\subsection{Connect}

\begin{frame}[fragile]{WPA}{\cod{wpa\_supplicant}, \cod{wpa}}
 \begin{itemize}
  \item Konfiguration: 
  \begin{itemize}
   \item Siehe {\em 3-network}
  \end{itemize}
  \item Process:
  \begin{itemize}
   \item \cod{wpa\_supplicant -D wext -i wlan0 -c {\em path\_to\_config}}
  \end{itemize}
  \item Bedienung (funktioniert nocch nicht)
  \begin{itemize}
   \item \cod{wpa\_cli -s {\em  wpa\_client\_socket\_file\_path}}
  \end{itemize}
 \end{itemize}
\end{frame}

\begin{frame}[fragile]{DHCP}
 \begin{block}{manuell}
  \begin{itemize}
   \item \cod{udhcpc -v -i wlan0}
   \item \cod{ifconfig wlan0 {\em ip}}
   \begin{itemize}
    \item \cod{\em ip} abgelesen von \cod{udhcpc -v -i wlan0}
   \end{itemize}
  \end{itemize}
 \end{block}
 \begin{block}{automatisch/callback}
\vspace{-3mm}
{\tiny
\begin{verbatim}
#!/bin/sh
#---------------------
#on-udhcpc.sh
#(c) H.Buchmann FHNW 2018
#---------------------
echo "-------------- on-udhcpc.sh ${1}" 
case ${1} in
 defconfig)
  echo defconfig------- ${interface} ${ip};;
  bound)
   ifconfig ${interface} ${ip};;
#set route here
esac
\end{verbatim}
}
\end{block}
\end{frame}

\begin{frame}[fragile]{route/dns}
\begin{itemize}
 \item route
 \begin{itemize}
  \item \cod{route add default gw {\em gw-ip} wlan0}
 \end{itemize}
 \item DNS
 \begin{itemize}
   \item \cod{/etc/resolv.conf:}
{\tiny
\begin{verbatim}
nameserver 147.86.4.21
#try nameserver 8.8.8.8
\end{verbatim}
 }
 \end{itemize}
 
\end{itemize}
\end{frame}



\section{Aufgaben}
\begin{frame}{Aufgabe}
\begin{block}{\kernel}
 \begin{itemize}
  \item Ethernet �ber USB
  \item Wi-Fi
 \end{itemize}
\end{block}
\begin{block}{\unix}
\begin{itemize}
 \item \cod{usb0}
 \item \cod{sshd}
 \item \cod{wlan0}
 \item WPA
 \item DHCP
 \item DNS
\end{itemize}
\end{block}
\end{frame}

\begin{frame}{Aufgabe}{Init \cod{//etc/init.d/rcS}}
\begin{itemize}
 \item erg�nzen
\end{itemize}
\end{frame}

\subsection{sshd}
\begin{frame}{Ein paar tools}
 \begin{itemize}
  \item \cod{touch} - change file timestamps
  \item \cod{chown} - change file owner and group
 \end{itemize}
\end{frame}

\begin{frame}{sshd}
\begin{itemize}
 \item sshd re-exec requires execution with an absolute path
 \item Privilege separation user sshd does not exist
 \item create group \cod{root}
 \begin{itemize}
  \item \cod{addgroup -g 0 -S root} 
 \end{itemize}
 \item create user \cod{root}
  \begin{itemize}
   \item \cod{adduser -h /home/root/ -s /bin/sh -G root -S -u 0 root}
  \end{itemize}
\item create group/user \cod{sshd}
 \begin{itemize}
  \item \cod{addgroup sshd}
  \item \cod{adduser -D -H -G sshd sshd}
  \end{itemize}
\item create key
 \begin{itemize}
  \item \cod{ssh-keygen -t rsa -f /etc/ssh\_host\_rsa\_key}
 \end{itemize}
 \item File \cod{/var/empty} geh�rt root
 \item File \cod{/etc/sshd\_config}
 \begin{itemize}
  \item \cod{PermitRootLogin yes}
 \end{itemize}
\end{itemize}
\end{frame}


\end{document}
