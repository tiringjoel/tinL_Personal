\section{Userspace}
\subsection{printf/fprintf}
\begin{frame}{printf/fprintf}
 \begin{itemize}
  \item Unterschied \cod{stdout} vs. \cod{stderr}
   \begin{itemize} 
    \item \cod{printf} (\cod{stdout}) 
    \item \cod{fprintf(stderr,....)}
   \end{itemize}
  \item Einfaches Beispiel \cod{hello-world.c}
 \end{itemize}
\end{frame}

\subsection{strace}
\begin{frame}{strace}{trace system calls and signals}
 \begin{itemize}
  \item \cod{strace ./hello-world}
 \end{itemize}
 \remark{Einen Haufen Optionen}
\end{frame}

\subsection{Debugger}
\begin{frame}{Debugger}{\url{sourceware.org/gdb/current/onlinedocs/gdb/}}
 \begin{itemize}
  \item compile \cod{-g}
  \item start: \cod{gdb -d {\em path-to-source}}
  \item \cod{break} point
  \item \cod{run},\cod{step}, \cod{cont}
 \end{itemize}
\end{frame}

\begin{frame}{die graphische Oberfl�che}{z.B. nemiver}
 \begin{itemize}
  \item \cod{break} point
  \item \cod{run} \cod{cont}
  \item etc.
 \end{itemize}
 \remark{es gibt noch andere graphische Oberfl�chen}
\end{frame}

\begin{frame}{TODO's}
 \begin{itemize}
  \item Optimierungen
  \begin{itemize}
   \item \cod{Makefile} \cod{CFLAGS} \cod{-O0,-O1, -O2}
  \end{itemize}
 \end{itemize}
\end{frame}
