%-------------------------
%minimal-unix
%(c) H.Buchmann FHNW 2014
%export TEXINPUTS=.:${HOME}/fhnw/edu/:${HOME}/fhnw/edu/tinL/config/latex:${HOME}/fhnw/edu/config//:
%-------------------------
\documentclass{beamer}
\usepackage{latex/beamer}
%---------------------
%local defines
%(c) H.Buchmann FHNW 2009
%$Id$
%---------------------
\def \target {\raspberry\xspace}
\def \host {{\em Host \xspace}}

\input{/home/buchmann/latex/dirtree/dirtree.tex}

\usepackage[absolute]{textpos}
\setlength{\TPHorizModule}{1mm}
\setlength{\TPVertModule}{1mm}

\begin{document}

\title{Hardware/Devices}

\frame{\titlepage}

\section{Begriffe}
\begin{frame}{Devices}
 \begin{itemize}
  \item Driver f�r Devices
  \begin{itemize}
   \item Vereinfacht den Zugriff 
   \item alles ist ein File {\em Stream of Bits} 
   \item geh�ren zum {\em kernel}
   \item Verbindung {\em userspace} $\leftrightarrow$ {\em kernelspace}
  \end{itemize}

  \item Userspace
  \begin{itemize}
   \item komfortabler Zugriff via {\em Driver} auf die {\em Devices}
   \item alles ist ein File {\em Stream of Bits}
   \item \cod{mmap} Verschiedene Arten von Memory 
  \end{itemize}
 \end{itemize}
\end{frame}

\begin{frame}{Kernelspace}
 \begin{itemize}
  \item Kernelspace
  \begin{itemize}
   \item extrem privilegierter Zugriff
   \item {\em kernelmodules}
   \item Heimat der {\em Driver}
  \end{itemize}
  \item Devicetree
  \begin{itemize}
   \item Versuch einer Normierung
   \item wo ist was
  \end{itemize}
 \end{itemize}
\end{frame}

\begin{frame}{\linux}{Befehle: funktionieren nicht alle auf \targetS}
 \begin{itemize}
  \item \cod{dmesg} 
  \begin{itemize}
   \item Meldungen von {\em kernel}
  \end{itemize}
  \item \cod{ls*}
  \begin{itemize}
   \item \cod{lspci}, \cod{lsblk}, \cod{lscpu} ...
  \end{itemize}
  \item\cod{hwdb}
  \item ...
 \end{itemize}
\end{frame}

\begin{frame}{\linux}{Verzeichnisse}
  \begin{itemize}
  \item \cod{/proc}
  \begin{itemize}
   \item \cod{/proc/iomem}: wo ist die Hardware
   \item \cod{/proc/devices}: die {\em major} Nummern
   \item \cod{/proc/config.gz}: die aktuelle {\em kernel} Konfiguration
  \end{itemize}
  \item \cod{/dev} Devices Namen
  \item \cod{/sys} Devices im Detail
  \begin{itemize}
   \item ziemlich komplex
  \end{itemize}
  \end{itemize}
\end{frame}

\begin{frame}{Um was geht es ?}{Zugriff auf mehrere Arten}
 \begin{itemize}
  \item userspace
  \begin{itemize}
   \item per \cod{/sys/class/gpio}
   \item per \cod{mmap} mit eigenem Programm: 
  \end{itemize}
  \item kernelspace
  \begin{itemize}
   \item mit eigenem module
  \end{itemize}
 \end{itemize}
\end{frame}

\section{Direkt}

\begin{frame}{Die Aufgabe}{die  LEDs \cod{USR\{0,1,2,3\}} setzen/l�schen}
 \begin{block}{Die Informationen}
 \begin{itemize}
  \item Schema \& Info
   \begin{itemize}
    \item \cod{doc/beaglebone-black/BBB*.pdf}
   \end{itemize}
   
   \item Prozessor System On Chip
   \begin{itemize}
     \item \cod{AM335x Sitara Processors (Rev. L)}\\
	  \cod{doc/beaglebone-black/spruh73l.pdf}
   \end{itemize}
 \end{itemize}
 \end{block}
\end{frame}


\begin{frame}{Die Komponente}{GPIO1}
 \begin{block}{In \cod{AM335x Sitara Processor}}
 \begin{itemize}
  \item MemoryMap: 
   \begin{itemize}
     \item Abschnitt 2.1ARM Cortex-A8 Memory Map
   \end{itemize}
  \item Beschreibung
  \begin{itemize}
    \item Abschnitt 25
  \end{itemize}
 \end{itemize}
 \end{block}
\end{frame}

\begin{frame}{U-Boot}{die Befehle}
 \begin{itemize}
  \item \cod{mw} Memory Write
  \item \cod{md} Memory Display
  \item \cod{tools/u-boot.cmd}
 \end{itemize}
\end{frame}

\section{Userspace}
\subsection{sys/class/gpio}
\begin{frame}{\cod{sys/class/gpio}}
 \begin{itemize}
  \item \cod{kernel/Documentation/gpio.txt}
  \item Gute Pins:
  \begin{itemize}
   \item USR\{0,1,2,3\} auf \cod{GPIO1\_\{21,22,23,24\}}
   \item {\em kernel} anpassen: {\em kein} LED support
  \end{itemize}
 \end{itemize}
\end{frame}

\subsection{mmap}
\begin{frame}{\cod{mmap}}{Direkter Zugriff auf die Hardware}
 \begin{itemize}
  \item Beschreibung
   \begin{itemize}
    \item Siehe Abschnitt {\em Direkt}
   \end{itemize}
  \item Wo im Speicher
   \begin{itemize}
    \item \cod{/proc/iomem}
   \end{itemize}
 \item Der wichtige Aufruf
  \begin{itemize}
   \item \cod{mmap}
  \end{itemize}
 \end{itemize}
\end{frame}

\begin{frame}[fragile]{mmap}
\begin{center}
 \includegraphics[width=8cm]{mmap.pdf}
\end{center}
\begin{lstlisting}[language=C]
 mem=(unsigned char*)mmap(0, /* addr hint */
     length,
     PROT_READ|PROT_WRITE,
     MAP_SHARED, 
     memId,
     offset);
\end{lstlisting}
\end{frame}

\subsection{Aufaben}
\begin{frame}{Aufgaben}
 \begin{itemize}
  \item Der Code \cod{src/led-demo.c}
 \end{itemize}
\end{frame}

\section{Kernelspace}
\begin{frame}{Kernelspace}{Kernelmodules}
 \begin{itemize}
  \item 6-modules
 \end{itemize}
\end{frame}

\subsection{Aufgaben}
\begin{frame}{Aufgaben}
 \begin{itemize}
  \item Umgebung f�r Kernelmodule einrichten
  \item Das Module \cod{simple-hw.c} erg�nzen
  \item Zugriff vom {\em userspace}
 \end{itemize}
\end{frame}

\end{document}
