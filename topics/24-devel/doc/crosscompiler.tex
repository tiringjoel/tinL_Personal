\renewcommand{\src}[1]{\cod{{\bf #1}.src}}
\newcommand{\exe}[2]{\cod{{\bf #1}(#2)}}
\newcommand{\tc}[3]
% #1 file
% #2 target
% #3 exec
{
 \src{#1}\to\boxed{\exe{tc[#2]}{#3}}\to\exe{#1}{#2}
}

\begin{frame}{CrossToolchain}{Begriffe}
 \begin{itemize}
  \item der CrossCompiler geh�rt zur CrossToolchain
  \item neben dem CrossCompile gibt es noch andere Komponenten:
  \begin{itemize}
   \item Assembler
   \item Linker
   \item ...
  \end{itemize}
 \end{itemize}
\end{frame}

\begin{frame}[fragile]{Die ganze gcc Toolchain}

\begin{lstlisting}{language=bash}
ls path-to-tc-bin
\end{lstlisting}

{\scriptsize
\verbatiminput{gcc-tc.lst}
}
\end{frame}

\begin{frame}{CrossToolchain}{Notationen}
\begin{description}[Targetrechner]
 \item[Hostrechner] $H$
 \item[Targetrechner] $T$ 
  \begin{itemize}
   \item Beispiel \target
  \end{itemize}
 \item[Sourcefile] \src{file}
  \begin{itemize}
   \item Beispiel \cod{hello-world.c}
  \end{itemize}
 \item[Executable] \exe{file}{M} ausf�hrbar auf dem Rechner $M$, \\$M=H|T$
  \begin{itemize}
   \item Beispiel \cod{hello-world(T)} \\f�r $T=\target$
  \end{itemize}
\end{description}
\end{frame}

\begin{frame}{CrossToolchain}{Definition}
\[
\tc{file}{T}{H}
\]
\begin{description}[$\exe{compiler}{M}$]
 \item[$\src{file}$] der Source File
 \item[$\exe{tc[T]}{H}$] der Compiler/Toolchain ein {\em executable} f�r den Rechner $H$
 	erzeugt {\em executables} f�r den Rechner $T$
 \item[$\exe{file}{T}$] das {\em executable} f�r den Rechner $T$
\end{description}
\remark{Der Compiler ist ein (wichtiger) Bestandteil der ganzen {\em Toolchain}} 
\end{frame}

\begin{frame}[fragile]{Beispiel}{Programm auf dem \host}
 \[
  \tc{{hello\_world}}{H}{H}
 \]
 \begin{lstlisting}{language=bash}
  gcc -O2  -std=c99 \
  ../src/hello-world-c.c \
  -o hello-world-c
 \end{lstlisting}
\end{frame}

\begin{frame}[fragile]{Beispiel}{CrossToolchain}
 \[
  \tc{{hello\_world}}{T}{H}
 \]
 \begin{lstlisting}{language=bash}
 ../tc/bin/arm-linux-gnueabihf-gcc -O2 \
  --sysroot=../target-root -std=c99\
  ../src/hello-world-c.c \
  -o hello-world-c
 \end{lstlisting}
 \begin{itemize}
  \item \cod{arm-linux-gnueabihf-} entspricht $T$ 
  \begin{itemize}
   \item im gcc Terminologie {\em target}
  \end{itemize}
 \end{itemize}
\end{frame}

\begin{frame}{CrossToolchain}{Herstellung}
\[
 \tc{tc[T]}{H}{H}
\]
\begin{itemize}
 \item erzeugt auf dem Rechner $H$ eine {\em toolchain} die auf dem Rechner $H$ l�uft
 und {\em executables} f�r den Rechner $T$ erstellt
\end{itemize}

\end{frame}
