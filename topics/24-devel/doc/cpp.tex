\section{\cpp/\c}
\begin{frame}{Outline}{f�r C/C++ }
 \begin{block}{\host}
  {\color{green} dem \git unterstellt}
  \dirtree{%
   .1 somewhere\_on\_your\_host.
   .2 {\color{green}config}.
   .3 {\color{green}Makefile}.
   .2 {\color{green}src} \DTcomment{the own source files}.
   .2 work \DTcomment{seen by \target} .
   .3 {$\to$ {\color{green}../config/Makefile}} \DTcomment{link}.
   .2 target-root \DTcomment{for the toolchain}.
   .2 tc \DTcomment{toolchain}.
  }
 \end{block}
 \begin{block}{\target}
  \dirtree{%
   .1 somewhere\_on\_your\_\target.
   .2 work \DTcomment{mounted on \host per \cod{sshfs}}.
  }
 \end{block}
\end{frame}


\begin{frame}{Entwicklung}{wo ist was}
 \begin{block}{\host}
  \begin{itemize}
   \item Toolchain/TargetRoot \url{http://sourceforge.net/projects/fhnw-tinl/files/}
   \item Beispiele: \cod{src/*}
   \item Herstellung: \cod{make  {\em the-app}}
  \end{itemize}
 \end{block}
 \begin{block}{\target}
  \begin{itemize}
   \item Runtime \linux POSIX
  \end{itemize}
 \end{block}
\end{frame}

\subsection{Die Schichten}
\begin{frame}{POSIX $\to$ Kernel}
 \fig{libc.pdf}{0.75}{0}
  \begin{description}[SysCalls]
   \item[POSIX] \cod{stdio.h} \& Co
   \item[SysCalls] $\to$  \cod{\targetroot/usr/include/syscall.h}
  \end{description}
\end{frame}

\subsection{Libraries}
\begin{frame}{Bibliotheken}{am Beispiel \cod{hello-world-c.c}}
\begin{itemize}
 \item Der Objectfile \cod{hello-world-c.o}
 \begin{itemize}
  \item Der Code   \cod{objdump -d hello-world-c.o},
  \item Die Symbole \cod{readelf -s hello-world-c.o}
 \end{itemize}
 \item Das Image \cod{hello-world-c}
 \begin{itemize}
  \item Der Code \cod{objdump -d hello-world-c}
  \item Die Symbole \cod{readelf -s hello-world-c.o}
 \end{itemize}
 \item \cod{puts} 
 \begin{itemize}
  \item ist in einer Bibliothek
 \end{itemize}
\end{itemize}

\end{frame}

\begin{frame}{Statische/Dynamische Bibliothek}{Kopie vs. Referenz}
 \begin{columns}
  \begin{column}{5cm}
   \begin{block}{Static}
   \begin{itemize}
    \item \hfill \vspace{-3mm}\fig{static.pdf}{0.5}{0}
    \item {\Large fr�hes} Binden
   \end{itemize}
   \end{block}
  \end{column}
  \begin{column}{5cm}
   \begin{block}{Dynamic}
    \begin{itemize}
    \item\hfill \vspace{-3mm}\fig{dynamic.pdf}{0.5}{0}
    \item {\Large sp�tes} Binden
    \end{itemize}
   \end{block}
  \end{column}
 \end{columns}
 \begin{block}{}
  \fig{legend.pdf}{0.5}{0}
 \end{block}
\end{frame}

\subsection{Aufgaben}
\begin{frame}{Entwicklungsumgebung}
 \begin{itemize}
  \item Entwicklungsumgebung aufsetzen
  \item Erste Programme
        \begin{itemize}
	 \item \cod{hello-world-c.c}%,\cod{hello-world-cpp.cc}
	\end{itemize}
%  \item Benchmark
%        \begin{itemize}
%	 \item \cod{primes.cc} auf \target und \host
%	\end{itemize}
  \item Minimale Programme
        \begin{itemize} 
         \item \cod{direct-call.S}
	 \item \cod{minimal-1.c} und \cod{minimal-2.c} Makefile anpassen
	\end{itemize}	
 \end{itemize}
\end{frame}

\begin{frame}{Statische/Dynamische Bibliothek}
 \begin{itemize}
  \item Die Programme 
  \begin{itemize}
   \item dynamisch linken 
   \item statisch linken
  \end{itemize}
  und vergleichen
  \begin{itemize}
   \item Gr�sse
   \item \cod{objdump}
   \item \cod{readelf}
  \end{itemize}
 \end{itemize}
\end{frame}
