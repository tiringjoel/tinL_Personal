%-------------------------
%minimal-unix
%(c) H.Buchmann FHNW 2014
%export TEXINPUTS=.:${HOME}/fhnw/edu/:${HOME}/fhnw/edu/tinL/config/latex:${HOME}/fhnw/edu/config//:
%-------------------------
\documentclass{beamer}
\usepackage{latex/beamer}
%---------------------
%local defines
%(c) H.Buchmann FHNW 2009
%$Id$
%---------------------
\newcommand{\target} {\beaglebone\xspace}
\newcommand{\targetS}{{\bf BBG}\xspace}
\newcommand{\host}   {{\em Host}\xspace}
\newcommand{\targetroot} {{\bf target-root}\xspace}
\newcommand{\kernel} {{\bf kernel}\xspace}
\renewcommand{\c}{{\bf C}\xspace}
\newcommand{\cpp}{{\bf C++}\xspace}
\newcommand{\posix}{{\bf POSIX}\xspace}

\input{/home/buchmann/latex/dirtree/dirtree.tex}

\usepackage[absolute]{textpos}
\setlength{\TPHorizModule}{1mm}
\setlength{\TPVertModule}{1mm}

\begin{document}

\newcommand{\md}{\cod{md-bbb-{\em version}.img}}
\newcommand{\mdev}{\cod{md-bbb-devel-{\em version}.tar.gz}}

\title{Tools}

\frame{\titlepage}

\begin{frame}{Ein paar wichtige Tools}
 \begin{itemize}
  \item \cod{dd}
  \begin{itemize}
   \item Kopiert rohe {\em bitstreams}
%   \item \cod{dd if=... of=... count=..}
  \end{itemize}
  \item \cod{fdisk}
  \begin{itemize}
   \item Ezeugt Partitionen
  \end{itemize}
  \item Cient \cod{ssh}/Server \cod{sshd}
  \begin{itemize}
   \item sichere Verbindung zwischen Rechnern
  \end{itemize}
  \item \cod{minicom}
  \begin{itemize}
   \item direkte unsichere serielle Verbindung 
  \end{itemize}
  \item \cod{rsync}
  \begin{itemize}
   \item Update/Backup 
  \end{itemize}
  \item Client \cod{tftp}/Server \cod{tftpd}
  \begin{itemize}
   \item elementare schneller Transfer von Files
  \end{itemize}
 \end{itemize}
\end{frame}

\begin{frame}[fragile]{{\tt rsync}}{\url{https://rsync.samba.org/}}
 \begin{itemize}
  \item effiziener Transfer von vielen Files
  \item typisch
\begin{lstlisting}{language=bash}
rsync -av rootfs/ path-to-sd-card/
\end{lstlisting}
 \end{itemize}
\end{frame}

\begin{frame}{\tt tftp}{{\bf t}rivial {\bf f}ile {\bf t}ransfer {\bf p}rotocol}
 \begin{itemize}
  \item Client-Server
  \item Client 
  \begin{itemize}
   \item als {\em shell}
   \item als einzelnes Kommando 
\lstinputlisting[language=bash]{../tools/tftp-put.sh}   
  \end{itemize}
   \item Server
\lstinputlisting[language=bash]{../tools/tftp-server.sh}   
 \end{itemize}
\end{frame}
\end{document}
