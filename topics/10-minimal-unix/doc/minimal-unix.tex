%-------------------------
%minimal-unix
%(c) H.Buchmann FHNW 2014
%export TEXINPUTS=${HOME}/fhnw/edu/:${HOME}/fhnw/edu/tinL/config/latex:${HOME}/fhnw/edu/config//:
%-------------------------
\documentclass{beamer}
\usepackage{latex/beamer}
%---------------------
%local defines
%(c) H.Buchmann FHNW 2009
%$Id$
%---------------------
\def \target {\raspberry\xspace}
\def \host {{\em Host \xspace}}

\input{/home/buchmann/latex/dirtree/dirtree.tex}

\usepackage[absolute]{textpos}
\setlength{\TPHorizModule}{1mm}
\setlength{\TPVertModule}{1mm}

\begin{document}

\newcommand{\qemu}{{\em qemu}\xspace}
\newcommand{\busybox}{{\em busybox}\xspace}
\title{Minimales \unix}

\frame{\titlepage}

\begin{frame}{Um was geht es ?}{Minimales \unix}
\begin{itemize}
 \item {\em minimal}: ein einziger Prozess
 \item \busybox ein kleines \unix
 \item ein kleines \unix aus einem grossen \unix
\end{itemize}
\remark{Grosse \unix gibt es viele
\begin{itemize}\item kleine weniger\end{itemize}}
\end{frame}


\section{Minimal}
\begin{frame}[fragile]{Single Process}{ein einfaches \cod{init}}
\label{Minimal}
 \begin{lstlisting}
#include <unistd.h>

int main(int argc,char** args)
{
 static const char MSG[]="Hello World from '"
                  __FILE__ "'\t '" __DATE__ "'\n";
 write(STDOUT_FILENO,MSG,sizeof(MSG));
 while(1){}
}
 \end{lstlisting}
\end{frame}

\begin{frame}{Was ist zu tun ?}
 \begin{itemize}
  \item \cod{minimal-1} herstellen
  \begin{itemize}
   \item siehe \cod{devel}
   \item \cod{-static} ohne dynamische Bibliotheken
  \end{itemize}
  \item Kleines \cod{root-fs} mit nur einem File
  \begin{itemize}
   \item \cod{minimal-1}
   \remark{versuche {\em extended} Partitionen}
  \end{itemize}
  auf SD-Karte
  \item {\em uboot} 
  \begin{itemize}
   \item \cod{setenv bootargs '{}...{}'}
  \end{itemize}
  \item starten   
 \end{itemize}
 \remark{ein sehr kleines aber vollwertiges \unix}
\end{frame}

\section{\busybox}
\begin{frame}{\busybox}{ist ein kleines \unix}
\begin{itemize}
 \item \url{https://github.com/raspberrypi/target_fs}
 \item Herstellung
 \begin{itemize}
  \item  Siehe \hyperref[Minimal]{Minimal \ref*{Minimal}}
 \end{itemize}
\end{itemize}
\end{frame}

\section{Von gross zu klein}
\begin{frame}{\cod{initrd}}{Initiale RAM Disk}
 \begin{itemize}
  \item enth�lt ein kleines \unix
  \item entsteht aus einem grossen \unix
  \item spezielles Fileformat: cpio
 \end{itemize}
\end{frame}

\begin{frame}{Aufgabe}{grosses \unix $\to$ \cod{initrd} $\to$ kleines \unix}
 \begin{itemize}
  \item erzeuge \cod{initrd} 
  \begin{itemize}
   \item mit \cod{mkinitcpio}
  \end{itemize}
  \item erzeuge {\em target-root} aus \cod{initrd} 
  \begin{itemize}
   \item mit \cod{cpio}
  \end{itemize}
  \item erzeuge {\em partition} auf {\em SD-Card}
  \item setze \cod{bootargs}
  \begin{itemize}
   \item im \cod{u-boot}
  \end{itemize}
 \end{itemize}
\end{frame}

\begin{frame}[fragile]{cpio}{eine Anwendung von {\em pipes}}
 \begin{block}{create}
   \begin{lstlisting}[language=bash]
find . | cpio --create  | gzip --stdout  - > file
   \end{lstlisting}
\vspace{-4mm}   
   \begin{description}
    \item[find] erzeugt Fileliste
    \item[cpio] alle Files in einen {\em stream}
    \item[gzip] Komprimation in {\em archive}
   \end{description}
 \end{block}
 \begin{block}{extract}
   \begin{lstlisting}[language=bash]
gzip -d -c  path-to-archive | cpio --extract
   \end{lstlisting}
\vspace{-4mm}   
   \begin{description}
    \item[gzip] Dekomprimiert archive in {\em stream}
    \item[cpio] erzeugt files aus {\em stream}
   \end{description}
 \end{block}
\end{frame}

\end{document}
