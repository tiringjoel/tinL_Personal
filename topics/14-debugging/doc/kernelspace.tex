\section{Kernelspace}
\begin{frame}{Files}
 \begin{itemize}
 \item dmesg
 \end{itemize}
\end{frame}

\subsection{printk}
\begin{frame}{printk}{Das {\em kernel printf}}
 \begin{itemize}
  \item Siehe \cod{6-modules/src/simple-module.c}
 \end{itemize}
\end{frame}

\subsection{strace}
\begin{frame}{\cod{strace}}
 \begin{itemize}
  \item \cod{strace} {\Huge gibt es} {\tiny nat�rlich} {\Huge nicht}
 \end{itemize}
\end{frame}


\subsection{gdb}
\begin{frame}{Zwei Rechner}{Host Target}
 \begin{tabular}{l||c|c|c||l}
  		&	Host	&	& Target\\
  \hline
  seriell	&\cod{/dev/ttyUSB0}	&$\leftrightarrow$& \cod{/dev/ttyS} &debug\\
  ethernet	&\cod{eth0}		&$\leftrightarrow$& \cod{eth0} &Bedienung\\
 \end{tabular}
 
 \remark{die serielle Schnittstelle kann nicht f�r die Bedienung gebraucht werden}
\end{frame}

\begin{frame}{Konfiguration}
 \begin{description}
  \item[Target] Kernel hacking $\to$ KGDB kernel debugger
  \item[Host] gdb f�r ARM
  \begin{description}[configuration]
   \item[download]  \url{www.gnu.org/software/gdb/}
   \item[configuration]
    \cod{configure --prefix={\em where-on-host} $\backslash$\\
     --target=arm-none-linux-gnueabi}
  \end{description}
 \end{description}
\end{frame}

\begin{frame}{Anwendung}
 \begin{itemize}
  \item Siehe \url{DocBook/index.html} kgdb
 \end{itemize}
\end{frame}

