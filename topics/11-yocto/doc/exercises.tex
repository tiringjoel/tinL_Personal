\section{Aufgaben}
\begin{frame}{\yocto}{einrichten}
 \begin{itemize}
  \item {\tiny \url{https://www.yoctoproject.org/downloads}}
  \item {\tiny\url{http://www.yoctoproject.org/docs/1.7/yocto-project-qs/yocto-project-qs.html}}
 \end{itemize}
\end{frame}

\begin{frame}{\yocto}{Predefined}
 \begin{itemize}
  \item {\tiny\url{http://www.yoctoproject.org/docs/1.7/yocto-project-qs/yocto-project-qs.html}}
  \begin{itemize}
   \item Using Pre-Built Binaries and QEMU
  \end{itemize}
  \item die Files:
  \begin{itemize}
   \item \cod{bzImage-qemux86.bin} \kernel
   \item \cod{core-image-minimal-dev-qemux86-*.rootfs.*} rootfs
  \end{itemize}
  \item \cod{qemu-system-i386} die Parameter:
  \begin{itemize}
   \item \cod{-kernel bzImage-qemux86.bin}
   \item \cod{-append "{}root=/dev/hda console=ttyS0"{}}
  \end{itemize}
 \end{itemize} 
\end{frame}

\begin{frame}{\yocto}{Erstes Target}
\begin{itemize}
 \item bitbake core-image-minimal
 \item Graphisches {\em backend}
 \begin{itemize}
  \item \cod{hob}
 \end{itemize}
\end{itemize}
\end{frame}

\begin{frame}{\yocto}{Raspberry}
 \begin{itemize}
 \item \cod{git clone git://git.yoctoproject.org/meta-raspberrypi}
 \item Konfiguration
 \item \cod{bitbake -k rpi-basic-image}
 \end{itemize}
 \remark{Fertiges Produkt:
 \begin{itemize}
  \item \cod{raspi-yocto.tar.gz}
  \item auf\\
 {\url{https://sourceforge.net/projects/fhnw-tinl/files/}}
 \end{itemize}
 }
\end{frame}
