\section{\yocto}
\begin{frame}{\yocto}{\url{https://www.yoctoproject.org}}
 \begin{block}{Gegeben}
  \vspace{-3mm}
  \begin{itemize}
   \item eine Hardware
   \item korrekte\footnote{meistens} Sourcefiles 
   \begin{itemize}
    \item haupts�chlich \C
    \item verstreut auf der Welt
    \item selber geschrieben
   \end{itemize}
  \end{itemize}
 \end{block}
 \vspace{-5mm}
 \begin{block}{Gesucht}
  \vspace{-3mm}
  \begin{itemize}
   \item ein Produkt
   \begin{itemize}
    \item z.B. ein \linux basiertes eingebettetes System
   \end{itemize}
  \end{itemize}
 \end{block}
 \vspace{-5mm}
 \begin{block}{L�sungsweg}
  \vspace{-3mm}
  \begin{itemize}
    \item \cod{bitbake {\em theProduct}}
  \end{itemize}
 \end{block}
\end{frame}

\begin{frame}{Wichtige Begriffe}
 \begin{itemize}
  \item Package
  \begin{itemize}
   \item Abh�ngigkeiten
  \end{itemize}
  \item Konfiguration
  \begin{itemize}
   \item die richtige Wahl
  \end{itemize}
  \item Versionen
  \begin{itemize}
   \item z.B. (und vor allem) \cod{git}
  \end{itemize}
  \item{Image}
  \begin{itemize}
   \item das Endprodukt
  \end{itemize}
 \end{itemize}
\end{frame}

\begin{frame}{Wichtige Begriffe}
 \begin{itemize}
  \item Layer
  \begin{itemize}
   \item die Verzeichnisse \cod{meta-*}
  \end{itemize}
  \item SDK Software Development Kit
  \begin{itemize}
   \item die {\em toolchain}
  \end{itemize}
  \item \cod{bitbake}
  \begin{itemize}
   \item das Arbeitspferd
  \end{itemize}
  \item Rezept {\em recipe}
  \begin{itemize}
   \item beschreibt die Herstellung einer Komponente
  \end{itemize}
 \end{itemize}
\end{frame}

\subsection{Demo}
\begin{frame}{Vorbereitung}
{\footnotesize\url{www.yoctoproject.org/docs/1.7/yocto-project-qs/yocto-project-qs.html}}
\begin{itemize}
 \item Installation
 \begin{itemize}
   \item \see{}Getting the Yocto Project
 \end{itemize}
 \item Setze korrekte Umgebung 
  \begin{itemize}
    \item \cod{. oe-init-build-env} 
  \end{itemize}
  \item  Starte
   \begin{itemize}
     \item \cod{bitbake -k core-image-minimal}
     \item braucht Zeit
     \item braucht $\approx 40 GiB$ Festplatte
   \end{itemize}
\end{itemize} 
\end{frame}

\begin{frame}{Das Resultat}{als virtuelle Machine}
\begin{block}{Die Files:}
 \begin{block}{\kernel}
  \cod{bzImage-qemux86.bin}
 \end{block}
 \begin{block}{\unix}
  \cod{core-image-minimal-dev-qemux86-20141018183055.rootfs.ext3}
 \end{block}
\end{block}
\begin{block}{Starten}
 \cod{runqemu qemux86 {\em kernel} {\em unix}  serial}
\end{block}
\remark{predefined:\\ 
{\tiny\url{http://downloads.yoctoproject.org/releases/yocto/yocto-1.7/machines/qemu/qemux86/}}}
\end{frame}
