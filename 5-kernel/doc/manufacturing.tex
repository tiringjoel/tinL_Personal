\section{Herstellung}

\subsection{Download}
\begin{frame}{\url{github.com/beagleboard/linux}}
             {Mehrere M�glichkeiten}
 \begin{itemize}
  \item das ganze git repository
  \item {$\to$ \Large nur die letzten $n$ Versionen \cod{--depth=$n$}}
  \begin{itemize}
   \item \cod{git clone -b4.19 --depth=4 path\_to\_repository}
  \end{itemize}
  \item zip File 
 \end{itemize}
\end{frame}

\subsection{Setup}
\begin{frame}{Tools}%{Siehe \cod{4-devel}}
 \begin{description}[toolchain]
 
  \item[toolchain] {\small \url{sourceforge.net/projects/fhnw-tinl/files}}
   \begin{itemize}
    \item \cod{tc-tinl-gcc-8.1.0-2018.05.21.tar.gz}
    \begin{itemize}
     \item auf \url{drive.switch.ch/index.php/s/A6H382zEGDrgfAL}
    \end{itemize}
    \item Prefix: \cod{arm-linux-gnueabihf-}
    \begin{itemize}
     \item beschreibt:
     \begin{itemize}
      \item Architektur: \cod{armv7}
      \item {\bf A}pplication {\bf B}inary {\bf I}nterface: \cod{gnueabihf}
     \end{itemize}
    \end{itemize}
    \end{itemize}
  \item[make] Normales \cod{make}
  \begin{itemize}
   \item \kernel Herstellung:
   \begin{itemize}
    \item \cod{make {\em cmd}}
   \end{itemize}
  \end{itemize}
 \end{description}
\end{frame}

\begin{frame}{Wo ist was ?}
 \dirtree{%
  .1 tinL.
  .2 5-kernel.
  .3 build \DTcomment{generated kernel files}.
  .4 {.config} \DTcomment{die aktuelle Konfiguration}.
  .3 tools \DTcomment{for making}.
  .4 kernel.sh \DTcomment{wrapper to \kernel Makefile}.
  .3 config.
  .4 config.sh \DTcomment{for kernel.sh}.
  .4 kernel.config \DTcomment{'gute' kernel Konfiguration}.
  .3 tc \DTcomment{link to toolchain}.
  .2 resources.
  .3 beaglebone-black.
  .4 linux \DTcomment{the source tree}.
 }
\end{frame}

\subsection{Herstellung}
\begin{frame}{Erste Konfiguration}
 \begin{itemize}
  \item Hilfe
  \begin{itemize}
   \item \cod{./tools/kernel.sh help}
  \end{itemize}
  \item Vordefinierte Konfiguration
  \begin{itemize}
   \item \cod{./tools/kernel.sh bb.org\_defconfig}
  \end{itemize}
  \item Anpassung der Konfiguration
  \begin{itemize}
  \item \cod{./tools/kernel.sh menuconfig}
  \item \cod{./tools/kernel.sh xconfig}
   \end{itemize}
 \end{itemize}
\end{frame}

\begin{frame}{Kompilation}
 \begin{itemize}
  \item \cod{./tools/kernel.sh zImage}
  \begin{itemize}
   \item erzeugt  \cod{build/arch/arm/boot/zImage}
  \end{itemize} 
  \item \cod{./tools/kernel.sh dtbs}
  \begin{itemize}
   \item erzeugt \cod{build/arch/arm/boot/dts/am335x-boneblack-wireless.dtb}
   {\em Devicetree} 
   \remark{{\em Devicetree} sp�ter behandelt}
  \end{itemize} 
 \end{itemize}
\end{frame}

\begin{frame}{Installation}{auf SD-Card}
 \begin{itemize}
  \item Kopiere
  \begin{description}[Devicetree]
   \item[Image] \cod{build/arch/arm/boot/zImage}
   \item[Devicetree] \cod{build/arch/arm/boot/dts/am335x-boneblack-wireless.dtb}
  \end{description}
  auf
  \begin{itemize}
   \item SD-Card Partition 2:Root File System
   \begin{minipage}{8cm}
   \dirtree{%
    .1 / \DTcomment{Partition 2}.
    .2 boot.
    .3 zImage.
    .3 am335x-boneblack-wireless.dtb.
   }
   \end{minipage}
  \end{itemize}
 \end{itemize}
\end{frame}


