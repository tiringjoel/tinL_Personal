\section{Herstellung}

\subsection{Download}
\begin{frame}{\url{https://github.com/beagleboard/linux}}
             {Mehrere M�glichkeiten}
 \begin{itemize}
  \item das ganze git repository
  \item nur die letzten $n$ Versionen \cod{--depth=$n$}
  \item zip File 
 \end{itemize}
\end{frame}

\subsection{Setup}
\begin{frame}{Tools}{Siehe \cod{4-devel}}
 \begin{description}[toolchain]
 
  \item[toolchain] {\small \url{https://sourceforge.net/projects/fhnw-tinl/files}}
   \begin{itemize}
    \item \cod{beaglebone-black-toolchain-64bit.tar.bz2}
    \item Prefix: \cod{arm-linux-gnueabihf-}
    \begin{itemize}
     \item beschreibt:
     \begin{itemize}
      \item Architektur: \cod{armv7}
      \item {\bf A}pplication {\bf B}inary {\bf I}nterface: \cod{gnueabihf}
     \end{itemize}
    \end{itemize}
    \end{itemize}
  \item[make] Normales \cod{make}
  \begin{itemize}
   \item \kernel Herstellung:
   \begin{itemize}
    \item \cod{make {\em cmd}}
   \end{itemize}
  \end{itemize}
 \end{description}
\end{frame}

\begin{frame}{Wo ist was ?}
 \dirtree{%
  .1 tinL.
  .2 5-kernel.
  .3 build \DTcomment{generated kernel files}.
  .4 {.config} \DTcomment{die aktuelle Konfiguration}.
  .3 tools \DTcomment{for making}.
  .4 kernel.sh \DTcomment{wrapper to \kernel Makefile}.
  .3 config.
  .4 config.sh \DTcomment{for kernel.sh}.
  .4 kernel.config \DTcomment{'gute' kernel Konfiguration}.
  .2 resources.
  .3 beaglebone-black.
  .4 linux \DTcomment{the source tree}.
 }
\end{frame}

\subsection{Herstellung}
\begin{frame}{Erste Konfiguration}{\cod{sh kernel.sh help}}
 \begin{itemize}
  \item \cod{sh tools/kernel.sh bb.org\_defconfig}
  \begin{itemize}
   \item Vordefinierte Konfiguration
  \end{itemize}
  \item \cod{sh tools/kernel.sh.sh menuconfig}
  \begin{itemize}
   \item Anpassung der Konfiguration
  \end{itemize}
 \end{itemize}
\end{frame}

\begin{frame}{Kompilation}
 \begin{itemize}
  \item \cod{sh tools/kernel.sh zImage}
  \begin{itemize}
   \item erzeugt  \cod{build/arch/arm/boot/zImage}
  \end{itemize} 
  \item \cod{sh tools/kernel.sh dtbs}
  \begin{itemize}
   \item erzeugt \cod{build/arch/arm/boot/dts/am335x-boneblack-wl1835mod.dtb}
   {\em Devicetree} 
   \remark{{\em Devicetree} sp�ter behandelt}
  \end{itemize} 
 \end{itemize}
\end{frame}

\begin{frame}{Installation}{auf SD-Card}
 \begin{itemize}
  \item Kopiere
  \begin{description}[Devicetree]
   \item[Image] \cod{build/arch/arm/boot/zImage}
   \item[Devicetree] \cod{build/arch/arm/boot/dts/am335x-boneblack.dtb}
  \end{description}
  auf
  \begin{itemize}
   \item SD-Card {\em boot-partition}
  \end{itemize}
 \end{itemize}
\end{frame}


