%-------------------------
%big-picture
%(c) H.Buchmann FHNW 2008
%$Id$
%export TEXINPUTS=${HOME}/fhnw/edu/:${HOME}/fhnw/edu/tinL/config/latex:${HOME}/fhnw/edu/config//:
%-------------------------
\documentclass{beamer}
\usepackage{latex/beamer}
%---------------------
%local defines
%(c) H.Buchmann FHNW 2009
%$Id$
%---------------------
\def \target {\raspberry\xspace}
\def \host {{\em Host \xspace}}

\input{/home/buchmann/latex/dirtree/dirtree.tex}

\title{Kernel}
\begin{document}

\frame{\titlepage}

\begin{frame}{Ziele}{Neuer \kernel auf \target}
 \begin{itemize}
  \item Download
  \item Setup
  \item Konfiguration
  \item Kompilation
  \item Installation
 \end{itemize}
\end{frame}

\section{The Big Picture}
\begin{frame}{The Big Picture}{grosses Projekt}
 \begin{description}
  \item[Gegeben] Eine grosse Anzahl {\em source} Files
  \item[Gesucht] 2 Files:
   \begin{itemize}
    \item das {\Large Image}
    \item der {\em Devicetree}
   \end{itemize}
  \item[L�sung] Klassisches Verfahren
  \begin{itemize}
   \item Toolchain
   \item Makefile
  \end{itemize}
 \end{description}
\end{frame}

\begin{frame}{Die Schichten}
 \begin{center}
  \includegraphics[width=0.75\textwidth]{layers.pdf}
 \end{center}
\end{frame}

\begin{frame}{Kernel}{Grosses Projekt}
\begin{block}{Was ist einfach ?}
 \begin{itemize}
  \item \kernel h�ngt nicht von anderen Software Komponenten ab
  \begin{itemize}
   \item stand alone
  \end{itemize}
  \item Braucht nur \cod{make} und {\em toolchain}
 \end{itemize}
\end{block}
\begin{block}{Was ist schwierig ?}
 \begin{itemize}
  \item Konfiguration
  \begin{itemize}
   \item Wahl der richtigen {\em source} Files f�r das {\Large Image}
  \end{itemize}
 \end{itemize}
\end{block}
\end{frame}



\section{Herstellung}

\subsection{Download}
\begin{frame}{\url{github.com/beagleboard/linux}}
             {Mehrere M�glichkeiten}
 \begin{itemize}
  \item das ganze git repository
  \item nur die letzten $n$ Versionen \cod{--depth=$n$}
  \item zip File 
 \end{itemize}
\end{frame}

\subsection{Setup}
\begin{frame}{Tools}%{Siehe \cod{4-devel}}
 \begin{description}[toolchain]
 
  \item[toolchain] {\small \url{sourceforge.net/projects/fhnw-tinl/files}}
   \begin{itemize}
    \item \cod{tc-tinl-gcc-8.1.0-2018.05.21.tar.gz}
    \begin{itemize}
     \item auf \url{drive.switch.ch/index.php/s/A6H382zEGDrgfAL}
    \end{itemize}
    \item Prefix: \cod{arm-linux-gnueabihf-}
    \begin{itemize}
     \item beschreibt:
     \begin{itemize}
      \item Architektur: \cod{armv7}
      \item {\bf A}pplication {\bf B}inary {\bf I}nterface: \cod{gnueabihf}
     \end{itemize}
    \end{itemize}
    \end{itemize}
  \item[make] Normales \cod{make}
  \begin{itemize}
   \item \kernel Herstellung:
   \begin{itemize}
    \item \cod{make {\em cmd}}
   \end{itemize}
  \end{itemize}
 \end{description}
\end{frame}

\begin{frame}{Wo ist was ?}
 \dirtree{%
  .1 tinL.
  .2 5-kernel.
  .3 build \DTcomment{generated kernel files}.
  .4 {.config} \DTcomment{die aktuelle Konfiguration}.
  .3 tools \DTcomment{for making}.
  .4 kernel.sh \DTcomment{wrapper to \kernel Makefile}.
  .3 config.
  .4 config.sh \DTcomment{for kernel.sh}.
  .4 kernel.config \DTcomment{'gute' kernel Konfiguration}.
  .3 tc \DTcomment{link to toolchain}.
  .2 resources.
  .3 beaglebone-black.
  .4 linux \DTcomment{the source tree}.
 }
\end{frame}

\subsection{Herstellung}
\begin{frame}{Erste Konfiguration}
 \begin{itemize}
  \item Hilfe
  \begin{itemize}
   \item \cod{./tools/kernel.sh help}
  \end{itemize}
  \item Vordefinierte Konfiguration
  \begin{itemize}
   \item \cod{./tools/kernel.sh bb.org\_defconfig}
  \end{itemize}
  \item Anpassung der Konfiguration
  \begin{itemize}
  \item \cod{./tools/kernel.sh menuconfig}
  \item \cod{./tools/kernel.sh xconfig}
   \end{itemize}
 \end{itemize}
\end{frame}

\begin{frame}{Kompilation}
 \begin{itemize}
  \item \cod{./tools/kernel.sh zImage}
  \begin{itemize}
   \item erzeugt  \cod{build/arch/arm/boot/zImage}
  \end{itemize} 
  \item \cod{./tools/kernel.sh dtbs}
  \begin{itemize}
   \item erzeugt \cod{build/arch/arm/boot/dts/am335x-boneblack-wireless.dtb}
   {\em Devicetree} 
   \remark{{\em Devicetree} sp�ter behandelt}
  \end{itemize} 
 \end{itemize}
\end{frame}

\begin{frame}{Installation}{auf SD-Card}
 \begin{itemize}
  \item Kopiere
  \begin{description}[Devicetree]
   \item[Image] \cod{build/arch/arm/boot/zImage}
   \item[Devicetree] \cod{build/arch/arm/boot/dts/am335x-boneblack-wireless.dtb}
  \end{description}
  auf
  \begin{itemize}
   \item SD-Card Partition 2:Root File System
   \begin{minipage}{8cm}
   \dirtree{%
    .1 / \DTcomment{Partition 2}.
    .2 boot.
    .3 zImage.
    .3 am335x-boneblack-wireless.dtb.
   }
   \end{minipage}
  \end{itemize}
 \end{itemize}
\end{frame}




\section{Startup}

\begin{frame}{Startup}{Bootloaders bei eingebetteten Systemen}
\begin{tabular}{ll|l}
 Reset $\to$ & fbl 		& {\em first stage bootloader}\\
 			 & sbc  	& ev. weitere bootloader {\em second stage bootloader}\\
			 & u-boot	& Hier haben wir Zugriff\\
			 & kernel	& Konfiguration/Parametrisierung \\
			 & linux	& 
\end{tabular}
\end{frame}

\begin{frame}{Startup}{Bootloader beim \host}
\begin{tabular}{ll|l}
 Reset $\to$ & BIOS 	& {\em first stage bootloader}\\
 			 & grub  	& {\em second stage bootloader}\\
			 & kernel	& Konfiguration/Parametrisierung \\
			 & linux	& 
\end{tabular}
\end{frame}

\subsection{U-Boot}
\begin{frame}{\url{www.denx.de/wiki/U-Boot/WebHome}}{ein typischer Bootloader f�r eingebettete Systeme}
 \begin{itemize}
  \item Kommandozeilen
  \item Verbindung zum \host via RS232/USB
  \begin{itemize}
   \item \host: \cod{minicom -D /dev/ttyUSB{\em N}}, $N=0,1..$
   \item 115200 Baud \cod{8N1}
   \item {\Large no} Handshaking
  \end{itemize}
  \item Kopiert Daten von
  \begin{itemize}
   \item SD-Karten
   \item Netz
  \end{itemize}
  in das Memory vom \targetS
 \end{itemize}
\end{frame}

\begin{frame}{Ein paar typische Befehle}
 \begin{itemize}
  \item \cod{help}
  \item \cod{printenv} Zeigt die Umgebung
  \item \cod{md addr} Memory display
  \item \cod{fatls mmc p}  vfat sd-card partition p
  \item \cod{fatload mmc p memAddr file}
  \item \cod{tftpboot [loadAddress] [[hostIPaddr:]bootfilename]} 
  \item \cod{bootz kernelAddr - fdt}
 \end{itemize}
\end{frame}


\begin{frame}{U-Boot Bedienung}{Siehe \cod{5-kernel/tools/u-boot.cmd}}
 \begin{itemize}
  \item copy paste
  \remark{kann sich �ndern}
 \end{itemize}
\end{frame}

\subsection{Kernel}
\begin{frame}[fragile]{�bergang U-Boot-Kernel}{Bootargs}
{\scriptsize
\begin{verbatim}
 bootargs 
 'root=/dev/mmcblk0p2 rw rootdelay=1 init=linuxrc console=ttyO0,115200n8'
\end{verbatim}
}
\begin{itemize}
 \item \cod{kernel-source/Documentation/kernel-parameters.txt}
\end{itemize}
\end{frame}

\begin{frame}[fragile]{Die Kernel messages}{}
{\scriptsize
\begin{verbatim}
Starting kernel ...

Booting Linux on physical CPU 0x0
Initializing cgroup subsys cpuset
Initializing cgroup subsys cpu
Initializing cgroup subsys cpuacct
...
Kernel command line: root=/dev/mmcblk0p2 rw rootdelay=1 init=linuxrc console=ttyO0,115200n8
...
Waiting 1 sec before mounting root device...
EXT4-fs (mmcblk0p2): couldn't mount as ext3 due to feature incompatibilities
EXT4-fs (mmcblk0p2): couldn't mount as ext2 due to feature incompatibilities
EXT4-fs (mmcblk0p2): mounted filesystem with ordered data mode. Opts: (null)
VFS: Mounted root (ext4 filesystem) on device 179:2.
\end{verbatim}
}
\end{frame}


\begin{frame}{Workflow}{Notationen}
 \begin{description}[\cod{target-root-{\em V}.tar.gz}]
  \item[\cod{\em sd-card}] die Partition vom rootfs auf der SD Karte
  \item[\cod{target-root-{\em V}.tar.gz}] das heruntergeladene rootfs
  \item[\cod{\em target-root}] das rootfs von \targetS auf dem \host
 \end{description}
\end{frame}

\begin{frame}{Workflow}{schrittweise Verbesserung}
 \begin{enumerate}
  \item Initialer Download \cod{target-root-{\em V}.tar.gz}
  \item \cod{target-root}
  \begin{itemize}
   \item \cod{tar -xf target-root-{\em V}.tar.gz -C {\em target-root}}
  \end{itemize} 
  \item Transfer auf \cod{\em sd-card}
  \begin{itemize}
   \item \cod{rsync -av {\em target-root}/ {\em sd-card}/}
   \item \cod{sync} 
  \end{itemize} 
  \item Test/Konfiguration auf dem \targetS
  \item Update auf dem \host
  \begin{itemize}
   \item \cod{rsync -av {\em sd-card}/ {\em target-root}/}
  \end{itemize} 
  \item $\to$ 4
 \end{enumerate}
\end{frame}

\begin{frame}{Die Files}
 \begin{block}{Partition 1: vfat}
  \begin{itemize}
   \item \cod{MLO}
   \item \cod{u-boot.img}
   \item \cod{zImage}
   \item \cod{am335x-boneblack-wireless.dtb}
  \end{itemize}
 \end{block}
% \vspace{-4mm}
 \begin{block}{Partition 1: ext4}
  \begin{itemize}
   \item rootfs auf dem \host
   \item \cod{rsync -av target-root/ {\em sd-card/}}
   \item \cod{sync}
  \end{itemize}
 \end{block}
 
\end{frame}



\section{}
\begin{frame}{Start}{auf \target}
 \begin{itemize}
  \item u-boot
  \begin{itemize}
   \item UART-USB Kabel
  \end{itemize}
  \begin{itemize}
   \item Befehle in \cod{scripts/u-boot.cmd}
  \end{itemize}
 \end{itemize}
\end{frame}

\begin{exercise}{\kernel}
 \begin{itemize}
  \item \target {\em default} Konfiguration
  \begin{itemize}
   \item herstellen
   \item auf SD-Karte
   \item ausprobieren
  \end{itemize}
  \item Die {\em default} Konfiguration �ndern:
  \begin{itemize}
   \item nur eine CPU
   \item keine ALSA Soundkarte
   \item ... 
  \end{itemize}
%  \item f�ge Treiber f�r WLAN hinzu
 \end{itemize}
\end{exercise}

\end{document}

